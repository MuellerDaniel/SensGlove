%% LyX 2.0.2 created this file.  For more info, see http://www.lyx.org/.
%% Do not edit unless you really know what you are doing.
\documentclass[12pt,a4paper,english,intoc,bibliography=totoc,index=totoc,BCOR10mm,captions=tableheading,titlepage,fleqn]{scrbook}
\usepackage{lmodern}
\renewcommand{\sfdefault}{lmss}
\renewcommand{\ttdefault}{lmtt}

\usepackage[T1]{fontenc}
\usepackage[latin9]{inputenc}
\usepackage{fancyhdr}
\usepackage{listings}
\pagestyle{fancy}
\setcounter{secnumdepth}{3}
\setlength{\parskip}{\medskipamount}
\setlength{\parindent}{0pt}
\usepackage[english]{babel}
\usepackage{amsmath}
\usepackage{amssymb}
\usepackage{graphicx}
\usepackage{subfig}
\usepackage{nomencl}
\usepackage{rotating} 
\usepackage[nohyperlinks, printonlyused]{acronym}
%\usepackage[numbers]{alphanat}
\usepackage{hyperref}
%\usepackage[square,sort,comma]{natbib}
%\usepackage{biblatex}


\usepackage{lipsum}
\usepackage{textcomp}
\usepackage{gensymb}
\usepackage[binary-units]{siunitx}
\usepackage{acronym}
\usepackage{subfiles}
\usepackage{tikzsymbols,xcolor}
\usetikzlibrary{arrows,shapes,positioning,calc}

% the following is useful when we have the old nomencl.sty package
\providecommand{\printnomenclature}{\printglossary}
\providecommand{\makenomenclature}{\makeglossary}
\makenomenclature

%\usepackage[unicode=true,
% bookmarks=true,bookmarksnumbered=true,bookmarksopen=true,bookmarksopenlevel=1,
% breaklinks=false,pdfborder={0 0 0},backref=false,colorlinks=false]
% {hyperref}
%\hypersetup{pdftitle={Ihr Titel},
% pdfauthor={Ihr Name},
% pdfsubject={Arbeit zur Erlangung des Masters der Technischen Fakult�t der Albert-Ludwigs-Universit�t Freiburg im Breisgau},
% pdfkeywords={Masterarbeit},
% pdfpagelayout=OneColumn, pdfnewwindow=true, pdfstartview=XYZ, plainpages=false}

\makeatletter

%%%%%%%%%%%%%%%%%%%%%%%%%%%%%% LyX specific LaTeX commands.
\pdfpageheight\paperheight
\pdfpagewidth\paperwidth


\@ifundefined{date}{}{\date{}}
%%%%%%%%%%%%%%%%%%%%%%%%%%%%%% User specified LaTeX commands.
% Linkfl�che f�r Querverweise vergr��ern und automatisch benennen
\AtBeginDocument{\renewcommand{\ref}[1]{\mbox{\autoref{#1}}}}
\newlength{\abc}
\settowidth{\abc}{\space}
\AtBeginDocument{%
%\addto\extrasngerman{
 %\renewcommand{\equationautorefname}{\hspace{-\abc}}
 %\renewcommand{\sectionautorefname}{Kap.\negthinspace}
 %\renewcommand{\subsectionautorefname}{Kap.\negthinspace}
 %\renewcommand{\subsubsectionautorefname}{Kap.\negthinspace}
 %\renewcommand{\figureautorefname}{Abb.\negthinspace}
 %\renewcommand{\tableautorefname}{Tab.\negthinspace}
%}
}



% f�r den Fall, dass jemand die Bezeichnung "Gleichung" haben will
%\renewcommand{\eqref}[1]{equation~(\negthinspace\autoref{#1})}

% Setzt den Link f�r Spr�nge zu Gleitabbildungen
% auf den Anfang des Gelitobjekts und nicht aufs Ende
\usepackage[figure]{hypcap}

% Die Seiten des Inhaltsverzeichnisses werden r�misch numeriert,
% ein PDF-Lesezeichen f�r das Inhaltsverzeichnis wird hinzugef�gt
\let\myTOC\tableofcontents
\renewcommand\tableofcontents{%
  \frontmatter
  \pdfbookmark[1]{\contentsname}{}
  \myTOC
  %\mainmatter
 }

% make caption labels bold
\setkomafont{captionlabel}{\bfseries}
\setcapindent{1em}


\usepackage{xcolor}
\newcommand{\note}[1]{\textcolor{red}{#1}}

%sections with pagebreak
\newcommand{\mysection}[1]{\newpage\section{#1}}

% erlaubt LaTeX-Berechnungen
\usepackage{calc}

% fancy page header/footer settings
\renewcommand{\chaptermark}[1]{\markboth{#1}{#1}}
\renewcommand{\sectionmark}[1]{\markright{\thesection\ #1}}

% Vergr��ert den Teil der Seite, in dem Gleitobjekte
% unten angeordnet werden d�rfen
\renewcommand{\bottomfraction}{0.5}

% Vermeidet, dass Gleitobjekte vor ihrem Abschnitt gedruckt werden
\let\mySection\section\renewcommand{\section}{\suppressfloats[t]\mySection}

% deutscher Name f�r die Nomenklatur 
\renewcommand{\nomname}{Nomenklatur}

\newcommand\todo[1]{\textcolor{red}{#1}}
\newcommand*{\rom}[1]{\expandafter\@slowromancap\romannumeral #1@}

\makeatother

\setcounter{tocdepth}{4}
\setcounter{secnumdepth}{4}

\begin{document}
%\hypersetup{pageanchor=false}

%\DeclareSIUnit\gauss{G}

\subject{Masterarbeit}


\title{Something with magnets and hand motion...}


\author{Daniel M�ller}


\date{aa.bb.cccc}


\publishers{%\includegraphics{images/Uni_Logo-Grundversion_E1_A4_CMYK.pdf}\vspace{\baselineskip}\\
Albert-Ludwigs-Universit�t Freiburg im Breisgau\\
Technische Fakult�t\\
Embedded Systems Engineering\vspace{-3cm}
}


\uppertitleback{Eingereichte Masterarbeit gem�� den Bestimmungen der Pr�fungsordnung
der Albert-Ludwigs-Universit�t Freiburg f�r den Studiengang Master
of Science (M.\,Sc.) Embedded Systems Engineering vom 3.\,6.\,2014.}


\lowertitleback{\textbf{Bearbeitungszeitraum}\smallskip{}
\\
aa.\,bb.\,20zz -- aa.\,bb.\,20cc \bigskip{}
\\
\textbf{Gutachter}\smallskip{}
\\
Prof.~Dr.~Kristof Van Laerhoven \smallskip{}
\\
Prof.~Dr.~Prof. Dr. Bernd Becker\bigskip{}
\\
\textbf{Betreuer}\smallskip{}
\\
Prof.~Dr.~Kristof Van Laerhoven \smallskip{}
\\
Dipl.\,Inf.~Philipp M. Scholl}


%\dedication{Widmung, Zitat, kluger Spruch oder �hnliches (ist nat�rlich optional)}

\maketitle
\cleardoublepage{}

\pagestyle{empty}

\section*{Erkl�rung}

Hiermit erkl�re ich, dass ich diese Abschlussarbeit selbst�ndig verfasst
habe, keine anderen als die angegebenen Quellen/Hilfsmittel verwendet
habe und alle Stellen, die w�rtlich oder sinngem�� aus ver�ffentlichten
Schriften entnommen wurden, als solche kenntlich gemacht habe. Dar�ber
hinaus erkl�re ich, dass diese Abschlussarbeit nicht, auch nicht auszugsweise,
bereits f�r eine andere Pr�fung angefertigt wurde.

\cleardoublepage{}


\lhead{\rightmark}


\rhead[\leftmark]{}


\lfoot[\thepage]{}


\cfoot{}


\rfoot[]{\thepage}


\pagestyle{plain}

%%\selectlanguage{english}%

\chapter*{Abstract}

\addcontentsline{toc}{chapter}{Abstract} 

\todo{
\begin{itemize}
\item check for consistent spelling of 3D (3d, three dimensional...)
\item check for consistent order of flexion-extension
\item check for consistent order of adduction-abduction
\item writing numbers out or not?
\item do not cut adduction-abduction / flexion-extension $ \rightarrow $ define it as an accronym!
\item centre vs center
\item adjust the line breaks
\item align the formulas in a pretty shape!
\end{itemize}
}



%% !TeX spellcheck = de_DE

%\begin{otherlanguage}{german}

\chapter*{Zusammenfassung}

Die Beobachtung und Rekonstruktion von Bewegungen der menschlichen Hand kann f\"ur moderne Mensch-Computer-Interaktionen oder f\"ur die Unterst\"utzung medizinischer Therapien eingesetzt werden. Moderne Systeme zur Bewegungsverfolgung, die f\"ur diesen Zweck angepasst sind, bringen oft sperrige Ger\"ate mit sich. Die vorliegende Arbeit pr\"asentiert einen Ansatz zur Rekonstruktion von Handbewegungen, indem das Magnetfeld gemessen wird, welches von Permanentmagneten, die sich auf den Fingerspitzen befinden, erzeugt wird. Wenn sich die Finger bewegen, \"andert sich die von den Magneten hervorgerufene magnetische Flussdichte, was von dem aus vier Sensoren bestehenden Sensorarray gemessen werden kann. Diese Daten werden benutzt, um die Fingerpositionen auf der Basis eines kinematischen Handmodels zu sch\"atzen.

Die Magnetsensoren werden in einem Gestell platziert, welches am Handr\"ucken getragen werden kann. Die Sensordaten werden \"uber Bluetooth Low Energy zu einem PC gesendet, auf dem die eigentliche Zustandssch\"atzung durchgef\"uhrt wird. Die Fingerpositionen werden durch die L\"osung eines Optimierungsproblems vorausgesagt. Das zugrunde liegende Modell zur Darstellung der menschlichen Hand beschreibt die Fingerposition durch die Angabe der Gelenkwinkel. Der Daumen wird zur Vereinfachung nicht beachtet.

Die Qualit\"at des entwickelten Systems h\"angt von mehreren Faktoren ab. Der kritischste ist hierbei die exakte Bestimmung der anatomischen Gegebenheiten und der Sensorpositionen. Diese Parameter haben einen nichtlinearen Einfluss auf die Modellgleichungen, welche die Grundlage f\"ur die Zustandssch\"atzung bilden. Diese Werte k\"onnen jedoch nur per Hand mit einem Messschieber gemessen werden, woraus Messfehler resultieren k\"onnen. Weiterhin ist die dynamische Eliminierung des Erdmagnetfeldes nicht m\"oglich. Deshalb ist das System nur benutzbar, wenn auch die Hand ruhig gehalten wird. Um diese nichtlinearen und h\"ochst kritischen Einflussfaktoren zu umgehen, werden die Sensorwerte an die Modellgleichungen angepasst. Diese Skalierungsfaktoren werden \"uber eine vorbestimmte Initialisierungsgeste bestimmt.

Mehrere Datensets wurden mit dem magnetischen System aufgenommen. Die Resultate der Zustandssch\"atzung wurden mit einem bestehenden kamerabasierten System verglichen. Einzelne Bewegungen von mehreren Fingern k\"onnen mit dem entwickelten System nicht nachverfolgt werden. Jedoch ist die Sch\"atzung eines einzelnen Fingers relativ zuverl\"assig und f\"uhrt zu nachvollziehbaren Ergebnissen. F\"ur diesen Fall betr\"agt der kleinste beobachtete Unterschied zum Kamerasystem $ 0.467 \si{\radian} \pm 0.027 $ (=$ 26.757 \si{\degree} \pm 1.547 $).


%\end{otherlanguage}




\cleardoublepage{}

\tableofcontents{}

\cleardoublepage{}

\listoffigures
\addcontentsline{toc}{chapter}{List of Figures} 

\newpage

\section*{Acronyms}
\begin{acronym}
\acro{IMU}{Inertial Measurement Unit}
\acro{ROI}{Region of Interest}
\acro{EKF}{Extended Kalman Filter}
\acro{DOF}{Degree of Freedom}
\acro{SDK}{Software Development Kit}
\acro{PCB}{Printed Circuit Board}
\acro{EMG}{Electromyography}
\acro{HCI}{Human Computer Interface}
\acro{CT}{Computed Tomography}
\acro{CMC}{Carpometacarpal}
\acro{MCP}{Metacarpophalangeal} 
\acro{PIP}{proximal Interphalangeal}
\acro{DIP}{Distal interphalangeal}
\acro{TM}{Trapeziometacarpal}
\acro{CEL}{complete elliptical integral}
\acro{FEM}{Finite element method}
\acro{LSB}{Least Significant Bit}
\acro{BLE}{Bluetooth Low Energy}
\acro{EKF}{Extended Kalman Filter}
\acro{GATT}{Generic Attribute Profile}
\acro{BFGS}{Broyden-Fletcher-Goldfarb-Shanno}
\acro{SLSQP}{Sequential Least Squares Programming}
\end{acronym}

\cleardoublepage{}

%\listoftables
%\addcontentsline{toc}{chapter}{List of Tables} 

\cleardoublepage{}


\mainmatter

\cleardoublepage{}


\pagestyle{fancy}


\lhead[\chaptername~\thechapter]{\rightmark}

%
\lhead[\chaptername~\thechapter]{\rightmark}


\rhead[\leftmark]{}


\lfoot[\thepage]{}


\cfoot{}


\rfoot[]{\thepage}


\chapter{Introduction}

The human hand is one of the most important part of the body for object interaction and communication. The five fingers show a broad range of motion and can perform powerful gestures, such as grasping, as well as sensitive and accurate movements like painting. In order to examine the natural behaviour of the human hand, some special motion tracking systems, capable of the complex finger movements, were developed till now. Those systems can be used for medical treatment of hand injuries or movement impairment, caused for example by an accident or stroke. Another novel field of application would be the interpretation towards \ac{HCI}, to establish a natural way of interacting with devices. Motion tracking of the human body as a whole is not a new feature, although a lot of research is going on in this area. For hand and finger tracking however, those general approaches have to be adjusted, since the granular but also complex finger motions are often performed only within a small movement range and need therefore a system with a higher resolution. Traditionally the developed hand tracking methods use standard motion capture approaches, such as cameras or \acp{IMU} to identify and reconstruct various movements. In order to make them capable for hand pose reconstruction, those solutions are often bulky and introduce external components. Those systems seem not to stand in a relationship to the small and neat hand. One approach, that uses the limited region of interest for hand state estimation, has received only little attention till now. The measuring of magnetic fields, excited by permanent neodymium magnets attached to the fingertips. This approach would decrease the number of sensors and external components for finger state estimation. Since the decrease of the magnetic field, excited by a static magnet decreases with the distance, this method is not so well suited for body tracking. Since the human hand is not too big and the overall measurable magnetic flux density can be adjusted by choosing suitable magnets, this approach seems promising for the objective.

The aim of this thesis is to develop a system for the reconstruction of finger joint angles, by measuring the magnetic flux densities, excited by artificial magnets on the fingertips. The system should present a novel approach, which comes with less bulky and complex components as the so far developed ones. Therefore the system size and usability is desired to be held small and simple. The thesis starts with a general review of related work on hand motion reconstruction and the possible fields of applications. After introducing the general anatomic and magnetic foundations, the hardware and software components of the developed approach will be described. A model will be evaluated to describe the magnetic flux densities, measurable at the deployed sensors, for each finger pose. From that, a reconstruction algorithm mapping the measured superimposed magnetic field to the joint angles of the hand will be implemented. The results of the developed system for the estimated finger poses will be compared to the values of a commercially available camera based approach. Therefore, the overall performance of the magnetic system regarding the physical resolution and accuracy of joint angle reconstruction will be determined. 






%
\lhead[\chaptername~\thechapter]{\rightmark}

\rhead[\leftmark]{}

\lfoot[\thepage]{}

\cfoot{}

\rfoot[]{\thepage}


\chapter{Related Work}
\label{cha:relatedWork}

%\section{Abstracts of papers}
%\begin{itemize}
%\item \cite{huang2008pianotouch}
%\item \cite{huang2010mobile}
%\item \cite{metcalf2013markerless}
%\end{itemize}

\section{Approaches for hand motion reconstruction} \label{sec:approaches}

\cite{sturman1994survey}, \cite{dipietro2008survey}

\subsection{Vision/Camera based} \label{subsec:approaches:vision}
\todo{Review the papers again! Where is the accuracy mentioned? When it is not mentioned, then don't say sth about it...\\}
% General part about vision based motion estimation
Vision based motion capturing systems are widely used. They consist of one ore more cameras, arranged in a certain configuration, to generate a quite exact replication of what wants to be tracked. Those systems are nowadays not only used for tracking and analyzing the motion of humans. The systems and applications range from general purpose devices for entertainment, like interacting with video games or for examining the movement of athletes \cite{zhang2012microsoft}, \cite{boyd2012situ}. So in the end it is no wonder that some groups came up with using a vision based system for hand motion reconstruction, even though those movements bring in some challenging aspects to consider. However the quality of the results of such vision based systems is very high and is often classified as ground truth for motion estimation of fingers. The Optotrak system, which is used by several groups, for example has an accuracy of up to \SI{0.1}{\mm} with an resolution of \SI{0.01}{\mm} \cite{optotrak}. It is very hard to manually reconstruct and measure the real values of finger angles and hand motion, since one can not see the bones without an x-ray. \\
% Steps and 3d model description
No matter what kind of vision based motion tracking system is used, to extract the actual hand pose and movement from an image, one has to perform the following steps:
\begin{enumerate}
\item Image acquisition, fusion (if more than one camera is used) and preprocessing
\item Image processing, to get a focus on the relevant sections (the region of interest - ROI) 
\item Pose estimation, to extract and calculate the actual body, respectively hand pose from the image
\end{enumerate}
A fundamental part for these steps is providing a proper three dimensional object model. No matter whether one tracks the whole body or only a small section, the more detailed the outcome of the system should be, the more detailed has to be the model. The model uses a mesh of triangles and vertices and applies (anatomical) constraints on them. After extracting the relevant sections from the camera image, the model calculates and maps the actual positions and relations between the joints and bones. This step can consume a lot of computation time, if the result should be as detailed as possible. Yun et al. for example solved this problem, by combining a system identification stage, which uses the hand model, with a state estimation stage, where an Extended Kalman Filter is used \cite{yun2013accurate}.

For reconstructing the hand motion with a vision based system, one can find two approaches in literature: The tracking of markers, placed on the hand or a textile glove and the markerless detection of palm and fingers. At first a short overview on the marker based systems is presented. \\
% Marker approach
Supuk et al. \cite{supuk2008evaluation} and Metcalf et al. \cite{metcalf2008validation} use passive reflective markers attached to the hand. While the Optotrak system \cite{optotrak}, used by Supuk et al., only needs one camera instance, the Vicon system \cite{vicon} consists of at least two and can handle up to six cameras for more accuracy. Comparing the effort and capabilities of those systems, measuring only the movement of small hands seems to break the relations. The accuracy of the outcome is directly dependent on the number and positions of used markers. Unfortunately there are no exact numbers given about the achieved accuracy of the vision systems. Metcalf et al. modeled the movement of wrist, hand, fingers and thumb. Therefore they compare the results for different test persons, each equipped with 26 markers in total for one hand. The passive stickers are placed at the three knuckles of each finger, the fingertips and on the back of the hand and lower forearm to guarantee a tracking of the whole hand motion and not only of the fingers (see figure\ref{fig:markers}). It is very important, to place the markers for each person at the same anatomical positions. Attaching the reflectors statically on a textile glove would make the system more flexible and easier to use, but this would also lead to a degradation of the results. Every person's hand varies not only in size, but also in the position and lengths of the individual bones and knuckles. So a general purpose glove is very hard to construct. One additional issue that Metcalf et al. came up with is, that one has to take the size of the surface area into account, too. The hands of children for example won't have the surface to place all markers at the desired positions properly. The placement of the stickers took them between three and five minutes each time. In fact the aim of their study was to show that persons perform specific tasks in their own way, but that one can still observe similarities. Supuk et al. used the camera system more or less only as ground truth, to validate the data from a flexion based Data Glove. They use 19 passive markers, placed in a similar shape than the other group. The group of Yun et al. are using active LED markers. Their paper emphasizes on an effective system identification algorithm and filtering method. They estimate one index finger with seven markers on it, recording it with a system from Phasespace Inc. The accuracy of the measurements was verified by comparing the results to an optimized kinematic model. Here again, no exact numbers about are accuracy are provided.\\
The group of Wang et al. uses a multi-colored glove for finger identification. Their glove is printed with a special color pattern, to simplify the pose estimation problem. This allows them to use one general purpose color camera, which is way cheaper than the motion capturing systems mentioned beforehand. A setup of their system is shown in figure\ref{fig:color}. The pose estimation is done upon a database, containing the glove in different articulations. The image gets processed, to extract the colors clearly, and the pose is found by a nearest neighbors approach, comparing to the database. To penalize the difference between the image and the matched pose from the database, they tune the result by applying inverse kinematics. In the end their systems works reliably. But it is only applicable for one glove size and the results are based on a single test person.\\

For all the above approaches it is very important to place the markers at the correct anatomical positions, to get good and reliable results. So the system presumes that the user knows how to attach the stickers or wear the glove and which mistakes can be made. In order to facilitate this process and make it less fault-prone, a markerless approach would be a better choice.\\
By realizing such a variant however, the region of interest, so to say the actual position of the hand, is not directly given. Further on, the orientation and alignment of the test person has to be interpreted. The following section presents some implementations.\\
For image acquisition Ionescu et al. use a gray scale camera and filter the image for the biggest white region. To get the best results they mention that one has to hold the hand in front of a black surface. In the end, they only try to detect certain hand gestures and not a whole motion. The images are compared to a pre-learned training set in order to recognize the poses. So this group doesn't use any models.\\
Metcalf et al. and Sharp et al. use the Microsoft Kinect. This system defines anatomic landmarks, to identify certain points of the hand. Actually the sensors are designed for declaring landmarks on the whole body, so for tracking a petite hand this identification process has to be adopted. Metcalf et al. use the binary depth image and define the landmarks by searching for reasonable maxima and minima in combination with a 3d hand model. Possible poses of the hand are simulated with the model and adopted to the image. In the end their approach led to an overall accuracy of \SI{78}{\percent}. A marker based system served as a comparison. The approach of Metcalf et al. was only tested with persons sitting on a table, so it is designed for a front-facing close-range scenario.  The group of Sharp et al. goes one step further and brings the system to a universal surrounding. They are able to extract the hand posture and movement from an arbitrary image, no matter at which distance the hand is or what the background looks like. The approach is to introduce a robust reinitializer to handle typical vision based problems like occlusion and image loss. In combination with a fast and effective comparison to a 3D hand model and a learned training data set, the movement and pose of the hand can be estimated very reliably, independent from the person or the environmental circumstances.\\
John et al. use two high resolution color cameras from Sony. To reconstruct the motion of the extracted human hand, the system compares the images to data from the 3d hand model. By matching the model to the input pictures, the position and configuration of the hand is estimated in real time.\\
The commercially available Leap Motion system \cite{leap} includes two IR cameras and three IR emitters and is particularly constructed for hand motion reconstruction. The small controller just has to be put under your hands, like shown in figure\ref{fig:leap}. The Leap Motion Inc. provides a well documented API and software tools for Windows and Mac systems to use their device for basic interaction with a PC. The system directly outputs hand- and fingerpositions. It can also detect whether one is holding a pen or specific tool. Via the API one can directly access the absolute positions of the hand and joints. Also specific gestures, like swiping or drawing a circle with a finger gets directly detected. The accuracy and robustness of the Leap system is analyzed in \cite{weichert2013analysis}. Different motions and positions were examined. To ensure a reliable \grqq test person\grqq~ they choose an industrial robot with a position accuracy of \SI{0.2}{mm}. The overall error of the system for dynamic motions was below \SI{0.7}{\mm}, which is better than the Microsoft Kinect system.\\
The Digits system, developed by Chang et al., is a wrist wearable IR sensor \cite{Digits}. It consists of an IR laser line generator, a ring of modulated IR LEDs, a IR camera, and an IMU (see figure\ref{fig:digits}). The system collects on the one hand a single 3D point for each finger from the line generator, on the other hand a uniformly illuminated image of the hand, produced by the modulated IR LEDs. With those informations and by using inverse kinematics of the underlying human hand model, the group can robustly reconstruct inward hand and finger movement. The IMU can track the alignment and movement of the whole forearm. The overall angular error of the system is~ $ \leq \ang{9}$~ for the joint angles. This value varies with the fingers, since the thumb has a more complex movement and is smaller than the index finger for example. However these values satisfy the clinical standards for joint measurement.\\

Concluding the presented techniques for hand motion reconstruction by a vision based system, one could assert the following basic characteristics of such systems. (The impact or applicability of each point varies with the system, of course):
\begin{itemize}
\item The quality and stability of the tracking is limited to light conditions.
\item Occlusion of unseen fingers, for example by crossing or making a fist, can occur. Also clothes or body parts, held in front of the camera can hide parts of the hand.
\item The proposed systems are usually quite big or even need multiple cameras.
\item This induces that the installation has to be static (like depicted in figure\ref{fig:setup}) and is only capable of localisation based tracking.
\item A three dimensional model of the human hand gets adopted to the images.
\item The algorithms for motion tracking are quite complex and are running on an external PC.
\item For marker based systems: The placement of the markers is crucial.
\end{itemize}

\begin{figure}[hp]
	\centering
	\subfloat[The positions of the markers, used in \cite{metcalf2008validation}]
	{\includegraphics[width=0.4\textwidth]{pictures/marker.png}\label{fig:markers}}
	\hfill
	\subfloat[The Color glove with the used camera setup \cite{Wang:2009:RTH}]
	{\includegraphics[width=0.4\textwidth]{pictures/color.png}\label{fig:color}}
	\hfill
	\subfloat[Example for a static camera - subject setup, used by \cite{metcalf2013markerless}]
	{\includegraphics[width=0.4\textwidth]{pictures/staticSetup.png}\label{fig:setup}}
	\hfill
	\subfloat[The Leap Motion system \cite{leap}. The hands get directly visualized on the screen.]
	{\includegraphics[width=0.4\textwidth]{pictures/leap.jpg}\label{fig:leap}}
	\hfill
	\subfloat[The Digits system \cite{Digits} with the relevant parts.]
	{\includegraphics[width=0.45\textwidth]{pictures/digits.png}\label{fig:digits}}
	
	\caption{Some examples of the described vision based systems}
	\label{fig:examplesVision}	
\end{figure}

\newpage

\subsection{IMU based} \label{subsec:approaches:IMU}
Another concept of motion tracking consists of using IMUs. Those sensors measure the angular rate, acceleration and magnetic field for three dimensions in space, they are also called 9-DOF sensors. With existing suitable algorithms, like a Madgwick filter \cite{madgwick2010efficient} one can calculate the absolute orientation of the sensor, relative to the earth magnetic field, from the sensor data. So one can track instantaneous the orientation of a sensor unit. Therefore it is no surprise that those sensors are commonly used for motion tracking applications in general. The Dutch company Xsens Technologies, for example is specialized on motion tracking using IMUs and develops several suits to track the whole body motion.\\
Kortier et al. use a self designed IMU system, consisting of 18 sensor units in total. The sensors are placed on the bare hand, like depicted in figure\ref{fig:imu}. The units are a gyroscope-accelerometer combination and placed on each proximal, intermediate and distal phalanges of each finger (for explanation of the bones see \ref{sec:anatomy}). For additional information three units are placed on the back of the hand. The PCBs on the fingertips and on the dorsal side are additionally equipped with a magnetometer, to get a more accurate estimation about the orientation of the hand. To filter, estimate and map the raw sensor data to an adequate biomechanical hand model the group uses an Extended Kalman Filter (EKF) framework. They achieved an adequate repeatability. By comparing their approach to a vision based one, a maximum error of \SI{12.4}{mm} was achieved. This value seems pretty high, but they claim that it is because of an misalignment between the optical and their own chosen coordinate frame. This shows once again, that the calibration plays an important role for vision based systems and is not so easy to manage.\\
The group of Fang et al. uses a similar method than Kortier et al. The 16 IMUs, which are all full 9-DOF units, are also placed on the three bones of each finger. For the palmar movement however they use only one sensor. The processing of the data is also done ~\grqq on-hand\grqq~ with the self designed processor-board. For data filtering and position estimation they also use a Kalman Filter and a hand model. The characteristic of the approach lies in the evaluation of the sensor values. Because the hand is composed of rotational joints, they assume that either all sensors are in rotation or none. So they neglect the measurements of the gyros, if the hand is held still and only the fingers are moving, such that they only take the accelerometer and magnetometer data into account. However when the hand is moving, the measurements of the accelerometer and magnetometer have lower dynamics and they use the advantage of the gyros. Further on, they first estimate the pose of the palm, then the attitude of the proximal finger bones, then the angles of the DIP and PIP and finally they calculate the full hand pose, based on the intermediate results. In the end they achieved the intended requirements and point out that the efficiency of their method is almost twice that of the original EKF. Unfortunately they don't provide exact numbers.\\
There also exist some commercially available IMU based glove systems. The company Synertial or Anthrotronix for example provide ready to use gloves. The IGS-Glove from Synertial comes in various editions (two of them shown in figure\ref{fig:synertial}), differing in the number of sensors. It is available with 7, 12 or even 15 IMUs, mounted on the easy to wear glove, delivering you the desired accuracy \cite{Synertial}. Anthrotronix however equip their ''Acceleglove'' with 6 IMUs \cite{anthrotronix}. Both deliver their systems with a SDK to have direct access to the raw sensor data but also to pre-calculated motion and gesture data.\\

Again, all the presented systems show some similarities. The following points try to summarize them:
\begin{itemize}
\item The IMUs are mounted on a textile cloth
\item However a unified cloth, that fits every human hand is difficult to produce
\item The accuracy varies with the number, position and measurement range of the sensors.
\item The more sensors used, the more wires are needed. Also the data traffic and processing time increases with the number of units.
\item A calibration procedure is needed, to increase the accuracy.
\item IMUs are cheap and available in a large variety
\end{itemize}

\begin{figure}[h]
	\subfloat[In \cite{kortier2014assessment} the IMUs are placed on the bare hand.]
	{\includegraphics[width=0.4\textwidth]{pictures/imu.png}\label{fig:imu}}
	\hfill
	\subfloat[The glove system, sold by Synertial]
	{\includegraphics[width=0.4\textwidth]{pictures/synertial.jpg}\label{fig:synertial}}
	
	\caption{Examples of glove systems, using IMUs}
	\label{fig:examplesIMU}
\end{figure}

\subsection{Flexion based} \label{subsec:approaches:flexion}
Another approach of measuring the hand movement is to monitor the flexion of fingers. There are different kinds of flexion sensors out there and many researchers use them for finger tracking. For example in 1977 Thomas de Fanti and Daniel Sandin developed one of the first data glove prototypes at the Massachusetts Institute of Technology (MIT). The Sayre Glove  \cite{sturman1994survey}. They equipped a glove with flexible tubes for each finger. At one end of each tube, they put a LED as light source and at the other end a photocell. The amount of light, arriving at the sensor varies with the flexion and extension of the finger. The more the finger is bent, the less light comes to the sensor.\\
Ten years later, in 1987 Visual Programming Language Research, Inc. rolled out some kind of successor to the Sayre Glove. Their device is equipped with five to ten flexion sensors, based on optical fibre \cite{zimmerman1985optical}. For more accuracy they place a flex sensor on each joint, to measure its angle. They even proposed a system with more sensors, to measure abduction and adduction between adjacent fingers.\\
Another way to measure the flexion are resistive or capacitive bend sensors. These devices can be printed with resistive ink and are therefore highly customizable in shape and size. Resistive bend sensors are used for example by O'Flynn et al., Zecca et al. or by the company 5DT (for representative pictures see\ref{fig:examplesFlexion}). The Didjiglove in contrast is based on capacitive bend sensors \cite{sturman1994survey}.\\
The Italian company Gloreha \cite{Gloreha} follows a more application specific approach. Their rehabilitative glove system consists of mechanical cables for each finger. You can measure how much a finger is bended by the amount of extended wire. On the other hand you also can support the patient by extending or contracting the wire mechanically. This system is big, unhandy and looks more like an exoskeleton, than an unimpressive wearable. Of course it is constructed for rehabilitation and aimed to support specific motions of a patient and not for general purpose measuring of flexion and extension in every day life (more about it in \ref{subsec:applications:reha}). But it still shows a mentionable approach.

In the end, one can say that flexion based hand tracking has the following characteristics:\\
\begin{itemize}
\item The sensors are mounted on the joints. Most groups use therefore a textile glove. 
\item The output of the system is dependent on the positions of the sensors. Ideally this should not change with the user. However each human hand is slightly different and there is not a universal glove size and sensor positioning, which would fit for all.
\item One way to improve this is to calibrate the glove system for each user.
\item The accuracy of the reconstructed finger positions or gestures is limited to the number and the measurement range of the used sensors. With one bend sensor per finger, one could at most only reconstruct the intention of the user's gesture or distinguish between several postures. However by introducing multiple sensors per finger, ideally more than one per joint, one could get an acceptable result. \cite{zecca2007development} used 15 bend sensors on a flexible PCB and reached an average error $ \ang{7.1} $  compared to a camera system.
\item One measures only the bending of a joint or finger. In order to reconstruct a relative or absolute position of the finger additional calculations have to be made.
\item It is a simple and highly customizable system.
\item Easy applications can be realized with only a few sensors (see \ref{sec:applications} for more information)
\end{itemize}

\begin{figure}[h]
	\subfloat[The flexible PCB, used in \cite{o2013novel}]
	{\includegraphics[width=0.4\textwidth]{pictures/flexPCB.png}\label{fig:flexPcb}}
	\hfill
	\subfloat[The glove system, devolped in \cite{zecca2007development}.]
	{\includegraphics[width=0.4\textwidth]{pictures/flexionWB.png}\label{fig:zecca}}
	\hfill
	\subfloat[The Gloreha system for rehabilitation.]
	{\includegraphics[width=0.4\textwidth]{pictures/gloreha.jpg}\label{fig:gloreha}}	
	
	\caption{Examples of glove systems, using flexion based approaches}
	\label{fig:examplesFlexion}
\end{figure}


\subsection{Magnetic/electromagnetic based} \label{subsec:approaches:magnetic}
Another approach is the use of measuring active and passive magnetic fields.\\
Hashi et al. are using an active, resonator based system. Their system consists of a driving coil, a pick up coil array and resonant LC markers. The markers, consisting of an inductive coil and a chip capacitor, are placed on the fingertips and have different resonant frequencies. The exciting coil modulates several signals and sends them out. An electromagnetic field is generated around the coil. Now, holding a marker inside this field the electromagnetic circuit begins to oscillate and generates his own resonant electromagnetic field. This can be measured by the pick up coil array. Each marker has a unique excitation frequency, so by modulating the received signals via FFT, the markers can be identified. Further on, the measured amplitude of the signal represents the intensity of the magnetic field. The group assumes that the excited field of the marker behaves like a magnetic dipole field. So by using the suitable equation \todo{add reference to magnetics section and formula!} one can calculate the position and orientation of each marker uniquely. They tested their approach and came up with a position accuracy up to 2 mm, for locations up to \SI{100}{mm} away from the pick up coil array. Increasing the distance further, the results get worse. The group of Schaffelhofer et al. tested the commercially available system of Northern Digital \cite{wave} with primates. They achieved an overall accuracy of $ \ang{2.41} \pm \ang{3.36} $ for the tracking of dynamic movements. However the system, consisting of very complex and bulky components is not so well suited for mobile or general purpose use.\\
The group of Ma et al. take the approach of determining the position of a passive cylindrical bar magnet by approximating it with the magnetic dipole field. They tried to reconstruct the movement of the fingers, by placing neodymium magnets on the fingertips and measuring the magnetic field. A draft of the system can be seen in\ref{fig:passiveMag}. Like the active approaches, they use the model for the magnetic dipole, to estimate the position and orientation of the passive magnets. To conclude from those estimated values to the actual finger position, they use inverse kinematics with an underlying human hand model. For verification of this approach they equipped a test person with one magnet on the index fingertip and six sensors on a wristband. The proband was asked to perform several flexion and extension tasks of his finger. The results for the estimated finger positions and orientations were consistent with the data recorded by a Vicon system. Exact accuracy values are not provided by the group.\\
In the end one can say, that this approach sounds promising. 

\begin{figure}[h]
	\subfloat[A draft of the passive magnetic system, proposed in \cite{ma2011magnetic}]
	{\includegraphics[width=0.4\textwidth]{pictures/magnetic.png}\label{fig:passiveMag}}
	%\hfill
	%\subfloat[The glove system, devolped in \cite{zecca2007development}.]
	%{\includegraphics[width=0.4\textwidth]{pictures/flexionWB.png}\label{fig:zecca}}
	
	\caption{Examples of hand tracking systems, using magnetic/electromagnetic approaches}
	\label{fig:examplesMagnetic}
\end{figure}


\subsection{Other approaches} \label{subsec:approaches:other}
As one can see, there is a lot of research going on in the area of motion estimation for the human hand. The focus of the so far presented approaches was relied to general purpose devices, designed for trying to track reliable and accurate the whole range of motions. Due to the wide range of possible applications (see~\ref{sec:applications}) there are also some more specialised variants in reconstructing gestures or movements. This section introduces some of them.\\
The Pinch Gloves, visualized in\ref{fig:pinch}, designed by Fakespacelabs represents an input device \cite{bowman2001using}. The system consists of a glove and conductive elements, sewn into the tips of each of the fingers. When two or more fingers are pinched together, the conductive parts come into contact and generate an individual signal. This signal can easily be interpreted by a computer and therefore serve as an input. There is also the possibility to attach a position tracker and add the motion of the hand as an input possibility. The system is used for virtual environment interaction. In \cite{bowman2001using} a more elaborated interaction technique with this simple system is presented. They developed an environment to navigate through menus or to type on a virtual keyboard.\\
The eRing, developed by Wilhelm et al. is another kind of gesture interaction device. It consists, as the name suggests, of a ring, enclosed by capacitive foils (a first prototype is presented in\ref{fig:ering}). The capacity of the system is related to the conductive environment. The human body has an influence on the magnitude of capacity, which changes by moving the fingers around the ring. This change in capacity can be measured, by determining the rise time $ \tau $ of the RC circuit. In their paper the group describes that the system is able to detect static and dynamic gestures, as long as the neighbouring fingers are not to far away from the ring. For recognizing the gestures they use a 1-nearest neighbour approach on a pre learned dataset.\\
The Rutgers master \rom{2} represents an exoskeleton like approach \cite{bouzit2002rutgers}. It consists of four pistons with rings on the ends, to clip them on the thumb, index, middle and ring finger (see\ref{fig:rutgers}). The movement of the pinky finger is neglected. The adduction and abduction of the fingers is measured by Hall-effect sensors, the flexion and extension via infrared sensors. From the piston movement one can calculate the finger angles via a kinematic hand model. The pistons are inside an air cylinder to reduce friction in the system. This glove can provide force-feedback to the fingers, since the pistons can be controlled externally. Its main application area is therefore the rehabilitation and learning of hand movements. However the exoskeleton structure restricts the range of movement. Only \SI{55}{\percent} of the natural grasping motion can be performed with this system. The accuracy of the system was evaluated to $ \ang{0.75} $ for the adduction/abduction and \SI{0.5}{mm} for the piston position. A similar system is the CyberGrasp haptic glove \cite{cyberglove}. It is slightly more accurate (resolution of \ang{0.5}) but also heavier and even more cumbersome.\\
An interesting approach is mentioned by Mascaro et al. They introduce a sensor placed onto the fingernail, measuring emitted light. LEDs are placed on top of the nail, emitting light with different wave lengths and measure the response from the nailbed. This technique is called reflectance photoplethysmography. If there is a force applied to the fingertip, one can see that the color of the skin beneath the nail changes. This behaviour can also be obtained when moving the finger. Those, often very small and not clearly visible, changes in color can be detected by their system. However there are many factors to take into account like skin color, blood flow in the fingers, texture of skin and so on, which complicate the reconstruction of motion. Hence they are only able to measure forces with this system \cite{mascaro2001photoplethysmograph}. The draft\ref{fig:nail} illustrates their approach.\\
Another approach for recognizing gestures is by measuring the electric potential of forearm muscles \cite{kim2008emg} By performing gestures with the hand, the electromyographic potential especially in the forearm changes. This can be recorded by a so called electromyograph (EMG). This approach is based on a learning data set, recorded for one person and by recognizing those gestures in real time. Zhang et al. designed a framework for this, taking also the data of an accelerometer into account.

\begin{figure}[h]
	\subfloat[Pinchglove system, used in \cite{bowman2001using}]
	{\includegraphics[width=0.4\textwidth]{pictures/pinchgloves.jpg}\label{fig:pinch}}
	\hfill
	\subfloat[First prototype of the eRing \cite{wilhelm2015ering}]
	{\includegraphics[width=0.4\textwidth]{pictures/ering.png}\label{fig:ering}}
	\hfill
	\subfloat[The Rutgers Master \rom{2} \cite{bouzit2002rutgers}]
	{\includegraphics[width=0.4\textwidth]{pictures/rutgers2.png}\label{fig:rutgers}}
	\hfill
	\subfloat[The photoplethysmography sensor approach of \cite{mascaro2001photoplethysmograph}]
	{\includegraphics[width=0.4\textwidth]{pictures/nail.png}\label{fig:nail}}
	
	\caption{Collection of various approaches for hand motion reconstruction}
	\label{fig:examplesOther}
\end{figure}

\section{Possible fields for applications} \label{sec:applications}

\subsection{Human-Computer-Interface (HCI)} \label{subsec:applications:HCI}
The way we interact with electronic devices becomes more and more natural and is still changing. In the recent years touch input became ubiquitous, for example. Gaming consoles already bring the possibility to physically interact with games. The Kinect camera for Xbox, the Nintendo Wii controller or other haptic systems. Hand motion based systems could be one way to bring the human computer interaction to the next level. Commercially available devices like the Leap Motion or the Myo wristband allow a broad gesture based interaction with the computer or smartphone. The Leap controller for example enables you to play games or interact with 3D graphics in a natural way \cite{leap}. The benefit of such hand tracking devices is, that you don't need a big camera set up and don't have to leave your desk position. You just place the device in front of you or wear it, perform the desired gesture and the system behaves as you want. For example by swiping, a window gets closed or the field of view gets enlarged by pinching two fingers. For the interaction with mobile devices, such as smartphones, the system has to be wearable, like in \cite{Digits}. Also the evolving field of virtual reality environments serves as a base for hand motion interaction. For the Leap motion again, there exists a mount to combine it with the Oculus Rift system. Also creative tasks can be performed by a hand motion tracking system, for example drawing and designing a 3D object with your fingers on a virtual canvas.\\
Another, a bit more serious, way for HCI is the control of robotic machines, especially arm like devices. Such an application could be safety critical, so the interpretation of the hand motion has to be accurate and reliable, like the system proposed by Sharp et al. Dependent on the application, those devices should be able to track the whole finger movement and not only react on predefined gestures. Examples could be controlling robots in space \cite{dipietro2008survey} or in dangerous environments like military territory or for bomb disposal \cite{greenleaf1996developing}. Also the execution of surgical tasks could be one field of application. Nowadays invasive operations should be performed in no time and leave very small scars. For that a lot of endoscopic surgery is done with remote controlled catheters. However most of those invasive devices have very low functionality, often limited to cutting or exhausting something. The devices are controlled by the surgeon often by a simple mechanical system. Giving the practitioners the ability to use a more sensitive and complex method of interaction, would not only reduce the time needed for the intervention and the risks for the patient, but also enlarge the possibilities of interventions.\\
\todo{A concluding sentence??? Saying sth. like \grqq... one can see that the accuracy and mobility is dependent on the application...\grqq. But this is obvious and in some sense stated above...}

\subsection{Therapeutic/Rehabilitation} \label{subsec:applications:reha}
The exact measurement of finger joints, called goniometry, is still a time consuming and error-prone task. Therapists and doctors have to measure patient's range of movement (ROM) in case of hand immobility diseases like arthritis, rheumatism \cite{o2013novel} or parkinson's disease \cite{su20033}, or after a hand operation and fractions. Till now they use mechanical goniometers \cite{williams2000goniometric} to measure static angle positions, dynamic measurements are not possible to evaluate. A more accurate and reliable method would ease their life tremendously. This is where the hand motion reconstruction comes in. As described in~\ref{sec:approaches} there are already systems which can measure the ROM of fingers very accurate. Williams et al. verified their flexion based glove as clinically admitted and can even measure adduction/abduction more exactly than with goniometers. The group of O'Flynn et al. developed a flexion and IMU based glove with a very detailed interface for doctors. They can record and visualize raw sensor data, like the bending angle or movement velocity, as well as a 3D hand, miming the actions of the user. The interface can also provide information about former measurements and can therefore monitor the development of the patients rehabilitation process. The diagnosis and measurements for Parkinson's disease are also critical. Till now the diagnosis relies on objective observations by the doctor and patients. This induces that the illness is often detected in an advanced stage. Su et al. developed a system especially to detect this neurological disease. With the help of an electromagnetic based glove, several experiments which could indicate Parkinson were carried out. In the end they compared the results for ill and healthy patients and could clarify significant differences in the execution of the tasks.\\
Another possibility to support especially the work of therapists is the home-based rehabilitation or telemedicine \cite{metcalf2013markerless}. This means that the patients have a suitable guidance system at home and can perform the exercises with it. The paper of Durfee et al. validated that such systems can return the same results as clinical tests and could therefore save the patients time and effort. The system could for example detect over- or under-exertion while performing a presented task. Further on, by providing an attractive interface like a virtual-reality environment, the patients are much more motivated to execute the training on their own \cite{popescu2000virtual}. Such an interface doesn't has to be complex. The patient could see for example virtual objects on a screen and try to interact with them. The environment, consisting of a PC and the glove system, records not only the performed motions but also takes a video of the exercise session and can automatically be transmitted to the therapist. This ensures a constant verification and interaction with the clinic personal. That such a system can improve the work of therapists is validated by Heuser et al. In their study five postsurgery subjects suffering from Carpal tunnel syndrome were trained to perform tasks. The effects in hand function improvement was tremendous for all subjects. The strength for grip and pinch movement increased up to \SI{150}{\percent}. The system was very good accepted by the patients and the therapists. Such therapeutic system can benefit from approaches with force feedback. The Rutgers Master \rom{2}, CyberGrasp or the Gloreha approaches, all mentioned beforehand, are such systems. Force feedback systems are often more bulky than the motion tracking ones, in \ref{fig:gloreha} the exoskeleton like Gloreha system is pictured. Popescu et al. \cite{popescu2000virtual} for example developed a system for the hand, using the Rutgers Master \rom{2}.\\ 
However for such a telemedicine system the accuracy has to be exact and reliable, also the data has to be processed in real time. One of the most important requirement is however the usability. The system has to be easy to set up and able to detect false focuses or attitudes. A glove based system for example brings in difficulties in donning and doffing \cite{metcalf2013markerless}, other systems are cumbersome or have fragile parts like wires \cite{bouzit2002rutgers}, which could complicate the usage. For camera based systems the focus, illumination and the background of the image are critical points \cite{ionescu2005dynamic}.\\
Till now only the mentioned studies and systems, described in \cite{heuser2007telerehabilitation} and \cite{popescu2000virtual} where performed, but certainly there will be more investigation in the near future.\\


\subsection{Activity and Gesture tracking} \label{subsec:applications:activity}

As already depicted in \ref{subsec:applications:HCI}, the reconstruction of human hand motion can be used for gesture recognition. Activity recognition is mainly based on using a training data set for the gestures and an algorithm, to compare the actual movement with this database to finally judge whether the gesture has been performed or not. But this feature can not only be used as an HCI. Having a wearable system, one can detect specific gestures ubiquitously, which can be used to produce a diary like tracking of hand gestures. Such a monitoring system could be a support again for therapists, trying to analyse the daily routines of their patients. It is possible to detect a grasp intention \cite{supuk2008evaluation}, \cite{zhang2011framework} and also the strength of a grasp \cite{ekvall2005grasp}. By downsizing the systems and making them wearable one could get a detailed logbook about hand activities. The eRing \cite{wilhelm2015ering} could be such a system. With an unimposing data glove system, like the one of 5DT or Synertial, one could even track the whole hand motion. Because of the mentioned reasons to use such a system in a ubiquitous environment, vision based systems are not so well suited for this application field. Apart from the Digits \cite{Digits} system, most of the other developments use a bigger, stationary camera system.\\
Another possible application field concerning the recognition of gestures is the understanding and translation of sign language. Deaf persons are using the sign language to communicate with the outside world. However most of the~\grqq talking\grqq~persons do not know this complicated alphabet, which is based on specific hand gestures and poses, such that it can be very tiring for a deaf person to interact with others. Several groups have done research in this field like presented in \cite{mehdi2002sign}, \cite{fels1993glove} and \cite{dipietro2008survey}. Again, a wearable and mobile system is better suited for this task than a static, camera based approach. As the sign language has a lot, but sometimes similar looking gestures the training data set and the corresponding classification algorithm have to be fast and reliable. In the end such a hand tracking system has to be reliable, accurate, and able to track the movement of the whole hand. As a classification algorithm Fels et al. propose use neural networks. The detected letters or words can then be visualized on a display or directly made audible by speakers \cite{fels1993glove}. To ensure a natural behaviour, the gestures have to be interpreted and visualized in real time. Fels et al. used a fibre optic data glove with 11 sensors, including an IMU to track the orientation of the hand. In the end they achieved an accuracy of about \SI{99}{\percent} for words and only about \SI{5}{\percent} of the attempts resulted in no detection. Mehdi et al. use only one flexion sensor per finger and two IMUs to track the orientation of the hand. They only achieved an accuracy of \SI{88}{\percent} and some gestures could not be detected, because the movement of the forearm was not tracked. To improve their systems, the two groups state to investigate more on the algorithm and to use gloves with a higher reliability and accuracy.


%
\lhead[\chaptername~\thechapter]{\rightmark}

\rhead[\leftmark]{}

\lfoot[\thepage]{}

\cfoot{}

\rfoot[]{\thepage}


\chapter{Foundations}
\label{cha:foundations}

\section{Anatomy of the human hand}
\label{sec:anatomy}
\begin{itemize}
\item \cite{john2006advanced}
\end{itemize}

\section{Magnetic}
\label{sec:magneticFound}

%\lhead[\chaptername~\thechapter]{\rightmark}

\rhead[\leftmark]{}

\lfoot[\thepage]{}

\cfoot{}

\rfoot[]{\thepage}

\chapter{System Design and Implementation} \label{cha:sysDesign}

\section{System design} \label{cha:design}

\begin{itemize}
\item general description of the system
\item four sensors, magnets on fingertips, sensors on rack at back of hand, ...
\item wooden hand for easy/repeatable measurements (it is not perfect! since the index and pinky finger do not move in x-z plane!)
\item pictures!
\end{itemize}


\section{Sensor design/Data acquisition...} \label{cha:sensors}

\begin{itemize}
\item describe arduino system (BLE, ...)
\item which kind of sensors and why (range is adjustable, how comes the conversion factor in, ...)
\item hard-soft iron effects
\item calibration methods for sensors (hard-soft things / Freescale approach)
\item also calibration/scaling towards the magnetic model equations! \\
		$ \rightarrow $ scaling/offset termination on flat paper with (almost) exactly known positions
\item the circumstance of the earth magnetic field and how to overcome it \\
		$ \rightarrow $ rotation estimation with Madgwick filter! Mention that you take an out of the box code for it
\end{itemize}


\section{Deriving the Human hand model} \label{sec:handModel}

\begin{itemize}
\item neglecting adduction-abduction
\item angle of PIP is 2/3 of DIP
\item bring here the picture of the sketched hand from the equation document!
\item describe that you can break every position/orientation of a fingertip to a combination of MCP and DIP
\item assumption, that every pose leads a unique B-field
\item only right hand
\end{itemize}


\section{Magnetic field interpretation } \label{sec:magmodel}

\begin{itemize}
\item describe the derivation of the position vector, relative to sensor position, joint angles, ... \\
		$ \rightarrow $ so the whole $ P_{sensor} - P_{fingertip} - P_{joint} $ thing
\item the derivation of the orientation vector (dip model) and orientation angle (cyl model)
\item how to formulate/evaluate/modify the B-field equations, to get the outcome\\
		$ \rightarrow $ for the dipole model (derivation of the orientation is \grqq a bit tricky \grqq)\\
		$ \rightarrow $ describe the whole turning and transforming process for the cylindrical model
		
\end{itemize}		


\section{Hand state estimation} \label{sec:estimation}

\begin{itemize}
\item general description of minimizing the error
\item how do I set up my matrices and therefore, how does my optimization problem look like
\item optional constrains
\item solving it with the EKF approach
\end{itemize}


\section{Visualization} \label{sec:visual}

\begin{itemize}
\item the Blender thing...
\item using a rigged hand model
\item setting it up as a Blender Game
\item I only pass the estimated joint angles
\item briefly describe bpy interface and how I manipulate the rigged hand
\end{itemize}






%\lhead[\chaptername~\thechapter]{\rightmark}

\rhead[\leftmark]{}

\lfoot[\thepage]{}

\cfoot{}

\rfoot[]{\thepage}

\chapter{Results} \label{cha:results}

\section{Data behaviour / Sensor values} \label{sec:dataRes}

\begin{itemize}
\item general behaviour of sensor (when you get to near, there is a clipping/oversteering)
\item timing behaviour with the earth-mag elimination

\item comparing the calibration methods \\
		$ \rightarrow $ hard-soft as good as Freescale, took Freescale
		
\item observations from the earth-mag elimination \\
		$ \rightarrow $ not to 100 \%  possible, but still better than nothing...
		
\item problems with determining the model fit parameters \\
		$ \rightarrow $ difference in orientation of magnet \\
		$ \rightarrow $ tried to fit it to wooden hand, but there the error is already in the simulated(fitting) data \\
		$ \rightarrow $ hard to derive!
		
\end{itemize}


\section{Pose estimation} \label{sec:estimationRes}

\begin{itemize}
\item results of minimization for simulated data with and without noise
\item effect of adding constraints
\item EKF results

\item results with real data
\item as described, sensor values are not good scalable to model...
\item though not so good results...
\item due to timing behaviour of RT minimization, not such a good way to do have a real time estimation
\item so I have results for single magnet - one/multiple sensor 
\item and for multiple magnet - multiple sensor combinations
\item presentation and comparison between EKF and minimizing approach
\item claiming that the difficulty in scaling the measurements is critical...

\end{itemize}





% REMIND: only for compilation speed up!
% when you are finished, comment this line out. The file is included by results.tex!
\section{Pose Estimation} \label{sec:estimationRes}

\subsection{Identification of the Minimization Process} \label{subsec:resSim}

\subsubsection{Utilized Minimization Methods} \label{subsubsec:miniMethod}

The size and complexity of the minimization problem, as described in \ref{sec:estimation}, is dependent on how many finger states $ K $ should be estimated with which number of sensors $ N $. The beforehand introduced minimization problem \ref{eq:minimization} is stated here once again for clarity:
\begin{equation*} \label{eq:minimization}
\begin{aligned}
\underset{\mathrm{X}_K}{\text{minimize}} & & f(\mathrm{X}_K) \\
\text{subject to} & & 0 & \leq {x}_1(\theta_{\mathrm{MCP}}) & \leq & 1/2 \cdot \pi, \\
				  & & 0 & \leq {x}_1(\theta_{\mathrm{PIP}}) & \leq & 110/180 \cdot \pi, \\
				  & & -30/180 \cdot \pi & \leq {x}_1(\phi_{\mathrm{MCP}}) & \leq & 30/180 \cdot \pi, \\
				  & & 0 & \leq {x}_2(\theta_{\mathrm{MCP}}) & \leq & 1/2 \cdot \pi, \\
				  & & & \vdots \\
				  & & -30/180 \cdot \pi & \leq {x}_K(\phi_{\mathrm{MCP}}) & \leq & 30/180 \cdot \pi
\end{aligned}
\end{equation*}
Remind, that the overall size of the observable measurements $ \tilde{\mathrm{M}} $ is $ (3 \cdot N) \times 1 $ (with $ N $ being the number of sensors, taken into account) and the size of the system state $ \mathrm{X} $ is $ (3 \cdot K) \times 1 $ (with $ K $ being the number of finger poses to describe). In order to gather a fully determined system, the number of used sensors has to be at least as high as the number of magnets. This means, trying to estimate the state of four fingers with only one sensor would lead to ambiguous results. Furthermore, the objective function $ f(\mathrm{X}_K) $ can be described by the dipole or the cylindrical magnetic model. The problem can be solved by applying the anatomic constraints as bounds or not. It is implemented with methods, provided by the \emph{SciPy} package. It provides the \emph{minimize} function, which is especially for solving scalar minimization problems. It can be invoked with different algorithms and their corresponding additional options.

The following explanations should give a short overview on the principle of the utilized minimization methods and why they were chosen. For further reading on numerical optimization methods, please have a look at the work of Nocedal et al. \cite{nocedal2006numerical} (on which the following paragraphs are based).\\
For solving the problem without taking the anatomic bounds into account, the \ac{BFGS} algorithm is used. It is an approximation of Newton's method, for finding a solution. Newton's method describes derivative based approaches, to find local minima around a certain initial guess $ \mathrm{X}_{0} $. To find values for the variable $ X_{K} $, which minimizes the outcome of the objective function $ f(\mathrm{X}_K) $, different search methods exist. The \ac{BFGS} algorithm uses a line search approach to find the local minimum along a line, which is determined by the Jacobian $ \nabla f $ and Hessian $ \nabla^{2} f $. The \ac{BFGS} approximates the Hessian  $ \nabla^{2} f $ and is therefore called a quasi-Newton method. The derivative $ \nabla f $ is updated at every iteration. To calculate it, the objective function is evaluated with a state having slightly different values, than the provided initial guess $ \mathrm{X}_{0} $. An iteration step consists of finding a value $ x_{k+1} $, which minimizes $ f $. This is done till the gradient norm $ || \nabla f|| < \epsilon$, with $ \epsilon $ representing the convergence tolerance. In other words, a solution is found, if the change in the value of $ ||\nabla f|| $ is smaller than $ \epsilon $. As a characteristic of the \ac{BFGS} method, only the first derivative needs to be approximated. The rate of convergence for the method is stated to be linear. The overall termination tolerance, defining the magnitude of $ f(\mathrm{X}_K) $ is denoted to be $ 1.0\text{e}-07 $. Shrinking this value, would lead to more exact results, but would also induce more iteration steps and therefore a higher computation time.\\
For solving the problem by taking the anatomic conditions into account, \emph{SciPy} provides a method called \ac{SLSQP}. The constraints can be passed in as a pair of $ (min,max) $ for each variable, and reflect hard bounds. The underlying principle is based on least-squares methods. Therefore the system has to be overdetermined or at least fully determined. It tries to fit the observed data (i.e. the measurements) to a given model, by adjusting the model parameters. This is actually often used for data-fitting. While a system state is desired and the model comes with no additional parameters, the method is used in a slightly different way. In contrast to the classic approach, the model is fitted to the measurement data. The parameters in this case are the values of the system state $ \mathrm{X}_K $. In the least squares sense, the sum of the errors between the model at state $ \mathrm{X} $ and the measurements is squared and minimized. Exactly this is expressed by the objective function $ f(\mathrm{X}_K) $. Again, a starting point $ \mathrm{X}_{0} $ has to be provided. For identifying the direction of $ x $ in each iteration step, Powells method \cite{powell1964efficient} is used. This derivative free approach identifies independent convergence vectors for each variable. It can be interpreted as the approximation of $ \nabla f $. At each iteration step, those search directions are redefined and therefore the new system state can be expressed by a combination of them in turn. In order to bring in the constraints, $ f $ is modified to represent those restrictions as a non-negative least squares problem. As the name suggests, the restriction to the system state is the following $ \mathrm{X} \geq 0$. Those reformulations are done by the \emph{SciPy} method, therefore no further adjustments to the model or the bounds have to be made by the user. In the end the recursion gets performed, till the termination tolerance for $ f(\mathrm{X}_K) $ is fulfilled. This value is again chosen to be $ 1.0\text{e}-07 $.\\
It should be mentioned, that for the implemented estimation routine, the initial guess $ \mathrm{X}_{0} $ is always chosen to be the state, estimated one step ahead. 

\FloatBarrier
\subsubsection{Classifying the Methods with Simulated Data} \label{subsubsec:simEval}

In order to get an impression on the expectable results of the minimization method, it is tested with a simulated dataset. A self chosen predefined set of states is determined, which should represent the motion of the fingers. This sequence of joint angles is simulated using the cylindrical model, to obtain the value of the expectable magnetic flux density, measurable by a specific sensor for the corresponding system state. The cylindrical model is used, since it represents the de facto magnetic flux densities, excited by the bar magnet. Those values for the expectable magnetic flux density are then passed to the minimization routine, to estimate the system states. The result of the minimization should of course reflect the predefined motion sequence. Therefore it can directly be compared to the known state values, to identify the quality of the solver and its behaviour.\\
As stated previously, there are several parameters for formalizing the estimation problem and to tune the solver:
\begin{itemize}
\item Expressing the minimization as an unconstrained (by using the \ac{BFGS} algorithm) or constrained (by using the \ac{SLSQP} algorithm) problem
\item Considering the influence of the movement of adduction-abduction or not.
\item Formalizing the objective function using the cylindrical or the dipole model.
\item The behaviour regarding different determinedness of the system, which means estimating the states of one or multiple fingers by taking one or multiple sensors into account.
\end{itemize}
The results will be compared by calculating the mean and standard deviation of the error-norm to the perfect system state for each finger. Moreover, the calculation times of the different methods allow a conclusion to the number of needed iterations to find a solution.

As a first step, the different optimization parameters are evaluated for the movement of a single finger. Therefore the size of the system state  is $ \mathrm{X}_{1} = 3 $ for taking $ \phi_{\mathrm{MCP}} $ into account and $ \mathrm{X}^\prime_{1} = 2 $ for neglecting this state variable. The size of the simulated measurement vector is dependent on the number of sensors, taken into account. The index finger is chosen for evaluating the different parameters, but the results are expected not to change, by choosing a different one. The used gesture sequence is displayed for the three states of the index finger in \ref{fig:indexStates}.\\
The angular change, and therefore the stepwidth between two states is determined by combining the observations for the angular velocity from Ingram et al. \cite{ingram2008statistics} and the data rate of the sensor system. An acquisition frequency of \SI{20}{\Hz} in combination with a mean angular velocity of \SI[per-mode=symbol]{10}{\degree \per \second} leads to an observable maximum change of \SI[per-mode=symbol]{0.5}{\degree \per \second} (= \SI[per-mode=symbol]{0.0085}{\radian \per \second}). Therefore, the whole set for the utilized motion is divided into 1419 datapoints. For obtaining this number of simulated measurements by the sensor units, would result in a total theoretical duration of \SI{70.95}{\second}. The state values are plotted against time. The motion is constructed to represent simple and complex movements of the finger, including flexion-extension, as well as adduction-abduction. The motion sequence includes joint movements, which happen as unique motions at a time. For example between \SI{0}{\second} and \SI{20}{\second} only the \ac{MCP} joint moves. Some, which arise together, like between \SI{40}{\second} and \SI{60}{\second}, where all three joints are performing flexion-extension. Also only small movements are simulated. Between \SI{38}{\second} and \SI{42}{\second}, $ \theta_{\mathrm{PIP}} \; \text{and} \; \theta_{\mathrm{DIP}} $ change only about \SI{0.26}{\radian}. The movement of adduction-abduction is applied during a short sequence, since the range of movement is small and also occurs more rarely, compared to flexion-extension.\\
\begin{figure}[!htb]
\centering
\includegraphics{pictures/plots/indexStates.png}
\caption[Introduced movement pattern for index finger estimation]
{The introduced motion pattern for the estimation of the state vector for the index finger. For a better readability, the four states are divided into individual figures. The movements are chosen to test whether the estimation is capable of changes, happening to a single state or a combination of them.}
\label{fig:indexStates}
% python script: 160224_plotSequence.py
\end{figure}
%\todo{Also explain WHY the results are like they are!
%	\begin{itemize}
%	\item Overdetermindness: Because the minimizer has more equations, to solve the problem and is therefore directed in the right direction already by the measurements/system values
%	\item Cylindrical: Because you predict the field with it (I think I got that...)
%	\item Constrained/Unconstrained: I think I got that, otherwise, obvious...
%	\item ad-ab: mainly because you simulate the movement of ad-ab at a small time span and have the B-field of it in your states. Without that, the results would be as good as with ad-ab. This proves, that $ \phi_{MCP} $ can be estimated properly! And since it is more natural, ... (think I got it at some points...)
%	\end{itemize}}
The obtained error means and standard deviations for each parameter combination are presented in \ref{tab:oneFing} in radians. The numbers in the very first column indicate the combination of fingers and sensors. The first number represents the estimated finger state vectors $ K $, which is for this comparison always one, since only the states of the magnet at the index finger are estimated. The second number represents the amount of simulated measurements $ N $. By using only one simulated sensor reading, the unit beneath the index finger is meant. By taking two into account, the index and middle sensors are pointed. And four means that all four simulated units are regarded. The abbreviations in the second column reflect whether the movement of $ \phi_{\mathrm{MCP}} $ is regarded or not. ``no ad-ab'' stands for no adduction-abduction movement and ``ad-ab'' for the opposite. 
\begin{table}[!htb]
\centering
\begin{tabular}{l l c c c c}
\toprule
 & &          				\multicolumn{2}{c}{Unconstrained}          &		\multicolumn{2}{c}{Constrained}\\ \cmidrule(lr){3-4}\cmidrule(lr){5-6}
 & & 								Dipole   			   & Cylindrical 	 			 & 		Dipole 			& 		Cylindrical \\ \midrule[2pt]
\multirow{2}{*}{11} & no ad-ab    & $ 0.194 \pm 0.002 $ & $ 0.074 \pm 0.001 $ & $ 0.367 \pm 0.015 $ & $ 0.035 \pm 0.000 $ \\ 
					& ad-ab		 & $ 0.252 \pm 0.003 $ & $ 0.257 \pm 0.013 $ & $ 0.570 \pm 0.020 $ & $ 0.570 \pm 0.020 $ \\ \midrule
\multirow{2}{*}{12} & no ad-ab    & $ 0.124 \pm 0.001 $ & $ 0.094 \pm 0.001 $ & $ 0.052 \pm 0.000 $ & $ 0.035 \pm 0.000 $ \\ 
					& ad-ab		 & $ 0.071 \pm 0.000 $ & $ 0.000 \pm 0.000 $ & $ 0.058 \pm 0.000 $ & $ 0.000 \pm 0.000 $\\ \midrule
\multirow{2}{*}{14} & no ad-ab    & $ 0.112 \pm 0.001 $ & $ 0.098 \pm 0.001 $ & $ 0.040 \pm 0.000 $ & $ 0.033 \pm 0.000 $ \\ 
					& ad-ab		 & $ 0.042 \pm 0.000 $ & $ 0.000 \pm 0.000 $ & $ 0.038 \pm 0.000 $ & $ 0.000 \pm 0.000 $\\										
\bottomrule
\end{tabular}
\caption[Quality of the minimization method for estimating one finger]
{The error mean and standard deviation for each tuning parameter of the minimization procedure are listed. The values are given in radians. ``no ad-ab'' means, that the used objective function did not comprise the state $ \phi_{\mathrm{MCP}} $, vice versa for ``ad-ab''. The system configuration is coded with the numbers. The first one stands for the size of the estimated finger state vectors $ K $, the second for the number of simulated sensor units $ N $ taken into account. The best promising results are represented by the constrained methods, which take $ \phi_{\mathrm{MCP}} $ into account. It is also observable, that the system has to be overdetermined in order to lead to a good estimation of the system states. The minimization is performed on an introduced movement pattern for the index finger, whose values for the excited magnetic flux densities are simulated using the cylindrical bar magnet model. So only one finger state vector is estimated.}
\label{tab:oneFing}
\end{table}


One thing, that can be observed directly, is that for the case ``11'' which still represents a full determined system, the results show a very high deviation from the perfect values, regardless how the model is adjusted. The mean over all errors is \SI{0.289}{\radian}. The best observable values for taking only one sensor unit into account can be obtained by the method using the constrained cylindrical model and neglecting adduction-abduction. By regarding, that the inserted magnetic values were predicted by this model and that the overall system state is simplified, this seems reasonable. Furthermore, the constraints restrict the algorithm not to drift too far away. \ref{fig:11cylNa1} shows the results for this best guess and the deviation from the perfect values over time. \\
By deploying only one set of forecasted sensor values more ($ N = 2 $), the results get much better. The mean over all error means is \SI{0.054}{\radian}. Also the standard deviation is almost constant. One could even state, that by using all four simulated sensor units, the error does not decrease very much (the mean over all errors is \SI{0.045}{\radian}). Therefore it can be stated as a first observation, that the system has to be overdetermined. The objective function is described by more equations as there are variables to find. The solver is therefore directed into the right direction already by the additional system values.\\
By comparing the error from the objective function using the dipole model with the one formulated with the cylindrical, a decrease can be observed. As already stated for the ``11'' case, this just seems reasonable, since the magnetic flux densities were calculated by the same. However, for real observed measurements, this has to be further evaluated.\\
Also, while considering that the field values for the estimation still comprise the movement of adduction-abduction and since the ability to estimate the system state with a reasonable accuracy, the neglecting of those values just results in worse results. The biggest difference to the perfect values occur here at the time, the lateral movement is performed. The remaining parts, where $ \phi_{\mathrm{MCP}} $ is 0 are also almost perfect. By looking at the difference between the results of the constrained and unconstrained methods a slight decrease of the error if the constraints are obeyed can be observed. The algorithm shows better convergence by the deployed constraints. In the end almost fault free results can be observed by the cylindrical model, which takes the movement of adduction-abduction into account. Here it does not count too much, whether the minimizer is constrained or not. %\todo{concluding statement? what are bad results? which dimension of errors is acceptable?}\\
\begin{figure}[!htb]
\centering
\includegraphics{pictures/plots/difOne2.png}
\caption[Estimated states vs. perfect states for using one magnet, one sensor.]
{The estimated states and their deviation from the perfect values over time is displayed. The used objective function is described by the equation of the cylindrical model and does not comprise the state $ \phi_{\mathrm{MCP}} $. Therefore the deviation around \SI{40}{\second} is acceptable. For the constrained minimization one sensor reading is used to estimate the pose of the index finger. This model represents the best expectable results for estimating one state vector with one magnet, with $ \mu = 0.035 \pm 0.000 $. At the end, the system tends to show bigger deviations from the perfect values.}
\label{fig:11cylNa1}
% script: 160223_compareResults.py
\end{figure}
\FloatBarrier

In \ref{tab:timeOneFing} the mean time, needed for one estimation cycle is listed in seconds. The computation time can be seen as a measure of how many iteration cycles are needed by the solver. It can be observed, that the time increases with the determinedness of the system. This is only reasonable, since the algorithm has more equations to take into account and to evaluate. Also the constrained methods show a faster timing behaviour, than the unconstrained. As a reason the restricted search space of the solver could be mentioned. To reach a reasonable result, the solver needs less iterations. The reduced system state by neglecting the adduction-abduction movement is also faster than the model, comprising this state, what is because of the reduced system size. The objective function, formulated with the dipole model shows also a faster evaluation time, compared to the one using the cylindrical. Since the cylindrical model represents a numerical approximation, which has to be evaluated at each iteration, the time consumption for evaluation is higher.\\
By comparing the quality of the solver with its timing behaviour, it can be stated that an increase in precision comes with higher computation times. For this example, using the perfect simulated data for the magnetic field, the estimation is not always fast enough, to match the observed sensor system frequency of \SI{20}{\Hz}. However, it is evaluated, that the estimation results won't degrade drastically, if one or two measurements would be skipped, due to the computation time. The actually estimated system state is only used as initial starting guess for the next estimation. It is observed that the solver is capable to intercept changes of a minimum of $ \pm 0.2 \si{\radian} $ between two measurements. So the initial starting point has a less important role for the solvability. For the assumed maximum angular velocity of \SI[per-mode=symbol]{0.175}{\radian \per \second} this change would reflect to a missing of one data set. What is more critical is the capability of estimating the state almost at real time. For the used simulated magnetic field values the best configuration for the minimizer to estimate the system state with an adequate frequency would be given by using the cylindrical magnetic model with adduction-abduction and taking the anatomic constraints into account. This would result in an estimation frequency of around \SI{7}{\Hz}, since the time needed to solve the problem is about \SI{0.148}{\second}. Compared to other hand tracking systems, this value is not good. However, for getting a rough feedback on the actual finger state, this value should be sufficient.
\begin{table}[!htb]
\centering
\begin{tabular}{l l c c c c}
\toprule
 & &         			\multicolumn{2}{c}{Unconstrained}		 & 	\multicolumn{2}{c}{Constrained}\\ \cmidrule(lr){3-4} \cmidrule(lr){5-6}
 & & 								Dipole & Cylindrical & Dipole & Cylindrical \\ \midrule[2pt]
\multirow{2}{*}{11} & no ad-ab    & 0.037 & 0.077 & 0.008 & 0.017 \\ 
					& ad-ab		 & 0.089  & 0.119 & 0.029 & 0.037 \\ \midrule
\multirow{2}{*}{12} & no ad-ab    & 0.063 & 0.139 & 0.014 & 0.031 \\ 
					& ad-ab		 & 0.114 & 0.214 & 0.031 & 0.074  \\ \midrule
\multirow{2}{*}{14} & no ad-ab    & 0.110 &  0.251 & 0.025 & 0.059 \\ 
					& ad-ab		 & 0.216 & 0.409 & 0.056 & 0.148 \\										
\bottomrule
\end{tabular}
\caption[Time for one estimation step of one finger state vector]
{The table allows a comparison of the mean evaluation time (in seconds), needed for the presented minimization methods for estimating one finger state vector. The time is displayed in seconds. Note the influence of using a constrained or unconstrained method. Also a reduction in evaluation time can be observed while shrinking the size of the measurement vector $ \tilde{\mathrm{M}} $. The fastest method shows an evaluation time of \SI{0.029}{\second} for estimating a single state vector. However, the quality of the results is not satisfying. For more exact estimated values, the time lies around 0.074 to \SI{0.148}{\second}. This results in low estimation frequencies, however an almost real-time-like representation should be possible.}
\label{tab:timeOneFing}
\end{table}

\FloatBarrier
For estimating the movement of multiple fingers, an adequate motion pattern is deployed. The simulated sequence consists only of 100 datapoints, reflecting a measurement time of only \SI{10}{\second}. This short time period is chosen, since first tests showed a high time consumption of the estimation stage. The utilized motion is visualized for each finger and each state in \ref{fig:multiFing}. As visualized in the figure, the fingers are moving individually, to test whether the estimation is capable of that.\\
\begin{figure}[!htb]
\centering
\includegraphics{pictures/plots/multiStates.png}
\caption[Introduced movement pattern for four finger estimation]
{The introduced motion pattern for the estimation of the state vectors for multiple fingers. In each column the states for one finger are displayed. The motion is deployed to represent individual movements of the finger, to check whether they cause a reasonable influence on the magnetic field to be estimated. Therefore each finger state itself is slightly different to the other.}
\label{fig:multiFing}
% python script: 160224_plotSequenceMulti.py
\end{figure}
For getting an insight, how good the states for multiple magnets can be estimated, several sensor-magnet configurations are simulated. The evaluation is done for two fingers (the index and middle) and all four. As learned from the previous results, four sensors are used for the estimation of two fingers, to ensure overdeterminedeness. However, for estimating all four finger state vectors, the introduced system can only satisfy determinedness. For reasons of completeness, four additional sensors were introduced to the simulation, placed behind the four existing ones, to achieve a value of $ N = 8 $. As for the estimation of one finger, the values are simulated using the cylindrical model. The results are listed in \ref{tab:multFing}. The corresponding parameters are coded in the same manner as beforehand.\\
\begin{table}[!htb]
\centering
\begin{tabular}{l l c c c c}
\toprule
 & &          				\multicolumn{2}{c}{Unconstrained}          &		\multicolumn{2}{c}{Constrained}\\ \cmidrule(lr){3-4}\cmidrule(lr){5-6}
 & & 								Dipole   			   & Cylindrical 	 			 & 		Dipole 			& 		Cylindrical \\ \midrule[2pt]
\multirow{2}{*}{24} & no ad-ab    & $ 0.119 \pm 0.000 $ & $ 0.081 \pm 0.000 $ & $ 0.051 \pm 0.000 $ & $ 0.039 \pm 0.000 $ \\ 
					& ad-ab		 & $ 0.114 \pm 0.000 $ & $ 0.000 \pm 0.000 $ & $ 0.085 \pm 0.000 $ & $ 0.005 \pm 0.000 $ \\ \midrule
\multirow{2}{*}{44} & no ad-ab    & $ 0.941 \pm 0.006 $ & $ 0.484 \pm 0.001 $ & $ 0.314 \pm 0.000 $ & $ 0.216 \pm 0.000 $ \\
					& ad-ab		 & $ 1.361 \pm 0.022 $ & $ 0.024 \pm 0.000 $ & $ 0.223 \pm 0.000 $ & $ 0.140 \pm 0.000 $ \\ \midrule
\multirow{2}{*}{48} & no ad-ab    & $ 0.543 \pm 0.001 $ & $ 0.509 \pm 0.001 $ & $ 0.236 \pm 0.000 $ & $ 0.183 \pm 0.000 $ \\ 
					& ad-ab		 & $ 0.494 \pm 0.000 $ & $ 0.005 \pm 0.000 $ & $ 0.385 \pm 0.000 $ & $ 0.098 \pm 0.000 $\\										
\bottomrule
\end{tabular}
\caption[Quality of the different minimization methods for estimating multiple fingers]
{The error mean and standard deviation for the different system configurations and minimization methods in \si{\radian} are listed. The values are given in radians. To achieve overall acceptable results, the overdeterminedness is again critical. The deployed system, consisting of four sensor units is therefore barely suited to estimate all four finger states which can be observed by an minimum error of \SI{0.024}{\radian}. As an extra, four additional sensor units were simulated. The results for this method show that it is in the end possible, to reconstruct four finger state vectors, as long as the system is overdetermined.}
\label{tab:multFing}
\end{table}
By looking at the results for the estimation of the state vectors for multiple fingers, a similar behaviour as mentioned for the fully determined case beforehand can be obtained. However, one interesting change can be observed. The unconstrained minimization method, described by the cylindrical model and taking $ \phi_{\mathrm{MCP}} $ into account shows here a better behaviour, than the constrained one. This is observable for each configuration of $ N $ and $ K $. One reason could be, that the \ac{BFGS} algorithm gives for those cases a better approximation for the search direction, than the constrained \ac{SLSQP} method. With the increasing number of system states, also the complexity increases. The constrained solver reaches its bounds, by using not good enough search directions. The unconstrained method however has more freedom, to look in each direction. For the estimation of two finger states, the unconstrained method using the cylindrical model and taking $ \phi_{\mathrm{MCP}} $ into account leads to the best results. For the estimation of all four fingers however, the minimization is not capable to reflect the perfect system states anymore. The overall smallest error for the estimation of four fingers with four sensors is \SI{0.024}{\radian}. The information, provided by the simulated measurement units is not sufficient, to describe the variables. For the actually built system, comprising four sensors, an estimation of all four fingers is expected not lead to reasonable results. For getting an impression on the estimated states, compared to the perfect ones, those obtained values are plotted in \ref{fig:difMult}. \\
However, by introducing four additional sensors (case ``48''), the results are getting better. A mean error of \SI{0.005}{\radian} is observed by the unconstrained method, using the cylindrical model with adduction-abduction. However, as stated beforehand, the introduction of such a high number of magnets would break the goal of constructing a mobile and unobtrusive system. \\
By looking at the required estimation time of the several methods, a tremendous increase can be observed (see \ref{tab:timeMultFing}). This is not only due to more iterations, but mainly induced by the higher system states $ N $ and $ K $. To still observe reasonable results of the estimated states, more than \SI{1}{\second} is needed. This can be observed by almost all minimization configurations. This means a proper real time evaluation of the finger pose estimation is not possible anymore. By increasing the number of sensors $ N $ to 8, about \SI{17}{\second} would be needed to achieve reasonable results, which is obviously far away from real time behaviour or acceptance for post processing. \\
\begin{figure}[!htb]
\centering
\includegraphics{pictures/plots/difMult.png}
\caption[Estimated states of four sensors and four magnets]
{The displayed results are for the objective function which comprises $ \phi_{\mathrm{MCP}} $ and is formulated with the cylindrical model equation. The unconstrained minimization estimates on the base of four simulated sensor readings the state vectors of four fingers. The estimated and perfect states are plotted for each finger and their observed deviation over time. The mean difference over all for fingers is $ \mu = 0.024 \pm 0.000 $. The highest deviation can be recognized for $ \theta_{\mathrm{MCP}} $ of the middle finger. The states of the other fingers can be estimated pretty well.}
\label{fig:difMult}
% script: 160224_compareResMulti.py
\end{figure}
\begin{table}[!htb]
\centering
\begin{tabular}{l l c c c c}
\toprule
 & &         			\multicolumn{2}{c}{Unconstrained}		 & 	\multicolumn{2}{c}{Constrained}\\ \cmidrule(lr){3-4} \cmidrule(lr){5-6}
 & & 								Dipole & Cylindrical & Dipole & Cylindrical \\ \midrule[2pt]
\multirow{2}{*}{24} & no ad-ab    & 0.920 & 1.632 & 0.291 & 0.382 \\ 
					& ad-ab		 & 2.129  & 3.346 & 0.602 & 1.275 \\ \midrule
\multirow{2}{*}{44} & no ad-ab   & 3.365 & 5.012 & 0.629 & 0.947 \\ 
					& ad-ab		  & 8.322  & 8.419 & 1.696 & 2.684 \\ \midrule
\multirow{2}{*}{48} & no ad-ab    & 7.130 &  9.670 & 1.137 & 1.988 \\ 
					& ad-ab		 & 14.346 & 17.558 & 3.945 & 4.677 \\										
\bottomrule
\end{tabular}
\caption[Time for one estimation step of multiple finger state vectors]
{The mean evaluation time in seconds for estimating multiple finger state vectors is listed above. It is observable, that the estimation of four fingers is far away from real time behaviour. This, in combination with the observations regarding the quality of the estimation, no good results are expected for real measurements. Only the estimation of the states for two fingers show adequate timing behaviour.}
\label{tab:timeMultFing}
\end{table}
\FloatBarrier
The presented results visualize the behaviour and influence of different system configurations $ N $ and $ K $ for different ways of describing the minimization problem. Note that the estimations are based on perfect, simulated magnetic field values. The following concluding statements can be derived:
\begin{itemize}
\item To get a reasonable accuracy for the estimated states, the system has to be overdetermined, i.e. $ N > K $.
\item The number of function iterations (and therefore the estimation time) increases significantly with the size $ K $ of the system state.
\item An estimation of four fingers with the designed system, consisting of four sensors is expected to be barely possible in an adequate quality or real time behaviour.
\item The state $ \phi_{\mathrm{MCP}} $ for adduction-abduction introduces higher estimation times, but can be estimated and should be used, to better reflect the human hand motion.
\end{itemize}
Since the results are based on perfect simulated magnetic flux densities from the cylindrical model, the estimation procedures comprising this model lead also to better results. It is evaluated, that the cylindrical method, including adduction-abduction and anatomic constraints leads to the overall best results for the estimation of one finger. However, when porting the observations to real measurements on a human hand, one has to note that several additional distortion factors are added to the system, like the inexact position information of the finger and sensor dimensions or the surrounding magnetic field. Therefore in the ongoing estimation of real datasets, the cylindrical and the dipole method (both including the state $ \phi_{\mathrm{MCP}} $ and constraints), are both used for the state estimation.

\FloatBarrier
\subsection{Results for Recorded Datasets} \label{subsec:resMeas}

\subsubsection{Recording Procedure}\label{subsubsec:recSetup}

For the evaluation of the system at a real use case on the hand, the following setup is established. The proband wears the developed sensor unit and a number of magnets on the fingertips. The obtained magnetic field values are received and saved by a Laptop, to perform the state estimation phase afterwards. In this way specific parameters of the objective function or the system, like the number of sensor readings taken into account, can be adjusted later on and the results can be compared. The quality of the system is further compared to the Leap Motion \cite{leap}. Therefore, the sensor data is recorded with the hand held above this device. For a rough qualitative judgement, a video of the performed movements is recorded. A poster with a circle and angular ticks is installed behind the motion area. A photo of the whole setup is shown in \ref{fig:setup}. Each recorded dataset consists of the states, obtained by the vision based Leap Motion system and the sensor readings from the measurement units. For the fitting of the sensor values to the hand measured joint and finger positions, each set comprises the introduced calibration gesture of bending the fingers around the \ac{MCP} joints about \SI{90}{\degree}. Since the dynamic cancellation of the earth magnetic field is not possible with the method presented in \ref{subsec:earthEli}, the hand was tried to be held calm and at a constant position throughout the whole recording session. So in the end, by this setup a comparison between the presented magnetic approach and the existing vision based Leap Motion system is possible.
\begin{figure}[!htb]
\centering
\begin{tikzpicture}
\centering
\node[anchor=south west,inner sep=0] (image) at (0,0,0) 
	{\includegraphics[width=0.8\textwidth]{pictures/setup22.jpg}};
\begin{scope}[x={(image.south east)},y={(image.north west)}]
    \draw[solid,red,<-,line width=1.5] (0.85,0.85) -- +(0,0.16)node[anchor=south] {Box with angle poster};
    \draw[solid,red,<-,line width=1.5] (0.7,0.73) -- +(0,0.1)node[anchor=south] {Leap Motion};
    \draw[solid,red,<-,line width=1.5] (0.23,0.95) -- +(0.,0.08)node[anchor=south] {Camera};
    \draw[solid,red,<-,line width=1.5] (0.4,0.1) -- +(0.1,0.)node[anchor=west] {PC};
    \draw[solid,red,<-,line width=1.5] (0.7,0.5) -- +(-0.1,-0.12)node[anchor=north] {Magnetic system};
\end{scope}
\end{tikzpicture}
%\includegraphics[width=0.8\textwidth]{pictures/setup22.jpg}
\caption[Measurement setup]
{The measurement setup with all relevant parts.}
\label{fig:setup}
\end{figure}


\FloatBarrier
\subsubsection{Evaluation, Discussion and Comparison to Leap Motion and Video Data}

The following paragraph presents and discusses the difference of the estimated states by the magnetic system with the data, returned by the Leap Motion. Both systems show different acquisition frequencies. While the magnetic system works at the evaluated \SI{20}{\Hz}, the Leap Motion shows a framerate of \SI{110}{\Hz}. For a direct comparison between the two systems, the data from the Leap is downsampled. For this, the magnetic and the Leap system save a timestamp, which represents the uptime of the system when the actual reading is observed. To adapt the Leap data to the magnetic system, only the data with the timestamp, the closest to the one of the magnetic system is chosen. This method was evaluated to do the resampling in a representative manner, without loosing too much information or degrading the data unintentional. 
The presentation and discussion is structured similar to \ref{subsubsec:simEval}. So at first the state vector of only one finger (the index) is estimated. The observations, made in the previous chapters are deployed for the formulation of the ongoing estimation problem. Therefore the finger state vector includes $ \phi_{\mathrm{MCP}} $ and the results are based on the minimization algorithm \ac{SLSQP}, which takes the anatomic constraints into account. 
Therefore the results of six different combinations of formalizing the optimization problem can be compared and evaluated. Those are: Describing the objective function with the dipole or the cylindrical model and using the measurements of one, two or all four sensor units. The comparison to the Leap will show, which influence the determinedness of the system has on real measurements and whether the states could be better estimated by describing the objective function with the cylindrical bar magnet or the magnetic dipole model. Since the previous results on simulated data showed that only the estimation of a single finger state vector leads to reasonable outcomes, the emphasize of the evaluation is put onto the calculated finger states of the index finger.

15 datasets were recorded, each consisting of the state values, returned by the Leap system and the corresponding measurements of all four sensor units, excited by the magnet on the index finger tip. As already described, each set of obtained magnetic flux densities includes the initialization gesture, to compensate the hand measured system dimensions and to eliminate the surrounding magnetic field. Therefore, as an introductory step this movement has to be extracted to determine the scaling factors for each sensor unit, which have to be applied to the respective measurements. Based on those values, the states for the index finger were estimated. The difference vector to the states, returned by the Leap Motion system is calculated and normed for each estimated state, to get an overall measure for the deviation for each estimated set. Since the sets differ not only in the duration from each other but also by the speed and complexity of the performed movements, they have to be examined individually. Also if all sets would comprise the same length and the same predefined movements, a cumulative mean and standard deviation would not serve good and representative values for those. This is induced, since the system is conditioned by many external influences, like the often mentioned hand size parameters, the continuity of the hand's height and orientation and the speed and movement range of the motion. However, to proper evaluate the recorded sets and to compare them to each other, the method, showing the smallest error mean for one recorded set is determined. The results are listed in \ref{tab:estSet}. For each set the best parameters of the optimization method are coded by \mbox{\emph{cyl/dip\_ $ K N $}}, with
\begin{itemize}
\item \emph{cyl} meaning that the cylindrical model and 
\item \emph{dip} saying that the dipole model is used to formulate the objective function
\item $ K $, the number of finger state vectors to estimate and
\item $ N $, the number of sensors, taken into account for the estimation.
\end{itemize}
So for example ``cyl\_12'' means, that the objective function is formulated with the cylindrical model for estimating one finger state vector ($ K = 1 $), by using the measurements from two sensor units ($ N = 2 $). In the case of $ N = 1 $, the sensor beneath the index finger is used, for $ N = 2 $, the one under the middle finger is added and $ N = 4 $ means that the measured values of all four sensors are taken into account.
\begin{table}[!htb]
\centering
\begin{tabular}{l c c}
\toprule
 & Method & $ \mu [\si{\radian}] \pm \sigma^{2} $ \\ \midrule 
\textbf{Set 1}  & cyl\_12 & $ 0.581 \pm 0.020 $ \\ 
\textbf{Set 2}  & cyl\_12 & $ 0.587 \pm 0.012 $ \\ 
\textbf{Set 3}  & dip\_12 & $ 0.646 \pm 0.010 $ \\ 
\textbf{Set 4}  & cyl\_12 & $ 0.467 \pm 0.017 $ \\ 
\textbf{Set 5}  & cyl\_14 & $ 0.668 \pm 0.027 $ \\ 
\textbf{Set 6}  & cyl\_12 & $ 0.495 \pm 0.011 $ \\ 
\textbf{Set 7}  & cyl\_14 & $ 0.567 \pm 0.026 $ \\ 
\textbf{Set 8}  & cyl\_14 & $ 0.670 \pm 0.052 $ \\ 
\textbf{Set 9}  & dip\_14 & $ 0.606 \pm 0.024 $ \\ 
\textbf{Set 10} & cyl\_14 & $ 0.581 \pm 0.010 $ \\ 
\textbf{Set 11} & cyl\_14 & $ 0.603 \pm 0.004 $ \\ 
\textbf{Set 12} & dip\_14 & $ 0.676 \pm 0.022 $ \\ 
\textbf{Set 13} & cyl\_14 & $ 0.680 \pm 0.018 $ \\ 
\textbf{Set 14} & cyl\_14 & $ 0.525 \pm 0.010 $ \\ 
\textbf{Set 15} & cyl\_14 & $ 0.672 \pm 0.012 $ \\ \bottomrule
\end{tabular}
\FloatBarrier


\caption[Parameters for the estimation results, showing the smallest difference to the Leap Motion]
{The table lists the methods, whose estimated states show the smallest difference to the one obtained by the Leap Motion system. The provided values represent the lowest mean and standard deviation in \si{\radian} for each set individually. The states from the Leap system serve here as the truth values. The results for Set 3, estimated with the dipole model by using two sensor units shows the overall smallest difference to Leap system. It can be noted that the average error of those best estimated states is relative high. One has to note, that at each set different motion patterns were performed. Also the duration of each set varies. Therefore a direct comparison between them would not lead to a representative statement.}
\label{tab:estSet}
\end{table}
The time needed for the estimation of a measurement set is also reported. Here, almost no differences to the results, obtained in \ref{subsubsec:simEval} are observed. The timing behaviour of the slowest set was evaluated to be \SI{0.092}{\second} and is observed by a method using four sensor readings. This value is in fact higher, than the system frequency, but the results are expected not to degrade with a loss of measurement sets. Since by the recognized maximum evaluation time at most one data set gets neglected. Regarding the realtime capability, the respective worst estimation frequency would be \SI{10}{\Hz}. Since those results are not very surprising and were already discussed in \ref{subsubsec:simEval}, the time values are not further explained here.

From \ref{tab:estSet}, the observations regarding the formulation of the optimization method from \ref{subsubsec:simEval} are confirmed. The system has to be at least overdetermined, since no good results are reached by the method using only one sensor unit for the estimation. Also taking as much measurements into account as available leads more often to better results, than taking only two sensors into account. As an explanation one can name the same reason as before, the minimizer has more information about the actual system state and can therefore find a more exact solution for the problem. However, 5 out of 15 sets show better results with $ N = 2 $, than with $ N = 4 $. Also the two sets with the lowest difference to the Leap states use only two sensor values. This can be putted down to faulty measurements, induced by unstable positions and alignments of the hand during the data acquisition. By using faulty measurements it is harder to find a solution. Introducing a higher number of those leads to a worsening of the results. Therefore sometimes it is better to take less measurements into account, if one knows that they are bad. By taking more sensor units into account, one puts a higher trust into them. Regarding less, one relies more on the capabilities of the minimization method. In the end it's a trade off between both.\\
Moreover it can be stated, that the objective function, described by the cylindrical bar magnet model leads to better results as the dipole model. So it is  verified, that this model describes the magnetic field, induced by a cylindrical bar magnet, better, than the approximation with the dipole model. However, 3 out of 15 datasets return better results for the dipole model. But by looking a bit closer to the errors, returned by the other minimization methods for these data sets, one recognizes, that the difference to the error, observed by a cylindrical method is only slightly smaller. For example for Set 3, the results for the \emph{cyl\_12} method are only worse by \SI{0.008}{\radian}. \\
So in the end, the smallest deviation to the observed states from the Leap motion can be provided most often from the highest overdetermineded minimization method, formulated with the cylindrical model equation, \emph{cyl\_14}. In numbers: 8 out of 15 sets. However, set 4 is showing the smallest deviation over all recorded datasets and uses only two sensor units for the estimation (\emph{cyl\_12}). Set 4 shows a mean and standard deviation of $ 0.467 \si{radian} \pm 0.027 $. By regarding at the general quality of the states, obtained of the magnetic estimation versus the data from the Leap system, one can note that the errors are pretty high. The mean over all sets is $ 0.602 \si{\radian} $, which corresponds to a difference of \SI{34.5}{\degree}. As mentioned beforehand this value has to be handled with care, since each dataset shows a different size and motions. That the presented magnetic system is despite that big difference capable to track the finger motions under certain conditions is further evaluated. It will be determined which motions and effects cause this high error value. For the visualization of the returned states of both systems, set 4 is plotted in \ref{fig:bestLeap}. An explanation to the magnitude and the characteristics of the differences is provided a bit later in this section. 

As next step, the results for estimating four finger state vectors are presented. The recording procedure is done in the same way as for one state vector, beside that now each finger is equipped with a magnet on its tip. The estimated finger state vectors are each compared individually to the corresponding data of the Leap system. In this way, the means and standard deviations of the differences between the two systems are determined for each finger individually. Mind, that the state estimation problem has now the following size: $ N = 4 $, $ K = 4 $. As additional parameter for the estimation phase, the problem is concerned to be constraint and to include the state $ \phi_{\mathrm{MCP}} $. Only the type of the objective function is varied. It was evaluated, that the one, described by the cylindrical model leads better results, compared to the states observed by the Leap Motion. The mean values for the difference of each finger state vector to the Leap data and the corresponding standard deviation over the sets are presented in \ref{tab:estSetFour}.
\begin{table}[!htb]
\centering
\begin{tabular}{l c c c c c}
\toprule
&  \multicolumn{5}{c}{$ \mu \si{\radian} \pm \sigma^{2} $} \\ \cmidrule{2-6}
& 			   				 Index 				 & Middle 			   & Ring 				 & Pinky 			   &  Cumulative \\ \midrule
\textbf{Set 1} &  $ 0.918 \pm 0.053 $ & $ 0.975 \pm 0.044 $ & $ 0.746 \pm 0.029 $ & $ 0.709 \pm 0.008 $ & $ 0.837 \pm 0.000 $ \\ 
\textbf{Set 2} &  $ 1.077 \pm 0.039 $ & $ 1.066 \pm 0.107 $ & $ 0.912 \pm 0.043 $ & $ 0.618 \pm 0.015 $ & $ 0.918 \pm 0.001 $ \\ \bottomrule
\end{tabular}
\caption[Difference of estimated states, compared to Leap Motion data for four finger estimation]
{The mean and standard deviation from the obtained estimation results to the Leap Motion data. Four finger state vectors are estimated. Since the obtained results show such a big difference to the Leap data and the actual performed movement, only two datasets were recorded. The high values for the mean difference and standard deviation for each finger, show that an estimation of four fingers with the deployed system is not possible.}
\label{tab:estSetFour}
\end{table}
\FloatBarrier


For the case of $ N = 4 $, only two datasets were recorded, since the results show similarities and the calculation time is quite long. For set 1 the average time per estimation step is measured to be \SI{0.837}{\second}, for Set 2 even \SI{0.918}{\second} were observed. Therefore an estimation with real time behaviour would not be possible anymore. While for the estimated results of one finger state, the difference to the Leap system is already pretty high, here a further increase is observable. The two sets comprise finger movements, which are performed simultaneously by all four fingers and alone, by only a single finger. In this way it can be checked whether the systems can distinguish between separate finger motions or not. The direct comparison of the two system states against each other shows that the magnetic estimation approach is not capable to identify individual finger movement reliably. \ref{fig:est44} tries to proof this visually. The norm of the states for the index and the middle finger, obtained by the Leap system and the magnetic estimation are plotted over time. The norm over all states for one finger is chosen, since it represents a measure for the actual predicted bending of the joints. 
\begin{figure}[!htb]
\centering
\subfloat{\includegraphics{pictures/plots/est44.png}}\\
\subfloat{\includegraphics{pictures/statePics/est44/set44s.jpg}}
\caption[Estimating the motion of four fingers]
{The norm over the states for the index and the middle finger are plotted. This should resemble a measure of the actual amount of bended angles of the finger, but does not reflect actual individual joint states. During the first \SI{17}{\second} parallel movements of all fingers are performed. Both systems show an increase for this. However, when it comes to individual movement of the fingers, the states of the magnetic system don't represent the truth anymore. At \SI{20}{\second}, the single flexion-extension of the index finger is also estimated for the middle. Furthermore between \SI{30}{\second} and \SI{35}{\second}, a movement of the pinky finger is performed, but the states are changing for the middle and index finger, which are held still. The states of the Leap motion system represent the truth much better. The pictures beneath the state figures serve as a rough visual reference.}
\label{fig:est44}
% script: 160226_leapVsEst.py
\end{figure}
In combination with the provided pictures, extracted every \SI{5}{\second} from the recorded videostream, \ref{fig:est44} serves as an example to proof, that individual finger movements can not be estimated reliably by the magnetic system. At the beginning of the short sequence, two movements which are performed by all four fingers were executed. Here the Leap and the magnetic system show both a change for the angles of the index and middle finger. Therefore both systems return in some sense the truth. The measurements of the magnetic system are fitted to the initialization gesture, which is performed around \SI{5}{\second}. However, when it comes to individual finger movements, the states for the presented fingers are estimated wrong. Around \SI{20}{\second} only the index finger is bent. This is captured by the Leap Motion correctly, since the state of the middle finger stays almost at \SI{0}{\radian}. Also the other finger states, which are unseen here for visualization reasons, are almost at \SI{0}{\radian}. The magnetic system however estimates an additional excessive change for the middle finger, which does not happen. During \SI{30}{\second} and \SI{35}{\second} a flexion-extension is performed by the pinky finger. However, the estimated states of the magnetic system during this time interval interpret a movement of the index and middle finger. The Leap system again reflects the right angles and shows only small changes for the two presented finger state vectors. This behaviour of the magnetic system can be obtained almost every time when individual finger movements occur. As a reason for the bad estimation results, one could head the following: On the one hand, the system is only fully determined, which degrades the results for the estimation, as observed in \ref{subsubsec:simEval}. On the other hand, the obtained changes of the magnetic field, induced by the movement of a single finger are only small. The parameters for the hand dimensions can only be determined up to a certain accuracy, additional erroneous contributions are introduced. The optimizer tries to fit the values to a slightly different hand model and can not reach reasonable results. The estimated angles could be improved, by acquiring more exact values for the hand dimensions and by taking the readings of more sensor units into account. However, since the deployed system consists only of four sensors, this is not further evaluated. In the end the presentation of this short example for the results of estimating the states of four magnets by using four sensor units shows, that no truthfully values can be estimated for this system configuration.
\FloatBarrier

The estimated values, obtained for the at first presented predictions of a single finger state vector showed a smaller difference to the states from the Leap Motion. On the basis of those recorded datasets and their results, a more detailed comparison to the Leap system is further presented, to identify the capabilities and drawbacks of the magnetic system. The reduced state size ($ K = 1 \rightarrow size(\mathrm{X}_{1}) = 3 \times 1 $) allows an easier examination of the results and possible sources of error. The estimated finger states of dataset 4 showed the smallest difference to the angles obtained by the Leap system. The values for the finger state vector of both systems are plotted in \ref{fig:bestLeap} over time.
\begin{figure}[!htb]
\centering
\subfloat{\includegraphics{pictures/plots/bestEst.png}}\\
\subfloat{\includegraphics{pictures/statePics/bestLeap/bestPic.jpg}}
\caption[Comparison of estimated states, which fit best to Leap data]
{Each state value of the index finger, obtained by the Leap Motion and the magnetic estimation are plotted over time. The results are for set 4, which shows the smallest deviation between both systems. The values for $ \theta_{\mathrm{MCP}} $ show the most similarities. $ \theta_{\mathrm{PIP}} $ and $ \theta_{\mathrm{DIP}} $ show common directions, however the estimated states are much higher. For the movement of adduction-abduction the Leap Motion recognizes a more restless behaviour but they also have common phases. The difference, normed over all four states is plotted at the bottom. Here the differences for $ \theta_{\mathrm{PIP}} $ and $ \phi_{\mathrm{MCP}} $ show the highest impact.}
\label{fig:bestLeap}
% script: 160226_leapVsEst.py
\end{figure} 
As a first impression and especially focusing on the values of $ \theta_{\mathrm{MCP}} $, the two systems show a quite common angle prediction. The initialization gesture, which happens at the beginning at around \SI{5}{\second}, is responded by both systems as nearly a bare movement of the \ac{MCP} joint. Only the Leap shows here contributions of adduction-abduction, which actually did not happen. This behaviour can be recognized right at the following gesture till \SI{15}{\second}, again. During this time, the beforehand mentioned movement is performed once again, just a bit slower. By regarding the magnetic estimation, some none smooth peaks for $ \theta_{\mathrm{MCP}} $ are observable over the whole set. They mainly occur, at the time, when a change of $ \theta_{\mathrm{PIP}} $ is estimated and the finger is bent to a fist. It is assumed, that at those points the solver can't find an optimal solution. It is also to note here, that for the estimated angles of the \ac{PIP} and \ac{DIP} the biggest differences between the two systems are observable. Note, that those two states are estimated as a combined one by the magnetic approach, since the anatomic condition $ \theta_{\mathrm{DIP}} = \frac{2}{3} \theta_{\mathrm{PIP}} $ is used. The direction returned by both systems is the same, which means that both show a parallel increase or decrease of the angles. However, the states from the magnetic system are much higher than the ones from the Leap Motion. Furthermore, the movement of adduction-abduction shows remarkable differences. The estimated values of $ \phi_{\mathrm{MCP}} $ by the magnetic system show a more stable behaviour than the ones returned by the Leap. It should be noted, that during the movement of flexion-extension the motion was tried to be performed with very small lateral movement. Nevertheless for this state both systems show also similarities. For example between \SI{40}{\second} and \SI{50}{\second}, the motion in negative direction and back is captured by both systems. Also the other three finger states have almost no contributions during this time interval. The pictures which are extracted each 5s from the video allow a qualitative comparison of the both systems.
\FloatBarrier


% % % drawbacks Leap
% massive ad-ab movement
For the presented dataset the estimation results were only compared to the Leap Motion, which is assumed to return the ground truth for the actual state. As already observed for set 4, those values are also not totally perfect and constant over time, which is shown for example by the very high lateral changes for $ \phi_{\mathrm{MCP}} $ in positive direction (at around \SI{5}{\second} and \SI{12}{\second}). The recorded angles indicate, that the bones move about \SI{0.8}{\radian} (=\SI{45}{\degree}) towards the middle finger, which was definitely not performed. Most of the datasets from the Leap show a high deviation for the state of adduction-abduction from the de facto values for this. For example in \ref{fig:set14} a similar behaviour during the movement to a fist is visualized. This time however the recorded contribution of $ \phi_{\mathrm{MCP}} $ at \SI{14}{\second} is negative. The observed values at around \SI{20}{\second}, where again the finger is bent to a fist, show a more or less small fluctuation and can therefore be stated as an evidence, that the false motion is not always detected. Over all the de facto performed movement did not comprise such high lateral motions. By the induced constraint and intuition from natural hand movement, it is accepted, that a maximum range of motion from \SI{-15}{\degree} to \SI{+15}{\degree} (\SI{-0.262}{\radian} to \SI{+0.262}{\radian}) is possible. During most of the time the recorded motions were performed to mainly show contributions of flexion-extension, therefore the view of the camera is also aligned to capture those movements best. Unfortunately, an exact value about the de facto size of the deviation from the real state of $ \phi_{\mathrm{MCP}} $ to the predicted cannot be stated. However, it can be stated, that the Leap shows here quite often values, which do not represent the truth. One reason for this could be the underlying method for the detection of the bone and hand directions. The Leap Motion provides normalized direction vectors for each finger and the palm. For calculating the angle of adduction-abduction from this, the angle between the direction vector of the proximal index bone and the palm, relative to the palm normal is determined.\\
\FloatBarrier
% movement of PIP induces movement of DIP
Further, concerning the behaviour of the Leap system, one can head that a kind of relationship between $ \theta_{\mathrm{DIP}} $ and $ \theta_{\mathrm{PIP}} $ exists. In almost every set, a motion of the \ac{PIP} joint introduces also a change of $ \theta_{\mathrm{DIP}} $. In \ref{fig:set14} this behaviour is presented. The observed states for $ \theta_{\mathrm{PIP}} $ and $ \theta_{\mathrm{DIP}} $ are plotted over time with the corresponding parts from the video, placed beneath. As an additional verification to the beforehand mentioned false interpretation of the adduction-abduction angle, those states are also plotted. This observation should not be stated to be false or introduce erroneous system states. The developed magnetic estimation assumes even a static relationship between those two state values. The Leap system verifies this assumption in some way. As stated in \ref{sec:anatomy} it is quite usual to assume the observed relationship. However, from the information available for the Leap, a hard programmed explanation of this behaviour is not provided.\\
\begin{figure}[!htb]
\centering
\subfloat{\includegraphics{pictures/plots/set14leap.png}}
\caption[Relationship between $ \theta_{\mathrm{PIP}} $ and $ \theta_{\mathrm{DIP}} $ observed by Leap Motion]
{The states, provided by the Leap Motion for performing two times a fist. By regarding the flexion-extension angles for \ac{PIP} and \ac{DIP}, the introduced intra finger relation between those values is verified. However, the Leap system does not show a static relation between those two angles, as it is assumed by the magnetic estimation. The states of $ \phi_{\mathrm{MCP}} $ are provided additionally, to visualize once more, that this value often returns erroneous state configurations. During the plotted motion sequence a change of this angle about the observed amount was definitely not performed. While performing the second fist, only small false motions are observed and therefore show that the returned values are not always erroneous.}
\label{fig:set14}
% script: 160229_leapDip.py
\end{figure}
% small values of PIP while performing fist
Another observation, by regarding the state vector for performing a fist from the Leap, is that the values for $ \theta_{\mathrm{DIP}} $ and $ \theta_{\mathrm{PIP}} $ are relative low and $ \theta_{\mathrm{MCP}} \simeq \pi/2 $. For example at set 4 \ref{fig:bestLeap} between \SI{18}{\second} and \SI{22}{\second}. When examining the video data qualitatively, one can recognize that the angle of $ \theta_{\mathrm{DIP}} $ and $ \theta_{\mathrm{PIP}} $ are actually higher than \SI{0.5}{\radian}. This can be explained by the fact of occlusion. While crooking the finger to a fist, especially the distal bone and the tip are hidden by the other bones. By comparing the estimated angle of the \ac{PIP} joint from the magnetic estimation, one can judge qualitatively, that those reflect the real behaviour a bit better than the Leap. The presented observations for the predicted angles from the visual system prove, that it is also not totally free of errors. So one has to keep those presented drawbacks in mind, while examining the data from the Leap Motion system.
\FloatBarrier


% % % qualitative evaluation/comparison
% ++ motions similar to init movement
% ++ slow motions of ad-ab (set4)
By checking the estimation results of the magnetic approach for all sets qualitatively with the Leap Motion and the recorded video data, one could determine some sort of gestures, which can be reconstructed relative reliable and correct. On the one hand the gesture, where the values were fitted to can be observed very stable along one dataset. This sounds only reasonable, since the obtained magnetic flux densities are adjusted right for this movement. Therefore all motions, which induce the most changes of flexion-extension on the \ac{MCP} joint fall into this category, too. Those motions can be classified as ``pre-states'' of the initialization gesture and are therefore a subset of it. This is also observable by set 4, since at \SI{10}{\second} and \SI{5}{\second} the angle of \ac{MCP} is estimated to be smaller \SI{90}{\degree}, which is also observed by the Leap and can be verified qualitatively with the pictures. Furthermore, as already observed by set 4, a slow change of $ \phi_{\mathrm{MCP}} $ can also be tracked quite well. However, by assessing the results of other datasets, the reliability of the estimation for the lateral motion can not be generalized. Especially where the state of $ \phi_{\mathrm{MCP}} $ changes, while the \ac{MCP} joint is additionally in flexion.\\ 
% -- ad-ab more complex
By adding the angular velocity and therefore the change over time as a parameter, some additional weaknesses of the developed system can be judged. With faster motions, the estimation results are getting worse. As already stated in \ref{sec:dataRes}, the overall system frequency for acquiring data of all four sensor units is \SI{20}{\Hz}. As further introduced in \ref{subsec:resSim}, the maximum detectable angular velocity was determined to be \SI[per-mode=symbol]{0.5}{\degree \per \second}. Some recorded datasets include very fast finger motions, by which the maximum detectable angular change is exceeded. Especially for the estimation of small motions, like the reconstruction of lateral changes, an adequate number of measurements is important. A detailed statement for the maximum detectable angular velocity is not evaluated. However, with the provided video material, this could be a future task to be determined. In the end it can be stated, that rapid or staccato like movements can not be tracked reliably and the angular velocity has an impact, due to the overall system frequency. If the states would now also be estimate in real time, the overall results are expected to worsen only marginally, because the most time consuming part would still be the sensor system.\\
%However, \todo{formulate it as \grqq the overall motion velocity, angular velocity is not allowed to be to high \grqq} when it is performed to fast, the estimation is not capable anymore of this movement. This is assumed to be on the one hand induced by the small movement range of adduction-abduction, but the main reason is the overall system frequency. Since it is with \SI{20}{\Hz} quite slow, rapid or staccato like movements can not be tracked quite well. A faster data acquisition rate would therefore lead to probably better results. So to reconstruct the motion reliably with the magnetic system, they should be performed with an adequate speed. As an additional time critical factor one could state the estimation phase. Beforehand it was stated, that a maximum duration of \todo{how fast is the worst?} seconds would not really degrade the impression of real time data. For the overall system frequency this is also true. However if the data acquisition rate could be increased, the time limiting factor would be represented by the estimation time. However, since the speed of the motions was not tracked, a detailed value for the maximum angular velocity of the motion could unfortunately not be provided.
Another, quite common observed behaviour of the system are the implausible values for $ \theta_{\mathrm{MCP}} $. Some of those peaks are exemplary discussed for set 4 (\ref{fig:bestLeap}). However, by examining other datasets, this behaviour can often be recognized in an extreme variant, where the flexion-extension angle for the \ac{MCP} joint even becomes \SI{0}{\radian} and stays at this value for some time. In set 5, such cases are detected. A sequence of the estimation results for the angles of flexion-extension is plotted in \ref{fig:set5}.
\begin{figure}[!htb]
\centering
\subfloat{\includegraphics{pictures/plots/set5.png}}\\
\subfloat{\includegraphics{pictures/statePics/set5/set5.jpg}}
\caption[Occurrence of wrong estimated states for $ \theta_{\mathrm{MCP}} $]
{The estimated results for $ \theta_{\mathrm{MCP}} $ are often wrong, if a fist is performed. The results for $ \theta_{\mathrm{PIP}} $ and $ \theta_{\mathrm{DIP}} $ show more realistic values. As soon as the intermediate and distal phalanges get to close to the palmar side, the angle of \ac{MCP} is estimated as \SI{0}{\radian}. The visualized sequence of set 5 includes two fist motions, each performed a bit different concerning speed and process. It can be stated, that a fist cannot be detected reliably, due to unsatisfied evaluations of the minimization procedure.}
\label{fig:set5}
\end{figure}
% -- set5 -> fist
In the presented figure, the errors during the performance of a fist are not only just small fluctuations anymore. Here the estimated angle of $ \theta_{\mathrm{MCP}} $ goes to \SI{0}{\radian} and also stays there, while the fingers are bent. The plot represents two sequences for the flexion and extension for a fist and back. Each sequence is performed a bit differently, concerning speed and process. But as it can be observed, at a certain angle for $ \theta_{\mathrm{PIP}} $ and $ \theta_{\mathrm{DIP}} $, the values for \ac{MCP} become 0. This behaviour comes up, because the solver cannot find a solution within the provided bounds, for the actual system configuration. This can be traced back, to the erroneous dimensions for the provided hand model. As stated beforehand, the positions and lengths of the sensors, joints and bones can only be determined by hand and therefore errors are introduced to the hand model. The minimizer tries to find a solution for a hand with exactly those erroneous provided hand dimensional values. Since they represent not exactly the real hand, the solver cannot for every sensor value a suitable system state, for solving the problem. The initialization gesture, which basically should remove the surrounding earth magnetic field and scale the values exactly for those measurement errors, is performed by only bending the \ac{MCP} joint and not the other two. For the case of bending the fingers to a fist, the false determined lengths of the bones are assumed to cause the most errors here. By regarding the results for the Leap system, also a slight inaccuracy in the state representation can be observed. The angles for the \ac{PIP} and \ac{DIP} joints are represented for each fist quite differently. At the first time, very small values are returned. The second fist movement however shows much higher values. As said, the movements were performed a bit different each time, but the differences were definitely not as high as returned by the Leap Motion. As beforehand mentioned, the occlusion of the distal bones is responsible for that. At the first bending to a fist, the bones were probably detected not as good as for the second performed motion. \\
The errors, induced by this characteristic behaviour of the magnetic system cause the main differences, compared to the almost perfect states of the Leap System. So the determination and the positioning of the sensor and hand dimensions is one of the most important parts for describing the system. The good results for set 4 can somehow be seen as a lucky coincidence, where the parameters suited best. As described \ref{sec:evalHand}, the estimation of the hand dimension did not lead to reasonable results. For the presented datasets for estimating one finger state vector by up to four sensor units, the hand dimensions are defined by the three bone lengths, the 3D joint position and the four three dimensional sensor positions. Therefore 18 values have to be measured by hand and can introduce nonlinear errors to the underlying hand dimensions. The obtained estimation results show, that the compensation of those erroneous hand model parameters by a fitting gesture can lead to reasonable results under certain conditions.

%The rapid fluctuations especially for $ \theta_{MCP} $ can be stated here. Those are also recognized in set 4, for example at \SI{20}{\second} or between \SI{30}{\second} and \SI{45}{\second}. For the other sets, these peaks are even worse and also lead sometimes to a $ \theta_{MCP} = 0 $ for a few seconds, which could be immediately be stated as false for most cases, by qualitatively looking at the video. This behaviour is observed, when a fist is performed. One set, where this effect is recognized quite often is Set 5 \todo{make plot set 5}. The estimated states for flexion-extension are plotted and the corresponding video parts are provided. It can qualitatively seen, that the states are wrong. However the angles for $ \theta_{PIP} $ and $ \theta_{DIP} $ reflect the real conditions quite good. For a performed fist, the Leap often sets those values quite low and overweights the angle for the \ac{MCP} joint. This can also be observed in \todo{figure set 5}, for example between \todo{time! 30-38 sec}. 
% -- motion velocity

\FloatBarrier	
\subsubsection{Influence of Different System Parameters}

In order to tune and improve the magnetic system for the estimation of finger poses, several methods were evaluated, based on the beforehand mentioned estimation results. As a very critical factor one can state the exact determination of the individual bone lengths, joint and sensor positions. Since those parameters are given into the equations, to represent the actual human hand, they are used by the solver, to estimate and reconstruct the measured magnetic flux densities. They are plugged in as static values and therefore have a constant nonlinear contribution to the estimated observable magnetic values. 

% Different fitting gesture
The hand dimensions are measured with a calliper. As introduced in \ref{sec:handModel}, the finger joints are assumed to have a static rotation point and the relative distance to each other is also static. Every time before measurements were recorded, the distances from the sensors to the joints are measured. Here the sensor rack is a big plus in position determination, since the locations of the sensors relative to each other are predetermined and exactly known. Furthermore, the bone lengths are measured by hand. In the end 12 bone sizes (3 for each finger), four 3D joint (one for each finger) and four 3D sensor positions (one for each sensor) have to be measured. This whole determination process is very error prone. For trying to compensate those false measured values, the initialization gesture to determine the scaling factors is introduced. By applying them to each sensor measurement one can only push the observed values into the direction of the expected results by the model. The single flexion of the \ac{MCP} joint was evaluated to serve as a repeatable motion. Also the pose of a fist was evaluated, whether it would suit the need better, since it also includes movements of the \ac{PIP} and \ac{DIP}. \ref{fig:set1mag} shows, that this is not the case. The observed magnetic fields by the sensor beneath the index finger, for a magnet located at the tip of it are displayed. The single flexion-extension of the \ac{MCP} is performed two times during \SI{3}{\second} and \SI{11}{\second}. Afterwards the finger is bended to a fist two times. From the first closed state, the motion to the straight position is performed slowly.\\
\begin{figure}[!htb]
\centering
\subfloat{\includegraphics{pictures/plots/set1mag.png}}\\
\subfloat{\includegraphics{pictures/statePics/set1mag/set1s.jpg}}
\caption[Measured magnetic flux densities for various initialization gestures]
{Sequence, showing the measured magnetic flux densities while performing two different initialization gestures. The named sensor is located beneath the index finger and the magnet is on that tip. During \SI{3}{\second} and \SI{11}{\second} only $ \theta_{\mathrm{MCP}} $ is moved. The remaining sequence shows the performance of closing and opening the fingers to a fist. While the first gesture leads to repeatable results, the second don't.}
\label{fig:set1mag}
\end{figure}
\FloatBarrier
By the development of the measured magnetic flux densities, it can be seen, that the single motion of the \ac{MCP} leads to more repeatable values, than the performance of a fist. The measurements obtained for the latter motion are highly dependent on the strength and manner of the end position. The first fist gesture shows much higher changes for the values, than the second one. Therefore the intensity of the closed hand is stronger. By looking at the simpler gesture, the two performed motion sequences look pretty much the same. Therefore the fist as initialization gesture is discarded. Like introduced in \ref{sec:evalHand} it was evaluated whether the performance along a rectangular, non-magnetic object, like a cardboard, would improve this initialization process. Therefore this gesture was performed along a cardboard before some datasets. It turned out, that the quality of the results did not increase or got even worse, by this more standardised fitting. The movement along the box and the subsequently aside putting introduces additional movements to the system. Therefore the pose of the hand changed during the calibration and the actual recording. This is critical, since the influence of the surrounding magnetic field changes due to that. Moreover, the process is performed directly over the Leap Motion controller. Since it is a vision based system, it adjusts its cameras to the surrounding light conditions. The cardboard covers the surface totally and by putting it away, the cameras have to refocus. It turned out, that if the Leap should detect a hand directly at this rescaling phase, the results are very bad or the hand even does not get detected at all. So in the end the best way for compensating the errors, induced by the hand dimensions, a gesture of single flexion-extension of \ac{MCP} has to be performed during the measurements.\\
Some of the recorded sets were also evaluated with slightly different dimensional parameters. However, no mentionable change in the results for the estimation could be observed. For estimating only a single finger by using four sensor units, the hand dimension comprises already 18 values. As presented in \ref{sec:evalHand}, tuning the measured dimensional parameters by trial and error is no option. Also the estimation of those parameters was evaluated and lead to implausible results, due to the high number of variables. 
\FloatBarrier


% Distance
As another approach, the influence of the distance from the sensors to the magnets was evaluated. For the presented datasets, the sensors were located at the back of the hand at around \SI{2}{\cm} beneath the joints. From there the corresponding bone lengths contributed, such that in full flexion, a maximum distance of around \SI{12}{\cm} was established between sensors and magnets. A few recordings were done, by placing the sensor rack at the wrist. This leads to a maximal sensor to magnet distance of around \SI{17}{\cm}. The measurable magnetic flux densities at this position, excited by the magnets at the fingertips were very low. Since the earth magnetic field cannot be eliminated reliably enough, even small motions of the hand induce errors here. The estimation results were reasonably bad.

% Earth cali
As one critical influence factor, the earth magnetic field is determined. However, the approach presented in \ref{subsec:earthEli} to overcome this showed non-satisfiable results (see \ref{subsec:resEarthEli}). To also verify this with the whole system, some experiments where executed, with different hand positions. The motions were recorded, with the hand facing different axes. This means, that the hand is also facing different orientations, compared to the Leap Controler, such as upward or downward, sideways or with the back to the camera. The results from the Leap suffered from inconsistencies, caused by occlusion of fingers or the wrist. The magnetic system, only showed reasonable results if the hand was oriented in the initial position.

% Sensor data rates
A further analysis of the sensor data acquisition rate was also evaluated. Since the system frequency for acquiring magnetic readings from all four sensors is evaluated to be \SI{20}{\Hz}, the sensor readings, which are actually sampled with \SI{50}{\Hz} are not very representable. Therefore the sampling frequency of the sensors was set at \SI{25}{\Hz}, to try to align the sensor and system frequency. However, after evaluation, the estimation results, based on the sensor data, acquired with the lower sampling rate showed the same quality as set with the higher sensor rate. Therefore, to gain the maximum possible system frequency, the sensor data rate should be put to \SI{50}{\Hz}

\FloatBarrier
\subsubsection{Concluding Observations}

Based on the presented results from the experiments, one could state that the system is dependent on a lot of variables. It can be stated, that the presented approach in combination with the utilized system cannot lead to constant and reliable results for hand pose reconstruction. In comparison to an existing camera based system, which in turn is not free of errors, the states of one finger could only be estimated with an accuracy of up to \SI{0.467}{\radian} (=\SI{26.757}{\degree}). This high difference is induced among others by not accurate determinable anatomic dimensions. The utilized hand model simplifies the natural human behaviour and constraints the range of motion in a reasonable way. However, on the real human hand, the position parameters, which are critical for the utilized model, can only be determined by hand with a caliper. Trying to reproduce the actual measurements by the hand model with error-prone position information leads to a bad model description and therefore to unsatisfying estimation results. Furthermore, the utilized models for describing the magnetic field of an artificial magnet with a certain position and orientation does not comprise the surrounding magnetic field. However, the earth field has a permanent influence on the measurements, dependent on the actual orientation, and cannot be eliminated through the presented approach. Therefore the mobility of the finger pose estimation system is highly restricted. It has to be noted, that the measurable magnetic flux density, induced by a single magnet on the fingertip, excites a field, only slightly higher, than the disturbing environment. It is evaluated, that the cylindrical bar model leads better results, than the description of the objective function with the dipole model. A fitting gesture is proposed to reduce the influence of the error-prone position and surrounding distortion factors from the measurement system. A flexion-extension about \SI{90}{\degree} of the \ac{MCP} joint is evaluated to be a reconstructible gesture for this. In the end reasonable results for the prediction where only achieved by movements, which are similar to this introduced gesture. Concerning the general solvability of the optimization problem, one can state that the system has to be overdetermined. This means that the number $ N $ of sensors, taken into account for the estimation has to be higher than the desired finger state vectors $ K $. 





%\lhead[\chaptername~\thechapter]{\rightmark}

\rhead[\leftmark]{}

\lfoot[\thepage]{}

\cfoot{}

\rfoot[]{\thepage}

\chapter{Comparison with Leap Motion system}

\begin{itemize}
\item describe angle extraction with Leap Motion
\item describe recording session
\item known issues of my system
\begin{itemize}
\item as described, sensor values are not good scalable to model...
\item though not so good results...
\item claiming that the difficulty in scaling the measurements is critical...
\end{itemize}
\end{itemize}

%\include{texfiles/discussion}

%
\lhead[\chaptername~\thechapter]{\rightmark}


\rhead[\leftmark]{}


\lfoot[\thepage]{}


\cfoot{}


\rfoot[]{\thepage}


\chapter{Conclusion and Future Work}

A magnetic sensor system for hand pose reconstruction is designed and evaluated. 
% Sensors
Since the objective was to develop a mobile and unobstrusive system, four sensors are used. They are mounted into a rack which can be worn at the back of the hand. The overall frequency for acquiring the most recent measurements of all four sensor units is evaluated to be \SI{20}{\Hz}. The calibration for hard and soft iron distortion factors of the sensors is implemented via a fit over a dataset of 1000 measurements.\\
% Magnetic models
To estimate the finger states, static magnets are mounted onto the fingertips with a ring aperture. The induced magnetic fields are used, to estimate the angles of flexion-extension and adduction-abduction for each finger joint. For describing the magnetic flux density, induced by the utilized bar magnets, two different models for the description of the magnetic field lines were established. The cylindrical bar magnet model lead to better results, since it is the more accurate description for utilized permanent magnets.\\
% Hand model
To represent the human hand, a kinematic chain with 12 \ac{DOF} is chosen, in order to represent the pose of the index, middle, ring and pinky finger. The movement of the thumb is left out. The introduced kinematic model is constraint to natural ranges of movement and can in the end describe the pose of one finger by three angular values. Those are the angles of flexion-extension and adduction-abduction of the \ac{MCP} and the flexion-extension angle for the \ac{PIP} joint. Via an introduced intra finger constraint, the angle of the \ac{DIP} joint is derived via the \ac{PIP} angle. In the end, by knowing the dimensions of the bone lengths and the positions of the joints and sensors, the distance and orientation of the fingertip (and therefore of the magnet) can be calculated relative to the sensor unit. Those distance and orientation vectors $ \vec{r} $ and $ \vec{h} $ can be plugged into the magnetic models, to estimate the cumulative measurement at the sensor units.\\
% Estimation stage
An optimization problem is formulated, which reduces the error between model and sensor measurement, by minimizing for the finger state angles. In this way the actual states for each finger can be estimated. By using a certain number of sensors to predict the position of the four fingers, it is evaluated, that the system has to be overdetermined. Therefore, the results for estimating all four magnets with the deployed four sensor units does not lead to reasonable results. Furthermore, it can be stated, that the problem has to be constrained and that the slight lateral movements of the \ac{MCP} joint can be reconstructed.\\
% Results and Leap Motion
For an evaluation of the system performance, the estimated finger states were compared to the vision based Leap Motion system. Reasonable results could only be reached for the prediction of a single finger state vector. The estimated states show a non static difference to the Leap system for most of the measurement sets. 
%Only a qualitative measure for the direction of the angular movement of a single finger state vector can be estimated. 
For the evaluated recorded sets, the overall smallest difference to the Leap Motion was observed to be $ 0.467 \si{radian} \pm 0.027 $ (=$ 26.757 \si{\degree} \pm 1.547 $), for predicting the states of the index finger with the measurements of two magnetic sensor units. This high deviation is due to the following drawbacks, the magnetic system suffers from:
\begin{itemize}
\item The parameteres of the underlying hand model can not be measured accurate enough. This causes nonlinear errors for the predictable magnetic flux densities. This is tried to be compensated by introducing a fitting gesture.
\item The surrounding magnetic field can not be eliminated dynamically. This is critical, since the observable magnetic field, induced by the static magnets is in the range of the earth magnetic field.
\item The hand, trying to be tracked has to be held calm during the measurements.
\item The achievable acquisition frequency for four sensor units is only capable of slow finger motions
\end{itemize} 

% Future work
In the end it can be stated, that the presented approach can be used to track finger poses reliably and dynamically under the named conditions. The detection of several movements and finger gestures can be distinguished and identified. The exact determination of the actual human hand parameters is critical for achieving good estimation results. Therefore, for future work on this concept, the exact determination of the anatomic hand dimensions could be further investigated. Additionally, the dynamic and reliable cancellation of the surrounding magnetic field would be important to remove the influence of the external magnets from the measurements. It is shown, that the determinateness of the system plays an important role for the estimation accuracy. Hence, designing a still wearable system by deploying more sensors could also be evaluated in the future. As another point, the overall acquisition frequency could be improved, to also being capable of faster finger movements. Moreover, it can be evaluated whether an approach, based on a learned set of motions would lead to better results for reconstructing finger postures.

 




%the other chapters go here...

\cleardoublepage{}


%\lhead[]{Acknowledgment}


%\rhead[Acknowledgment]{}

%
\chapter*{Danksagung}

\addcontentsline{toc}{chapter}{Danksagung} 

Danksagungstext

an anfang des papers?

und irgendwo sollte noch sowas wie meine vorstellung rein mit passbild undso?


\appendix
%
\lhead[\chaptername~\thechapter]{\rightmark}


\rhead[\leftmark]{}


\lfoot[\thepage]{}


\cfoot{}


\rfoot[]{\thepage}


\chapter{Algorithm Energy Efficiency}
\label{cha:algorithmEnergyEfficiency}




\cleardoublepage{}


\lhead[]{\rightmark}


\rhead[\leftmark]{}

\bibliographystyle{alphadin}
%\bibliography{biblio/Plasma}
\bibliography{biblio/bibFiles}


\cleardoublepage{}


\lhead[]{Nomenklatur}


\rhead[Nomenklatur]{}

\printnomenclature[2.5cm]{}
\end{document}

%\selectlanguage{english}%

\chapter*{Abstract}

\addcontentsline{toc}{chapter}{Abstract} 
% Intro/Motivation
The tracking and reconstruction of human hand motion can be used for modern human-machine interaction or as support for medical rehabilitation. State of the art motion tracking systems adopted for this purpose introduce often bulky constructions.  
%to track the fine and complex movements of the hand. However those systems are often not only constructed for the estimation of finger postures and introduce therefore often the need bulky constructions. 
The underlying work presents an approach for the reconstruction of human hand motion by measuring the magnetic field, induced by permanent magnets on the fingertips. When the fingers move, the cumulative sum of the magnetic flux density, excited by the magnets, is recorded with a sensor array, consisting of four measurement units. This data is used to estimate the finger poses, based on a kinematic hand model. 

% General approach
The magnetic sensors are placed inside a bracket to be worn on the back of the hand. The recorded data is transferred via Bluetooth Low Energy to a host PC, where the actual state estimation is executed. The finger postures are estimated, by solving an optimization problem. The underlying human hand model is represented as a kinematic chain and uses the joint angles to describe finger postures. The thumb is neglected, for simplicity reasons. 

% Results
The performance of the developed system is highly dependent on several parameters. The most critical factor is the exact determination of the anatomic hand dimensions and the positions of the sensors. Those parameters have a nonlinear contribution to the model equations and build the base for the state estimation. However, they can only be measured by hand with a calliper, which is very error-prone. Furthermore, the dynamic elimination of the surrounding earth magnetic field is not possible. Therefore, the system is only usable in static hand positions. To overcome those nonlinear and highly critical distortion factors, a scaling of the sensor values to the model parameters is adopted. The scaling factors are determined by a predefined fitting gesture.

Several datasets were recorded with the magnetic system and the estimated states were compared to data provided by a commercially available vision based approach. The individual motions of multiple fingers cannot be tracked by the developed system. However, equipping a single finger with a magnet and estimating its pose can be done almost reliable and leads to reasonable results. The smallest predicted difference for this case to the vision system is $ 0.467 \si{\radian} \pm 0.027 $ (=$ 26.757 \si{\degree} \pm 1.547 $).


%\todo{
%\begin{itemize}
%\item check for consistent spelling of 3D (3d, three dimensional...)
%\item check for consistent order of flexion-extension
%\item check for consistent order of adduction-abduction
%\item writing numbers out or not?
%\item do not cut adduction-abduction / flexion-extension $ \rightarrow $ define it as an accronym!
%\item centre vs center
%\item adjust the line breaks
%\item align the formulas in a pretty shape!
%\end{itemize}
%}


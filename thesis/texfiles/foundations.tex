
\lhead[\chaptername~\thechapter]{\rightmark}

\rhead[\leftmark]{}

\lfoot[\thepage]{}

\cfoot{}

\rfoot[]{\thepage}


\chapter{Foundations}
\label{cha:foundations}

\section{Anatomy of the human hand} \label{sec:anatomy}
The anatomy and the motion of the human hand is very complex. Describing the whole interaction of tendons, muscles and bones would go beyond the scope of this thesis. This section will focus on the kinematic structure of the hand, concentrating on the bones and knuckles which are the most relevant parts for describing the range of possible motions. The goal is to understand and derive a model to reliably reconstruct the human hand motion.\\
Bullock et al. \cite{bullock2012assessing} and Lin et al. \cite{lin2000modeling} serve as a good and application oriented introduction to the anatomic structure. A CT scanned image of the hand with explanations is provided by \ref{fig:skeletHand}. The metacarpals are enclosed by muscles and tendons and form the palm. This section is in principle very static and only small ranges of flexion/extension is possible. During strong motions, like grasping, the metacarpals can move slightly, such that a maximum flexion-extension of \SI{20}{\degree} around the \ac{CMC} joints is possible. However this movement is not very natural and common, the skeletal palm is assumed to be fixed. From that it directly follows that also the positions of the \ac{MCP} joints lie in the same plane and their positions are also static. A static joint position means, that the center of rotation stays the same. Anatomically seen this is again a simplification, since the joint axes are only fixed within \SI{1.5}{mm} during a full flexion/extension of the \ac{MCP}. Each \ac{MCP} has two degrees of freedom, since abduction/adduction is possible within a range of around \SI{30}{\degree} and flexion/extension of approximately \SI{90}{\degree} is possible. Strictly speaking one also has to consider axial rotation of the MCP, but since this small amount of movement can only be performed by applying external forces it is neglected here. Next to the MCP comes the proximal phalanges with the \ac{PIP} joint, the intermediate phalanges with the \ac{DIP} joint and the distal phalanges. The \ac{PIP} and \ac{DIP} joints show one degree of freedom each, since only flexion/extension is possible here. The maximum range of movement for the \ac{PIP} will be assumed as \SI{90}{\degree} and for the \ac{DIP} as \SI{110}{\degree}. One could assume that the three bones (proximal, intermediate and distal phalanges) representing a finger move in one plane, relative to the first joint (\ac{MCP}). In other words, that the flexion/extension axes of the three joints (\ac{MCP},\ac{PIP} and \ac{DIP}) are all parallel to each other. From an anatomical view, this is not totally right, but since this shift is only small it is neglected by most groups. So in total one finger comes with 4 \ac{DOF}. Lin et al. propose to introduce an intra-finger constraint, concerning the movement of the \ac{PIP} and \ac{DIP} joints. In order to bend the \ac{DIP} joint, the \ac{PIP} must also be bended. The relation between those two joints is commonly approximated as $ \theta_{DIP} = \frac{2}{3} \theta_{PIP} $. This reduces the overall \ac{DOF} for one finger to three. The thumb was not described till now. This special finger of the human hand, being the most important for powerful and helpful interactions induces even more complexity. It's flexibility allows reaching and touching the other fingertips. Like the others it has three joints, however comprising more \ac{DOF}. The \ac{MCP} has two \ac{DOF} as well as the \acs{TM}. In total the thumb has therefore five \ac{DOF}. To obtain an easier model, the movement of the thumb is neglected.


\begin{figure}[h]
\centering
	\subfloat[Bone structure of the right hand.]
	{\includegraphics[width=0.5\textwidth]{pictures/boneHand.png}\label{fig:boneHand}}
	\hfill
	\subfloat[Joints of the right hand.]
	{\includegraphics[width=0.5\textwidth]{pictures/jointsHand.png}\label{fig:jointsHand}}	
\caption{Skeletal representation for the bones and joints of the right human hand. Dorsal view \cite{bullock2012assessing}}
\label{fig:skeletHand}
\end{figure}

The following list summarizes the proposed assumptions and simplifications, applied to the model:
\begin{itemize}
\item Only movements without applying external forces are modeled.
\item Joints are modeled as a combination of ideal revolute joints
\item Bones serve as perfect rigid bodies.
\item So the fingers can be modeled as a kinematic chain.
\item The movement of the thumb is neglected for simplicity reasons.
\item The flexion/extension of metacarpals is neglected, this means the palm is assumed as a rigid plane which comprises the fixed positions of the \acp{MCP}.
\item The flexion/extension of one finger is planar
\item \ac{MCP} has 2 \ac{DOF} with flexion/extension angle: $ \ang{0} \leq \theta_{MCP} \leq \ang{90} $ and adduction/abduction angle: $ \ang{0} \leq \psi_{MCP} \leq \ang{90} $
\item \ac{DIP} and \ac{PIP} have 1 DOF with flexion/extension angle: $ \ang{0} \leq \theta_{PIP} \leq \ang{90} $ and $ \ang{0} \leq \theta_{DIP} \leq \ang{110} $
\item Dynamic constraint between $ \theta_{DIP} = \frac{2}{3} \theta_{PIP} $
\item So each finger has 3 \ac{DOF} in total.
\item A local frame is used for motion reconstruction, so one can represent a pose by describing the joint angles.
\end{itemize}

The introduced assumptions try to restrict the human hand in a useful meaning, without loosing generality. There can even be introduced inter-finger constraints, due to naturalness of hand motions. For example that one can not totally bend the index finger without also bending the others. However each hand is different, implying the range and ability of finger poses, so this behavior is very difficult to generalize and to model.\\
The application specific restrictions applied to the model are further depicted in \todo{add section reference!}




\section{Magnetic}
\label{sec:magneticFound}

\begin{itemize}
\item paramagnetic materials have a permanent magnetic moment $ M $
\item this means that the magnetic moments within the bar magnet are static and aligned in the same direction
\item good introduction \href{http://www.ece.umd.edu/class/enee380-1.F2005/lectures/lecture24.htm}{lecture}
\item picture of magnetic dipole field lines
\item \cite{mladenovic2009magnetic} in fact, you can find sth here! But only numerics... no hand calculations...
\item only magnetic dipoles exist...
\end{itemize}

\cite{derby2010cylindrical} describe sth hot!\\
The magnetic field of a cylindrical bar magnet is equivalent to a tightly wound solenoid
\begin{itemize}
\item length of $ 2b $, radius $ a $
\item origin at the center of the solenoid!
\item in presence of a cylindrical coordinate system \ref{fig:cylCoSys}! This means:
\item the height $ z $ along the length of magnet
\item radial distance $ \rho $ being the euclidean distance from the z axis
\item and azimuth $ \varphi $ being the angle between the reference direction and the chosen plane
\end{itemize} 
\begin{figure}[h]
\centering
\includegraphics[width=0.5\textwidth]{pictures/cylCoSys2.png}
\label{fig:cylCoSys}
\caption{Representation of cylindrical coordinate system}
\end{figure}

One has to solve a complete elliptical integral(cel)
\begin{equation}
C(k_{c},p,c,s) = \int_{o}^{\frac{\pi}{2}} \frac{c \, \cos^{2}\varphi + s \, \sin^{2}\varphi}
{(\cos^{2}\varphi + p \, \sin^{2}\varphi)\sqrt{\cos^{2}\varphi + k_{c}^{2} \, \sin^{2}\varphi}} \; d\varphi
\end{equation}

\begin{itemize}
\item This can be done by a known algorithm...
\item Deriving the magnetic field for the ideal solenoid with the help of Biot-Savart law
\item Just slice it in $ dz' $ pieces and see each peace as a current loop 
\item Calculate the field for this loop with Biot-Savart and then add up the fields for the remaining $ dz' $ current loops
\end{itemize}


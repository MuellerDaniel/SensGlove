
\lhead[\chaptername~\thechapter]{\rightmark}

\rhead[\leftmark]{}

\lfoot[\thepage]{}

\cfoot{}

\rfoot[]{\thepage}


\chapter{Foundations} \label{cha:foundations}

The following two sections provide further information about the anatomy of the human hand and the description of permanent magnets. A basic insight to those two fields will be given, to better understand the upcoming parts about the developed hand model and the formulation of the finger state estimation.

\section{Anatomy of the Human Hand} \label{sec:anatomy}
The anatomy and the motion of the human hand is very complex. Describing the whole interaction of tendons, muscles and bones would go beyond the scope of this thesis. This section will focus on the kinematic structure and relevant components of the hand for describing motion. The goal is to understand and derive a model to reliably reconstruct the human hand motion.\\
Bullock et al. \cite{bullock2012assessing} and Lin et al. \cite{lin2000modeling} serve as a good and application oriented introduction to the anatomic structure. A \ac{CT} scanned image of the hand with explanations is provided by \ref{fig:skeletHand}. The metacarpals are enclosed by muscles and tendons and form the palm. Those parts are in principle static and only slight movements of flexion-extension are possible. During strong motions, like grasping, the metacarpals can move slightly, such that a maximum flexion-extension of \SI{20}{\degree} around the \ac{CMC} joints is possible. Since this movement is not very natural and common, the skeletal palm is often seen as static. For this reason it directly follows that also the positions of the \ac{MCP} joints lie in the same plane and their positions are also static. This means that the center of rotation for the movement of a finger stays the same. Anatomically seen this is a simplification, since the joint axes are only fixed within \SI{1.5}{mm} during a full flexion-extension of the \ac{MCP}. 
Each \ac{MCP} shows two degrees of freedom. One is represented by the lateral movement of abduction-adduction, which is possible within a range of around \SI{-15}{\degree} to \SI{+15}{\degree}. The second is the flexion-extension of approximately \SI{90}{\degree}. Strictly speaking one also has to consider axial rotation of the MCP, but since this small amount of movement can only be performed by applying external forces it is neglected here. Next to the \ac{MCP} comes the proximal phalanges with the \ac{PIP} joint, the intermediate phalanges with the \ac{DIP} joint and the distal phalanges. The \ac{PIP} and \ac{DIP} joints show one degree of freedom each, since only flexion-extension is possible. The maximum range of movement for the \ac{PIP} will be assumed as \SI{110}{\degree} and for the \ac{DIP} as \SI{90}{\degree}. One could assume that the three bones (proximal, intermediate and distal phalanges) representing a finger move in one plane, relative to the first joint (\ac{MCP}). In other words, that the flexion-extension axes of the three joints (\ac{MCP},\ac{PIP} and \ac{DIP}) are all parallel to each other. From an anatomical view, this is not totally right, but since this shift is only small it can be neglected for a representable model. So in total one finger comes with 4 \acp{DOF}. 

When it comes to modelling and simplifying the anatomy of the human hand, one can make some general assumptions. As already stated above, the small movement of the \ac{CMC} is negligible, and so it is assumed that all \ac{MCP} joints lie in the same plane. Lin et al. propose to introduce an intra-finger constraint, concerning the movement of the \ac{PIP} and \ac{DIP} joints. In order to bend the \ac{DIP} joint, the \ac{PIP} must also be bended. The relation between those two joints is commonly approximated as $ \theta_{DIP} = \frac{2}{3} \theta_{PIP} $. This reduces the overall \ac{DOF} for one finger from four to three. The thumb was not described till now. This special finger of the human hand, being the most important for powerful and helpful interactions induces even more complexity. It's flexibility allows reaching and touching the other fingertips. Like the ones before, it has three joints, however comprising more \ac{DOF}. The \ac{MCP} has two \ac{DOF} as well as the \ac{TM}. In total the thumb has therefore five \ac{DOF}. To obtain an easier model, the movement of the thumb is neglected.\\
For analysing the speed and naturalness of hand motions, Ingram et al. \cite{ingram2008statistics} examined six healthy male subjects in their study. The probands were equipped with resistive CyberGloves and asked to wear them for a certain time at a day, in order to become an insight in the natural range of movement. Regarding the angular velocity, they observed a mean value of \SI[per-mode=symbol]{10}{\degree \per \second}. The overall velocity of all joints was always less than \SI[per-mode=symbol]{100}{\degree \per \second}.


\begin{figure}[h]
\centering
	\subfloat[Bone structure of the right hand.]
	{\includegraphics[width=0.5\textwidth]{pictures/boneHand.png}\label{fig:boneHand}}
	\hfill
	\subfloat[Joints of the right hand.]
	{\includegraphics[width=0.5\textwidth]{pictures/jointsHand.png}\label{fig:jointsHand}}
\caption[Bone and joint definitions of the human hand.]
{Skeletal representation for the bones and joints of the right human hand. The view is from the dorsal side.\cite{bullock2012assessing}}
\label{fig:skeletHand}
\end{figure}

The following points summarize the proposed assumptions and simplifications, applied to the model:
\begin{itemize}
\item Only movements without applying external forces are modeled.
\item Knuckles are modeled as a combination of ideal revolute joints
\item Bones serve as perfect rigid bodies.
%\item So the fingers can be modeled as a kinematic chain.
\item The movement of the thumb is neglected for simplicity reasons.
\item The flexion-extension of metacarpals is neglected, this means the palm is assumed as a rigid plane which comprises the fixed positions of the \acp{MCP}.
\item The flexion-extension of one finger is planar
\item \ac{MCP} has 2 \ac{DOF} with flexion-extension angle: $ \ang{0} \leq \theta_{MCP} \leq \ang{90} $ and adduction/abduction angle: $ \ang{-15} \leq \phi_{MCP} \leq \ang{+15} $
\item \ac{DIP} and \ac{PIP} have 1 DOF with the following ranges for the flexion-extension angle: $ \ang{0} \leq \theta_{DIP} \leq \ang{90} $ and $ \ang{0} \leq \theta_{PIP} \leq \ang{110} $
\item Dynamic constraint between $ \theta_{DIP} = \frac{2}{3} \theta_{PIP} $
\item So each finger has 3 \ac{DOF} in total.
\item A local frame is used for motion reconstruction, so one can represent a pose by describing the joint angles.
\item A mean angular velocity of \SI[per-mode=symbol]{10}{\degree \per \second} is assumed.
\end{itemize}

The introduced assumptions try to restrict the human hand in a useful meaning, without loosing too much generality. Utilizing those statements, the fingers of the human hand can be modelled as kinematic chain, which is fully described by all joint angles. Inter-finger constraints, due to the naturalness of hand motions could also be introduced. For example that one can not totally bend the index finger without also bending the others. However each hand is different, implying the range and ability of finger poses, so this behavior is very difficult to generalize and to model. The application specific restrictions, applied to the model used for this work are further depicted in \ref{sec:handModel}.

\FloatBarrier

\section{Magnetic} \label{sec:magneticFound}
Permanent magnets are widely used nowadays. Their constant magnetic characteristics are used for example in electric engines, \ac{CT} or by interacting with other ferro-, para- or diamagnetic materials, in the simplest case for holding things at a dedicated place. The principle behind it is the excitation of a magnetic field. It is commonly known that only magnetic dipoles exist, meaning that a magnetic north pole never comes without its respective south pole. The shape of the magnetic field lines is commonly known and part of undergraduate physics. However when it comes to the physical description of the magnetic field lines of a permanent magnet, the subject gets more complex. In the following a mathematical approximation for describing the magnetic field, excited by a permanent cylindrical neodymium bar magnet of length $ 2b $ and radius $ a $ will be presented. The strength of the magnetic field, or in other words the magnetic flux density is denoted by $ \mathrm{B} $. Its corresponding SI-Unit is called Tesla ($ \SI{1}{\tesla} = 1 \si[per-mode=fraction]{\kilogram \per \ampere \per \second \squared} = 1 \si[per-mode=fraction]{\newton \per \ampere \per \meter}$). However it exists a second unit to describe this strength, called Gauss, which is used by the \ac{CGS}-system($ 1 \mathrm{Gs} = 10^{-4}\SI{}{\tesla} $). \\
At first, one has to mention that a current carrying loop also excites a magnetic field. This is described among others by the Biot-Savart law. The law is valid for the approximation of static magnetic fields. The representation as magnetic dipole by a single current loop or a pair of contrary poles, and the description of its field lines is the base for the ongoing derivation and a common method \cite{derby2010cylindrical}. Accordingly the literature often treats the magnetic field as excited by electric current. The electromagnetic equivalent to a cylindrical bar magnet would be a tightly wound solenoid with a number of turns per unit length $ n $, carrying a current $ I $.\\
An approximation for describing the magnetic field of a permanent magnet can be done by describing it as a magnetic dipole. This approximation is used by Ma et al. \cite{ma2010magnetic}. As briefly mentioned beforehand there are only magnetic dipoles, represented by a pair consisting of a (positive) north pole and a (negative) south pole. The field lines are oriented along the direction of the magnetic dipole moment $ \vec{m} $ and go from south to north (see \ref{fig:magDipole}). The dipole moment can be interpreted as a measure for the orientation and the strength of the bar magnet. If one shrinks the distance of two contemporary charges to a point and keeps the orientation of the magnetic moment constant, the field strength $ \mathrm{B} $ can be described by the following formula:
\begin{equation} \label{eq:dipole}
\mathrm{B}(\vec{r},\vec{m}) = 
\frac{\mu_{0}}{4 \pi |\vec{r}|^{2}} \, \left (\frac{3 \vec{r} (\vec{m} \cdot \vec{r}) - \vec{m}|\vec{r}|^{2} }{|\vec{r}|^3} \right)						
\end{equation}
with 
\begin{equation*} \label{eq:magMoment}
\vec{m} = \vec{h} \cdot \frac{\mathrm{B}_r \mathrm{V}}{\mu_{0}}
\end{equation*}
$ \vec{r} $ represents the distance from the magnetic source, which is located at the origin of a Cartesian coordinate frame. For a bar magnet, this means that its center is located at the origin. The constant $ \mu_{0} $ is the vacuum magnetic susceptibility \cite{camacho2013alternative} and has a value of $ 4 \pi \cdot 10^{-7}\si[per-mode=fraction]{\newton \per \ampere \squared}$. The factor $ \mathrm{B}_r $ represents the Remanence field of the magnetic material. For the used neodymium magnet this value is in between 1.26-\SI{1.29}{\tesla} (for the ongoing calculations it is assumed to be \SI{1.29}{\tesla}). $ \vec{h} $ is the normalised orientation in space of the magnet. One thing that can be seen by this formula is the relationship of the magnetic field at a distance $ \vec{r} $. The field decreases by a cubic magnitude of the distance. So roughly spoken the field strength $ \mathrm{B} \sim \frac{1}{|r|^ {3}}$. Another point is, that due to the dot product between $ \vec{m} $ and $ \vec{r} $ the model gets nonlinear.\\
\begin{figure}
\centering
\includegraphics[width=0.3\textwidth]{pictures/magDipole.jpg}
\caption[Magnetic field lines, excited by a dipole.]
{The shape of the magnetic field lines is visualized, excited by a magnetic dipole. The magnetic moment is denoted as $ \vec{m} $ and represents the orientation of the magnet.}
%\todo{Reference!}\href{http://cdn1.askiitians.com/Images/201529-161356533-148-download.jpg}{link}
\label{fig:magDipole}
\end{figure}
Since this is an approximation for the behaviour of the magnetic field of a bar magnet, Camacho et al. \cite{camacho2013alternative} follow a more detailed way, by taking the shape of the cylindrical permanent magnet into account. They also state that for a homogeneous, constantly magnetized body the distribution and orientation of magnetic dipoles inside a volume $ dV $ of this body is constant and has the magnetic dipole moment $ d\vec{m} = \vec{M}dV $. Where $ \vec{M} $ is the volume magnetization of the magnet. Further on it holds for a permanent magnet that the magnetization does not change for any external fields, they are supposed to be hard. Using this for the cylindrical bar magnet, which is assumed to be magnetized along its symmetry axis, the following can be applied: It is assumed that the bar magnet is centered at the origin of a Cartesian coordinated system and aligned, such that the symmetry axis is oriented in $ z $ direction, like shown in \ref{fig:cylMag}. The magnet is sliced into infinitesimal small discs with height $ dz $, such that one element $ dV $ is defined by the radius $ a $ and the height $ dz $. 
\begin{figure}
\centering
\includegraphics[width=0.25\textwidth]{pictures/cylMagCoSys.png}
\caption[Cylindrical bar magnet with infinitesimal slice $ dz $]
{A cylindrical magnet, aligned along its magnetization axis in $ z $ direction. A infinitesimal piece $ dz $ is sketched. The vector R to a dedicated measurement point is visualized \cite{derby2010cylindrical}.}
\label{fig:cylMag}
\end{figure}
The magnetic flux along the symmetry axis of such an infinitesimal element $ dV $ is given by \ref{eq:dipole}. In order to calculate $ \mathrm{B}_{z} $ for the whole magnet, one has to integrate these contributions over the entire volume. In the end one gets the following formula for the magnetic field of a bar magnet along its magnetization axis(for a detailed derivation, please have a look at \cite{camacho2013alternative}, \cite{derby2010cylindrical}):
\begin{equation} \label{eq:b_z}
\mathrm{B}_{z}(z) = \frac{\mathrm{B}_r}{2} \left ( \frac{z + b}{\sqrt{(z + b)^2 + a^2}} - \frac{z - b}{\sqrt{(z - b)^2 + a^2}} \right)
\end{equation}
Remind, that the solenoid has a length of $ 2b $ and radius $ r $.\\
For deriving the mentioned equation there exist several methods. As shortly depicted, an integral over the whole surface has to be derived. Those integrals with a cylindrical symmetry are usually quite complex. However for the symmetrical case along the $ z $ axis, one can use some mathematical properties to ease and obtain the \ref{eq:b_z} in the end. However if one wants not only to describe the magnetic field along the magnetization axis, those properties and therefore the presented equation does not hold any more. Derby et al. \cite{derby2010cylindrical} propose to overcome this problem, by solving a \ac{CEL}. The generalized form of this is the following:
\begin{equation}\label{eq:cel}
C(k_{c},p,c,s) = \int_{0}^{\frac{\pi}{2}} \frac{c \, \cos^{2}\varphi + s \, \sin^{2}\varphi}
{(\cos^{2}\varphi + p \, \sin^{2}\varphi)\sqrt{\cos^{2}\varphi + k_{c}^{2} \, \sin^{2}\varphi}} \; d\varphi
\end{equation}
A cylindrical coordinate system is introduced, which makes sense, since one tries to describe a cylindrical shape. The corresponding structure and naming of axes is shown in \ref{fig:cylCoSys}. With the help of this cylindrical representation, the general magnetic field components can be expressed by
\begin{equation} \label{eq:cylB_rho}
\mathrm{B}_{\rho} = \mathrm{B}_{o}[\alpha_{+} C(k_{+},1,1,-1) \, - \, \alpha{-} C(k_{-},1,1,-1)]
\end{equation}
for the radial magnetic component along $ \rho $. And
\begin{equation} \label{eq:cylB_z}
\mathrm{B}_{z} = \frac{\mathrm{B}_{o}a}{a+\rho}[\beta_{+} C(k_{+},\gamma^2,1,\gamma) \, - \, \beta_{-} C(k_{-},\gamma^2,1,\gamma)]
\end{equation}
for the axial magnetic component along the $ z $ axis. Along with the following introduced variables, which are dependent on the cylindrical position $ (\rho, \varphi, z) $ of the dedicated point in space.
\begin{equation*}
\mathrm{B}_{o} = \frac{\mu_{0}}{\pi}nI = \frac{\mathrm{B}_{r}}{\pi}
\end{equation*}

\begin{equation*}
z_{\pm} = z_{\pm} b
\end{equation*}

\begin{equation*}
\alpha_{\pm} = \frac{a}{\sqrt{z_{\pm}^2+(\rho+a)^2}}
\end{equation*}

\begin{equation*}
\beta_{\pm} = \frac{z_{\pm}}{\sqrt{z_{\pm}^2+(\rho+a)^2}}
\end{equation*}

\begin{equation*}
\gamma = \frac{a-\rho}{a+\rho}
\end{equation*}

\begin{equation*}
k_{\pm} = \sqrt{\frac{z_{\pm}^2 + (a-\rho)^2}{z_{\pm}^2 + (a+\rho)^2}}
\end{equation*}
The azimuthal component $ \varphi $ can be neglected, since the magnetic field does not change, by moving around a circle which is aligned perpendicular to the $ z $ axis. In order to derive the equation \ref{eq:b_z} from \ref{eq:cylB_rho} and \ref{eq:cylB_z}, one just has to set the radial component $ \rho $ to 0 and $ z > b $\\
The presented \ac{CEL} are not easy to solve, therefore numerical methods are used to find a solution. However trying to find an exact solution with software like \textit{Mathematica} or \textit{Maple} failed \cite{camacho2013alternative}. Since \acp{CEL} are not only restricted to cylindrical bar magnets, there is some research going on in this field. For the finally used method and further information about \ac{CEL}, please look at \cite{derby2010cylindrical}. Another method consists in using numerics, like \ac{FEM}. This can lead to satisfying results \cite{mladenovic2009magnetic}.
\begin{figure}[b]
\centering
\includegraphics[width=0.4\textwidth]{pictures/cylCoSys2.png}
\caption[Cylindrical coordinate frame]{Representation of the cylindrical coordinate system frame. The height is denoted by the $ z $ axis, the radial component $ \rho $ is the euclidean distance from the $ z $ axis, the azimuth $ \varphi $ describes the angle between the reference direction and the chosen plane. \cite{derby2010cylindrical}}
\label{fig:cylCoSys}
\end{figure}
The two presented models represent the basis for describing the magnetic flux at a certain position in space, relative to a permanent magnet. They both assume the absence of any additional, static magnetic fields, for example the one produced by the earth. Both are used in the further evaluation. The equation, describing the magnetic field as an ideal dipole is refereed as the \textbf{dipole model} and represents a full nonlinear mathematical equation. The model approximating the cylindrical shape of the bar magnet is refereed to as the \textbf{cylindrical model}. Since a numerical algorithmic evaluation method is used for this model, mathematical operations like differentiation or integration is not applicable. A comparison between the two methods with real measurements and its effects on the finger pose reconstruction is done in \ref{cha:results}.

\FloatBarrier
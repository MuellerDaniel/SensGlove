% !TeX spellcheck = de_DE

%\begin{otherlanguage}{german}

\chapter*{Zusammenfassung}

Die Beobachtung und Rekonstruktion von Bewegungen der menschlichen Hand kann für moderne Mensch-Computer Interaktionen oder für die Unterstützung medizinischer Therapien von Schlaganfall oder behinderten Menschen eingesetzt werden. Hochmoderne Systeme zur Bewegungsverfolgung, die für diesen Zweck angepasst sind, bringen oft sperrige Geräte mit sich. Die vorliegende Arbeit präsentiert einen Ansatz zur Rekonstruktion von Handbewegungen, indem das Magnetfeld gemessen wird, welches von Permanentmagneten auf den Fingerspitzen erzeugt wird. Wenn sich die Finger bewegen ändert sich die von den Magneten hervorgerufene magnetische Flussdichte, was von dem aus vier Sensoren bestehenden Array gemessen werden kann. Diese Daten werden benutzt, um die Fingerpositionen auf der Basis eines kinematischen Handmodels zu schätzen.

Die Magnetsensor sind in einem Gestell platziert, welches am Handrücken getragen werden kann. Die Sensordaten werden über Bluetooth Low Energy zu einem PC gesendet, auf dem die eigentliche Zustandsschätzung durchgeführt wird. Die Fingerpositionen werden durch die Lösung eines Optimisierungsproblems vorausgesagt. Das zugrundeliegende Model zur Beschreibung der menschlichen Hand beschreibt die Fingerposition durch die Angabe der Gelenkwinkel. 

%\end{otherlanguage}


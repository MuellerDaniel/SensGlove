% !TeX spellcheck = de_DE

%\begin{otherlanguage}{german}

\chapter*{Zusammenfassung}

Die Beobachtung und Rekonstruktion von Bewegungen der menschlichen Hand kann f\"ur moderne Mensch-Computer-Interaktionen oder f\"ur die Unterst\"utzung medizinischer Therapien eingesetzt werden. Moderne Systeme zur Bewegungsverfolgung, die f\"ur diesen Zweck angepasst sind, bringen oft sperrige Ger\"ate mit sich. Die vorliegende Arbeit pr\"asentiert einen Ansatz zur Rekonstruktion von Handbewegungen, indem das Magnetfeld gemessen wird, welches von Permanentmagneten, die sich auf den Fingerspitzen befinden, erzeugt wird. Wenn sich die Finger bewegen, \"andert sich die von den Magneten hervorgerufene magnetische Flussdichte, was von dem aus vier Sensoren bestehenden Sensorarray gemessen werden kann. Diese Daten werden benutzt, um die Fingerpositionen auf der Basis eines kinematischen Handmodels zu sch\"atzen.

Die Magnetsensoren werden in einem Gestell platziert, welches am Handr\"ucken getragen werden kann. Die Sensordaten werden \"uber Bluetooth Low Energy zu einem PC gesendet, auf dem die eigentliche Zustandssch\"atzung durchgef\"uhrt wird. Die Fingerpositionen werden durch die L\"osung eines Optimierungsproblems vorausgesagt. Das zugrunde liegende Modell zur Darstellung der menschlichen Hand beschreibt die Fingerposition durch die Angabe der Gelenkwinkel. Der Daumen wird zur Vereinfachung nicht beachtet.

Die Qualit\"at des entwickelten Systems h\"angt von mehreren Faktoren ab. Der kritischste ist hierbei die exakte Bestimmung der anatomischen Gegebenheiten und der Sensorpositionen. Diese Parameter haben einen nichtlinearen Einfluss auf die Modellgleichungen, welche die Grundlage f\"ur die Zustandssch\"atzung bilden. Diese Werte k\"onnen jedoch nur per Hand mit einem Messschieber gemessen werden, woraus Messfehler resultieren k\"onnen. Weiterhin ist die dynamische Eliminierung des Erdmagnetfeldes nicht m\"oglich. Deshalb ist das System nur benutzbar, wenn auch die Hand ruhig gehalten wird. Um diese nichtlinearen und h\"ochst kritischen Einflussfaktoren zu umgehen, werden die Sensorwerte an die Modellgleichungen angepasst. Diese Skalierungsfaktoren werden \"uber eine vorbestimmte Initialisierungsgeste bestimmt.

Mehrere Datensets wurden mit dem magnetischen System aufgenommen. Die Resultate der Zustandssch\"atzung wurden mit einem bestehenden kamerabasierten System verglichen. Einzelne Bewegungen von mehreren Fingern k\"onnen mit dem entwickelten System nicht nachverfolgt werden. Jedoch ist die Sch\"atzung eines einzelnen Fingers relativ zuverl\"assig und f\"uhrt zu nachvollziehbaren Ergebnissen. F\"ur diesen Fall betr\"agt der kleinste beobachtete Unterschied zum Kamerasystem $ 0.467 \si{\radian} \pm 0.027 $ (=$ 26.757 \si{\degree} \pm 1.547 $).


%\end{otherlanguage}


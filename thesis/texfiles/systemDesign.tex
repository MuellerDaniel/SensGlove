\lhead[\chaptername~\thechapter]{\rightmark}

\rhead[\leftmark]{}

\lfoot[\thepage]{}

\cfoot{}

\rfoot[]{\thepage}

\chapter{System Design and Implementation} \label{cha:sysDesign}

\section{System design} \label{cha:design}

\begin{itemize}
\item general description of the system
\item four sensors, magnets on fingertips, sensors on rack at back of hand, ...
\item wooden hand for easy/repeatable measurements (it is not perfect! since the index and pinky finger do not move in x-z plane!)
\item pictures!
\end{itemize}


\section{Sensor design/Data acquisition...} \label{cha:sensors}

\begin{itemize}
\item describe arduino system (BLE, ...)
\item which kind of sensors and why (range is adjustable, how comes the conversion factor in, ...)
\item hard-soft iron effects
\item calibration methods for sensors (hard-soft things / Freescale approach)
\item also calibration/scaling towards the magnetic model equations! \\
		$ \rightarrow $ scaling/offset termination on flat paper with (almost) exactly known positions
\item the circumstance of the earth magnetic field and how to overcome it \\
		$ \rightarrow $ rotation estimation with Madgwick filter! Mention that you take an out of the box code for it
\end{itemize}


\section{Deriving the Human hand model} \label{sec:handModel}

\begin{itemize}
\item neglecting adduction-abduction
\item angle of PIP is 2/3 of DIP
\item bring here the picture of the sketched hand from the equation document!
\item describe that you can break every position/orientation of a fingertip to a combination of MCP and DIP
\item assumption, that every pose leads a unique B-field
\item only right hand
\end{itemize}


\section{Magnetic field interpretation } \label{sec:magmodel}

\begin{itemize}
\item describe the derivation of the position vector, relative to sensor position, joint angles, ... \\
		$ \rightarrow $ so the whole $ P_{sensor} - P_{fingertip} - P_{joint} $ thing
\item the derivation of the orientation vector (dip model) and orientation angle (cyl model)
\item how to formulate/evaluate/modify the B-field equations, to get the outcome\\
		$ \rightarrow $ for the dipole model (derivation of the orientation is \grqq a bit tricky \grqq)\\
		$ \rightarrow $ describe the whole turning and transforming process for the cylindrical model
		
\end{itemize}		


\section{Hand state estimation} \label{sec:estimation}

\begin{itemize}
\item general description of minimizing the error
\item how do I set up my matrices and therefore, how does my optimization problem look like
\item optional constrains
\item solving it with the EKF approach
\end{itemize}


\section{Visualization} \label{sec:visual}

\begin{itemize}
\item the Blender thing...
\item using a rigged hand model
\item setting it up as a Blender Game
\item I only pass the estimated joint angles
\item briefly describe bpy interface and how I manipulate the rigged hand
\end{itemize}






\lhead[\chaptername~\thechapter]{\rightmark}


\rhead[\leftmark]{}


\lfoot[\thepage]{}


\cfoot{}


\rfoot[]{\thepage}


\chapter{Introduction}

The human hand is one of the most important part of the body for object interaction and communication. The five fingers show a broad range of motion and can perform powerful gestures, such as grasping, as well as sensitive and accurate movements like painting. In order to examine the natural behaviour of the human hand, some special motion tracking systems, capable of the complex finger movements, were developed till now. Those systems can be used for medical treatment of hand injuries or movement impairment, caused for example by an accident or stroke. Another novel field of application would be the interpretation towards \ac{HCI}, to establish a natural way of interacting with devices. Motion tracking of the human body as a whole is not a new feature, although there is a lot of research going on in this area. For hand and finger tracking however, those general approaches have to be adjusted, since the granular but also complex finger motions are often performed only within a small movement range and need therefore a system with a higher resolution rate. Traditionally the developed hand tracking methods use standard motion capture approaches, such as cameras or \acp{IMU} to identify and reconstruct various movements. In order to make them capable for hand pose reconstruction, those solutions are often bulky and introduce external components. Those systems seem not to stand in a relationship to the small and neat hand. One approach, that uses the limited region of interest for hand state estimation, has received only little attention till now. The measuring of magnetic fields, excited by permanent neodymium magnets attached to the fingertips. This approach would decrease the number of sensors and external components for finger state estimation. Since the decrease of the magnetic field, excited by a static magnet decreases with the distance, this method is not so well suited for body tracking. Since the human hand is not too big and the overall measurable magnetic flux density can be adjusted by choosing suitable magnets, this approach seems promising for the objective.

The aim of this thesis is to develop a system for the reconstruction of finger joint angles, by measuring the magnetic flux densities, excited by artificial magnets on the fingertips. The system should present a novel approach, which comes with less bulky and complex components as the so far developed ones. Therefore the system size and usability is desired to be held small and simple. The thesis starts with a general review of related work on hand motion reconstruction and the possible fields of applications. After introducing the general anatomic and magnetic foundations, the hardware and software components of the developed approach will be described. A model will be evaluated to describe the magnetic flux densities, measurable at the deployed sensors, for each finger pose. From that, a reconstruction algorithm mapping the measured superimposed magnetic field to the joint angles of the hand will be implemented. The results of the developed system for the estimated finger poses will be compared to the values of a commercially available camera based approach. Therefore, the overall performance of the magnetic system regarding the physical resolution and accuracy of joint angle reconstruction will be determined. 





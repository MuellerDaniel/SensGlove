\lhead[\chaptername~\thechapter]{\rightmark}

\rhead[\leftmark]{}

\lfoot[\thepage]{}

\cfoot{}

\rfoot[]{\thepage}

\chapter{Results} \label{cha:results}

\section{Data behaviour / Sensor values} \label{sec:dataRes}

\begin{itemize}
\item general behaviour of sensor (when you get to near, there is a clipping/oversteering)
\item timing behaviour with the earth-mag elimination

\item comparing the calibration methods \\
		$ \rightarrow $ hard-soft as good as Freescale, took Freescale
		
\item observations from the earth-mag elimination \\
		$ \rightarrow $ not to 100 \%  possible, but still better than nothing...
		
\item problems with determining the model fit parameters \\
		$ \rightarrow $ difference in orientation of magnet \\
		$ \rightarrow $ tried to fit it to wooden hand, but there the error is already in the simulated(fitting) data \\
		$ \rightarrow $ hard to derive!
		
\end{itemize}


\section{Pose estimation} \label{sec:estimationRes}

\begin{itemize}
\item results of minimization for simulated data with and without noise
\item effect of adding constraints
\item EKF results

\item results with real data
\item as described, sensor values are not good scalable to model...
\item though not so good results...
\item due to timing behaviour of RT minimization, not such a good way to do have a real time estimation
\item so I have results for single magnet - one/multiple sensor 
\item and for multiple magnet - multiple sensor combinations
\item presentation and comparison between EKF and minimizing approach
\item claiming that the difficulty in scaling the measurements is critical...

\end{itemize}




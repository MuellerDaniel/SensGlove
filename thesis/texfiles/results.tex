\lhead[\chaptername~\thechapter]{\rightmark}

\rhead[\leftmark]{}

\lfoot[\thepage]{}

\cfoot{}

\rfoot[]{\thepage}

\chapter{Experimental Results} \label{cha:results}

\section{Sensor behaviour} \label{sec:dataRes}

The utilized LSM303D sensors show some general measurement characteristics. If they are exposed to a magnetic field, higher than the configured measurement range, a clipping of the returned value can be observed. In \ref{fig:clipping} this effect can be seen on the observed values for the $ x $-axis. The magnet is moved along this axis towards the sensor. As expected, the measured field increases/decreases, by shrinking the distance $ \Delta d $ between sensor and magnet. Looking at \ref{fig:negClip}, approximately at a distance of \SI{4.5}{\cm} between sensor and magnet, the measured field reaches the current lower range. The sensor first stays a while on this value, before it jumps to positive. By turning the magnet around \SI{180}{\degree} and therefore measuring an increasing magnetic field by the sensor, a similar behaviour can be recognized. This time the clipping obviously occurs from positive to negative and happens at a gap of \SI{4}{\cm} (see \ref{fig:posClip}). Further on, this time the returned value clips directly and does not stay on a maximum value. To overcome this effect, one has to set the magnetic full scale range to an appropriate value. However by setting for every user the maximum range of $ \pm \SI{1.2}{\milli \tesla} $, one looses precision in the measurements. The approximately reachable maximum full scale range for one user can easily be determined. By simulating the range of possible flexion-extension, which would be performing a fist, and looking at the predicted outcome of the  model, one gets an image for the result of the measurable magnetic field. From that the measurement range can be determined. However due to the influence of surrounding magnetic fields, this value is only a guideline for the de facto measured field. Based on this context, the magnetic full scale range of the sensors for the ongoing measurements and experiments was set to $ \pm \SI{0.4}{\milli \tesla} $.

\begin{figure}
\subfloat[Moving the negative pole towards the sensor]
{\includegraphics{pictures/plots/negClipping.png} \label{fig:negClip}}
\hfill
\subfloat[Moving the positive pole towards the sensor]
{\includegraphics{pictures/plots/posClipping.png} \label{fig:posClip}}
\caption{The magnet is moved towards the measurement unit, to record the clipping behaviour of the sensors. The distance from sensor to magnet is represented by $ \Delta d $ in \si{\cm}. For \label{fig:negClip}, the oversteering of the sensor begins at around \SI{4.5}{\cm}, for \label{fig:posClip}, this effect starts at \SI{4}{\cm}. The sensor range is adjusted to be $ \pm \SI{0.4}{\milli \tesla} $, which can be observed by the dataplots.}
\label{fig:clipping}
\end{figure}

For evaluating the timing behaviour of the system, the code on the RFduino is debugged. As described in \ref{cha:sensors}, the sensor data rate could be set to a maximum value of \SI{100}{\Hz}, such that one could retrieve new magnetometer values each \SI{10}{\milli \second}. The switching and forwarding of the clock signal via the utilized multiplexer takes only \SI{21}{\nano \second} into account. This value composes a \grqq Break-before-make\grqq \, pause of \SI{6}{\nano \second}, to prevent crosstalk between the channels and a propagation delay time of \SI{15}{\nano \second}. In order to verify those values and to identify the overall time for acquiring, scaling to distortion factors, sending and receiving the measurement data by the host system, the respective code sections were timed. It is observed that the overall sampling frequency of the sensors can only be set to \SI{50}{\Hz}. So the time between two new sensor read outs is \SI{20}{\milli \second}. The read out of the registers and the scaling for the hard- and soft-iron distortion values shows an insignificant influence on the timing. Further on it makes no difference whether only one sensor unit is read out, or all four, since they all show the same data rate and have measurements available after \SI{20}{\milli \second}. 
The sending via BLE is implemented by the RFduino environment. The maximum transferable packet size is 20 bytes \cite{rfduino2015data}. The three float values of one sensor, plus an additional float for indicating the device number have a size of 16 bytes. The RFduino has implemented a queue of size 20 bytes, where the data is stored, till it is sent. The sending frequency depends on the distance between the host PC (which represents the client) and the RFduino module (which is the server). It is specified to range from \SI[per-mode=symbol]{32}{\kilo \bit \per \second} to \SI[per-mode=symbol]{24}{\kilo \bit \per \second}. In order to ensure that no data packet is overwritten, before it is sent, one has to check the size of the queue. The client registers via the \ac{GATT} protocol for listening to the notifications of the microcontroller. For the ongoing interpretation of the sensor values, only measurements from all four units are interesting. Therefore, for identifying the overall data rate, the time for receiving four individual data packets from the server is measured. It is observed, that the receiving rate is not constant. For a sensor rate of \SI{50}{\Hz} it is observed, that approximately every \SI{50}{\ms} four new packets are received. This leads to a frequency of \SI{20}{\Hz} for the whole system. Since the sensors are triggered with a frequency more than twice as high as the values can be received, their quality decreases. Therefore the data rate of the sensors is reduced to \SI{25}{\Hz}, in order to try to acquire more representative measurements. By doing this, the system frequency decreases to \SI{12.5}{\Hz}. However, this leads only to a slightly more representative measurements, since the frequency for receiving the obtained data packets is still twice as high as the sensor data rate. Those results were observed, by measuring the acquisition time of 200 packets (each representing the measurements of four sensor units). The stated system frequencies represent the mean over the 200 observed timestamps. The histogram \ref{fig:sensTime}, represents the distribution of the measured duration for both sensor frequencies.

%\todo{Source of error for timing: The sending via BLE is time critical!... The RFduino safes the data into a queue. When you just put more and more data into this queue, while it is not sent (or without taking care whether the queue is full), you will overwrite entries! With checking the \grqq fullness \grqq of the queue with a while-loop (look at code!) you wait, till you are allowed to push the data to the queue. This ensures, that you don't overwrite entries. However your system is stuck during that time and since the sensor datarate with 50Hz is faster than the sending, data packets are lost! So you should try to adapt the sensor data rate to the sending rate. Taking one measurement out of 5(case with $ f_{sens} \gg f_{ble} $) is less good, than taking one out of 2(case for $ f_{sens} \geq f_{ble} $ ). This could lead to an overall slower acquisition of packets (since you have to wait longer for you sensors), but the values from the sensor are more representative!}

\begin{figure}[h]
\centering
\includegraphics{pictures/plots/timingRFd_v2.png}
\caption{200 acquisitions of complete measurement packets, comprising the values of four sensor units, where timed. One time the data rate of the sensor is set to \SI{50}{\Hz}, the other to \SI{25}{\Hz}. The mean value for $ f_{Sensor}=25\si{Hz} $ is higher than, for $ f_{Sensor}=50\si{Hz} $. However the variance of the acquisition time is smaller for this and therefore the outcome is more consistent to itself. Further on the sensor values from a data rate of \SI{25}{\Hz} represent the actual obtained measurements slightly better, than the higher data rate. In the ongoing evaluation, the results for both data rates are taken into account and compared to each other.}
\label{fig:sensTime}
\end{figure}




\section{Quality of calibration procedures} \label{sec:cali}

\subsection{Calibration for hard- and soft-iron effects}\label{subsec:resHardSoft}

Two methods are compared and classified for determining the hard- and soft-iron factors. On the one hand the approach, presented by Winer, declaring hard- and soft-iron values by using the maximum and minimum of the obtained measurements for an axis. This one is chosen, since it is an often cited and easy method for compensating the distortion factors. On the other hand, the version from Freescale \cite{ozyagcilar2012calibrating} which takes a whole series of measurements into account and only compensates for the hard-iron effects. 1000 measurements were collected, by rotating the sensor slowly around all possible axes. The environment is a normal lab, without any protections against additional, artificial magnetic fields. 

In \ref{fig:hs2d} the measurements for each axis combination are plotted in a one dimensional representation, for a clearer identification. The raw values are represented by the red dots, the calibrated by the green and cyan ones. The outline of the corresponding ideal sphere is visualized by the blue circle. It can easily be observed, that the hard iron effects dominate the soft iron factors. The scaling factors for the soft iron values, obtained by the Winer approach also reflect this. They lie in the range of $ 1 \pm 0.03 $. Another fact is, that both calibration methods lead almost the same results. In order to compare the quality of the two calibration methods, the distance of each calibrated measurement value to the perfect sphere with radius $ \mathrm{B}_earth $ is calculated. The deviation is plotted as a histogram in \ref{fig:devi}. The obtained mean for the Freescale approach is calculated to be $ \mu_{Freescale} = -0.02\si{\milli \tesla} $, for the calibrated values, using the method of Winer to be $ \mu_{Winer} = -0.8\si{\milli \tesla} $. So in the end the values calibrated by the Freescale approach represent slightly more the shape of a perfect centered sphere. One reason for this is, that for the utilized sensors the hard iron distortion effects dominate over the soft iron ones. Further on, since the whole measurement series is taken into account, the behaviour of the sensor is represented much better. The Winer approach is very sensitive for noisy signals, since the peak values characterize the calibration factors. One has to note, that this holds only for the used sensor units. For another \ac{PCB} environment or measurement unit, the obtained values could be different and the soft iron factors could show a higher influence. For this, the Winer approach would probably lead to better results, than the Freescale. So in the end the calibration has to be verified and adjusted for the specific sensor and application.\\
The presented procedure was evaluated for several times and sensors, each time showing similar and constant results. As already mentioned in \ref{subsec:hardSoft} this calibration procedure has to be performed for each sensor and the observed values all have to be scaled to a common value $ \mathrm{B}_{earth} $.


\begin{figure}
\centering
	\subfloat[Obtained measurements along the $ x $ vs. the $ y $ axis]
	{\includegraphics{pictures/plots/cali_xy.png}\label{fig:xy}}
%	\hfill
	\subfloat[Obtained measurements along the $ y $ vs. the $ z $ axis]
	{\includegraphics{pictures/plots/cali_yz.png}\label{fig:yz}}
	\hfill
	\subfloat[Obtained measurements along the $ x $ vs. the $ z $ axis]
	{\includegraphics{pictures/plots/cali_xz.png}\label{fig:xz}}
\caption{The measurements where recorded by rotating the sensor around each axis in an environment without artificial magnetic sources. 1000 measurements were collected. The obtained raw values are represented by the red dots. The results of the calibration procedures are plotted by the respectively color. The perfect centered sphere is represented by a circle with $ r=\mathrm{B}_{earth} $. Already the unscaled values show only a very low influence of soft iron distortion. It is also observed, that the calibrated results do not differ much.}
\end{figure}

\begin{figure}
\centering
\includegraphics{pictures/plots/cali_devi.png}
\caption{The procentual deviation of the two calibration methods to the corresponding perfect centered sphere is visualized. It can be obtained, that the Freescale approach leads to slightly more accurate results, since the mean and variance of the scaled measurements are smaller than for the Winer approach.}
\label{fig:devi}
\end{figure}



%In \todo{3d} the calibrated and the uncalibrated data points are plotted, with a sphere around them. One thing that is clearly visible here is the shift, caused by the hard iron effects. In \todo{2d} this offset is also observable. Recognizable by those two plots is also the fact, that the influence of the soft-iron distortion is very small. The data points lie already on an almost perfect sphere. This is also represented in those scale values, calculated by the Winer approach. They lie in the range of $ 1 \pm 0.03 $. That both calibration methods serve almost the same results can already be seen by the fact that the blue and green datapoints are overlapping each other. In order to compare the two calibration methods, the distance of each calibrated value to the perfect sphere with radius $ \mathrm{B}_earth $ is calculated. The outcome is shown in \todo{deviation}. In the end, the mean value of the deviation for the Freescale approach is smaller than those, calibrated by the Winer method (for this particular example: $ \mu_{Freescale} = -0.0842\si{\micro \tesla}, \mu_{Winer} = 6.5525\si{\micro \tesla} $). Because the hard-iron offsets dominate the soft-iron effects, the Freescale method is more accurate than the Winer approach, since it takes all the observed measurements into account and not just the minimum and maximum values. Therefore for the utilized sensor units, which show small soft-iron deviation, the Freescale approach is preferred. The provided procedure was evaluated several times, each time showing similar and constant results. As already mentioned in \ref{subsec:hardSoft} this calibration procedure has to be performed for each sensor and the observed values all have to be scaled to a common value for $ \mathrm{B}_{earth} $.


\subsection{Determining the fitting parameters for the model} \label{subsec:resModelFit}

For determining the scaling factors to the model, the sensor rack is placed onto a cardboard box with a height of \SI{2}{\cm}. The magnet is statically aligned in the same direction as the sensor $ x $-axis and is moved along the y-axis on a flat surface. The distance in $ x $-direction is held static. The origin of the coordinate frame is determined to be the position of the upper most sensor unit. The magnet is moved at a distance of $ x=5\si{\cm} $ and $ x=7\si{\cm} $ from $ y=-6\si{\cm} $ (which is the $ y $-height of the under most sensor) to $ y=0\si{cm} $ (which is the $ y $-heigth of the upper most sensor). For defining the scaling factors, the B-field, which should be observed by the sensor is calculated. The range of the actually measured values is then fitted to the simulated ones for each axis. The scaling factors for the two different $ x $-positions should be same. However this is not the case. The comparison of the two scaling factors is visualized in \ref{fig:flatFit} for $ x=7\si{cm} $.

\begin{figure}[h]
\includegraphics{pictures/plots/flatFit.png}
\caption{Comparison between the scaling factors, observed from $ x=5\si{\cm} $ and $ x=7\si{\cm} $. The simulated data is for a distance of $ x=7\si{\cm} $. Please note, that the measured values cannot be compared directly to the results of the simulation. The simulation assumes a constant movement of the magnet. This however cannot be performed perfectly, since it is done by hand. Due to this fact, also the distance in $ x $ direction can slightly change. The plot reflects the differences between the two different scaling factors and the raw values for each sensor axis. In this way it can be observed, that the results show the most variation along the magnetic $ x $-axis. \todo{arange plot x-axis!}}
\label{fig:flatFit}
\end{figure}

The maximal difference between the two observed factors is $ 0.195 $, the mean over all differences is $ 0.12 $. This deviation between the results is not negligible. An exact difference value between the perfect B-field values from the model and the actually observed and scaled ones is not directly possible. The values of the model assume a movement with a constant velocity. Since the magnet motion is performed by hand, the measured field is not changing constantly. However for this reason, the magnet is held still at some dedicated positions, to get an impression on the difference of the scaled measurements to the calculated field. This is represented by the flat sections in the graph. The deviation from the perfect values for those positions is calculated and normed, to get a representation of the overall magnetic field strength. \ref{tab:diffScaled} represents the results.
\begin{table}[h!]
\centering
\begin{tabular}{|c|c|c|c|}
\hline
y-Position [\si{cm}] & scaled to $ x=5\si{cm} $ [\si{\micro \tesla}] & scaled to $ x=7\si{cm} $ [\si{\micro \tesla}] & raw [\si{\micro \tesla}] \\ \hline
-6    & 0.0  &  0.0  & 0.0 	\\ \hline
-4    &	0.94 &  2.5  & 3.7 	\\ \hline
-2    &	2.5  &  2.6  & 5.7 	\\ \hline
0.0	  &	6.0  &  0.0  & 6.3 	\\ \hline
\end{tabular}
\caption{The deviation of the scaled magnetic field to the simulated at the stopped positions. The observed values show, that an exact fitting, by scaling only for the maximum and minimum values for a certain movement does not lead identical and exact results.}
\label{tab:diffScaled}
\end{table}

As a source for the observed disagreement, one could head that the whole procedure is performed by a human being. The position values are only determinable up to a certain amount of accuracy, as well as the start and end points of the performed movement. Since this calibration procedure does not result in clear scaling values for the sensors, it is not further used in this work. For scaling the observed sensor values to the model, a predefined fitting gesture is used, as described in \ref{subsec:modelFit}.  


\subsection{Elimination of earth magnetic field} \label{subsec:resEarthEli}

Since a constant elimination of the earth magnetic field would be very important for a portable system, two methods of the approach, presented in \ref{subsec:earthEli} are tested. The difference between those two lies in the determination of the sensor orientation. The one estimates it by using an implementation of a Madgwick Filter, provided from \cite{mikeshub2012}. This algorithm can directly be executed on one sensor device, since the accelerometer and the gyroscope are already on the breakout board. So for this method, no additional sensors have to be mounted onto the sensor bracket. The other approach uses an additional \ac{IMU}, which can output the orientation directly as quaternion. The MPU9250 from Invensense \cite{MPU2014} is used for this. The orientation of the magnetometers relative to each other does not change, since they are placed inside the self designed bracket. Therefore it is sufficient, to determine the orientation of the sensor rack only. For implementation follow the steps, presented in \ref{subsec:earthEli}. As an intermediate step, the calculated relative orientation $ R_{d} $ of both methods was inspected and was proven to represent the real relative rotation.

As an early observation, the approach using the Madgwick filter is considered bad. Since the readings from the magnetometer are used, for guaranteeing a stable and non-drifting estimation of the orientation, the artificial magnets interfere this algorithm. This was observed by a constant drift of the values over time, when introducing the artificial magnets. So the further verification was only performed with the MPU9250 sensor, with whom this drift behaviour was not observed. Nevertheless it is mentionable that the upcoming results for cancelling solely the earth magnetic field (in absence of artificial magnets) were similar for both methods.

A proper working system should constantly return a magnetic field of almost \SI{0}{\tesla}, when it is rotated in an environment without artificial magnets. In order to verify this, the sensors are slowly moved around each axis. By comparing the results with and without the subtraction of the initially observed magnetic field, one should get an impression on the quality of the algorithm. In \ref{fig:earthCancelRes} the observed data of each axis is displayed.

\begin{figure}[h]
\centering
\includegraphics{pictures/plots/earthCanc.png} 
\caption{The result for the cancellation of the earth magnetic field, relative to the sensor rotation is displayed. One change in orientation is performed, to observe the capability of constantly subtracting the initially observed surrounding magnetic field. The plot can be divided into three orientation phases. The initial orientation 0, orientation 1 and orientation 2. That the surrounding magnetic field can be cancelled is shown section 0 and 2. However the large deviations from \SI{0}{\tesla} during the movement in between the two orientation sections show, that the implemented approach does somehow not work for every rotation. As comparison, the raw values without subtracting the rotated surrounding magnetic field are also plotted.}
\label{fig:earthCancelRes}
\end{figure}

The two plots show the measured magnetic field along all three axes for one sensor with (represented by the green line) and without (represented by the yellow line) subtracting the rotated initially observed field $ \mathrm{B}_{earth} $. It is visible, that unfortunately the elimination method does not work properly. At the beginning, the B-field with the cancellation is 0. However by rotating the device, the observed field changes a lot. For the results along the x-axis, the offset can be compensated relatively good. But for the values observed along the y- and z-axis, this does not hold. Moving the sensor in its starting position again, one sees, that the surrounding field is eliminated pretty well again. This short example visualizes only a simple movement around one axis. Even for this, the surrounding magnetic field can not be eliminated. Small changes could be claimed upon calibration errors or small static magnetic sources in the environment, such as cell phones or metal bodies. But the observed deviation from 0 is much higher than this. So in the end, the surrounding magnetic field can not be cancelled with the presented method. Further investigation has to be done for this. However, since this work focuses on the evaluation for pose estimation with magnets, the cancellation of the earth magnetic field is left by that. The proposed method would just has been a benefit for the work, by bringing it to a mobile system. So for the ongoing evaluation, it has to be noted, that the hand is always held still. 




\section{Pose estimation} \label{sec:estimationRes}

\subsection{Quality/Results for simulated data; Identifying the minimization process} \label{subsec:resSim}

\begin{itemize}
\item $ \rightarrow $ try to get a feeling for the shape of the data/B-field
\item the difference between the results for the dip and the cyl model (just difference plot...)

\item describe the three different minimization algorithms \\
	$ \rightarrow $ unconstrained BFGS, COBYLA ... forget COBYLA, since it only returns shitty results...\\
	$ \rightarrow $ constrained SLSQP (this is the choice!)
	
\item describe how you tested it, so describe the script 160203\_miniComp.py
\item explain, that you always take ad-ab for the b-field into account, since it represents a normal movement and anyway you only have it at some simulated states

\item 1 magnet - 1 sensor
	\begin{itemize}
	\item overall bad results...
	\item best results for 1dip/cylA
	\item or good results, neglecting ad-ab (1dip/cylnA)
	\item pretty fast way however (ca 0.05 sec)
	\end{itemize}
\item 1 magnet - 4 sensors
	\begin{itemize}
	\item good results ad-ab, regardless of method	
	\item good speed (ca. 0.1 sec)
	\item best result for 1dipA (also fastest! 0.07 sec) and 1cylA (0.05 sec)
	\end{itemize}
\item 4 magnets - 4 sensors
	\begin{itemize}
	\item very slow! 
	\item bad results!
	\item best results for: 1dipA, 1cylA (also the fastest)
	\end{itemize}
	
\item 4 magnets - 8 sensors
	\begin{itemize}
	\item test it...
	\end{itemize}	
	
\item overall results/conclusion for algorithms
	\begin{itemize}
	\item method 1 is best!
	\item 14 leads best results
	\item 44 very time consuming and inaccurate
	\item 11 also very inaccurate
	\end{itemize}	
		
\begin{itemize}
\item results of minimization/EKF for simulated data with and without noise
\item results for easy movement (making a fist/90 deg fit) with and without noise
\item results for complex movement with and without noise \\
		$ \rightarrow $ simple movements pretty good reconstructible\\
		$ \rightarrow $	complex movements and noise not so good...\\
		$ \rightarrow $ when estimating more than one finger, very time consuming...\\
		$ \rightarrow $	behaviour as with singe finger
\end{itemize}

\end{itemize}



\subsection{Results for recorded/real data} \label{subsec:resMeas}

Describe the recording procedure

\begin{itemize}

%
\item \textbf{estimation of four fingerstates}
	\begin{itemize}
	\item very slow! And very poor results!
	\item 160210\_set6 nice for showing things for all fingers
	\item no good results without ad-ab (for 90 fitted gesture)
	\item moving together can a bit be tracked (initialization gesture reconstructed, for dip better than cyl)
	\item individual movement can not be tracked! (values 257:471)
	\item $ \rightarrow $ so no further details about four - four estimation, it is not possible!
	\end{itemize}

\item textbf{estimation one - one}
	\begin{itemize}
	\item results are \grqq a bit\grqq worse than with four sensors (should be valid for all sets, tested for 160217\_set3 (no ad-ab))
	\item results are much worse than with four sensors (160210\_set2)
	\item ad-ab can also be recognized (160217\_set4)
	\item \todo{more comparison of datasets!}
	\item estimation is ca. \textbf{two times} faster than one - four!
	\item e.g. four-one better than one-one 160210\_set4 (beginning of fist)
	\end{itemize}

%
\item \textbf{recognizing movements}
	\begin{itemize}
	\item not constantly \grqq good or bad\grqq, it differs and is dependent on beforehand movements (seen by almost all datasets) and movement of hand
	\item 160217\_set3 nice detection between 50s and 70s by cyl\_A (but directly afterwards, stretched position is not recognized...)
	\end{itemize}


\item \textbf{difference between model with and without ad-ab}:
	\begin{itemize}
	\item with ad-ab results are better (even for doing no movement in ad-ab)
	\item $ \rightarrow $ no ad-ab results over all are not so good...
	\item 160217\_set7: for simple movements almost no difference to with ad-ab
	\item 160210\_set1: big difference (no ad-ab worse...)
	\end{itemize}


\item \textbf{comparing results/difference for using cyl and dip}
	\begin{itemize}
	\item dip is much faster than cyl (around 7 times faster!)
	\item regarding accuracy, no clear statement possible...
	\item sometimes cyl is better, sometimes dip

	\end{itemize}

\item \textbf{detectability of ad-ab / comparing cyl\_A with dip\_A}
	\begin{itemize}
	\item 160210\_set4 shows that dip and cyl without ad-ab show only small differences!
	\item in the end: cylindrical estimation with ad-ab is more accurate than dipole with ad-ab! (160217\_set3)
	\item well it really depends... sometimes dip is better, sometimes cyl...
	\item but ad-ab can be estimated much better by cyl model, but only if finger is stretched! (160217\_set2)
	\item and only slow and clear ad-ab movements can be detected (160210\_set3 nothing is detected...)
	\item 160217\_set4 cyl\_A detects negative AND positive ad-ab!
	\item $ \rightarrow $ ad-ab is hard to detect. Even Leap is not capable
	\end{itemize}

\item \textbf{comparison to Leap}
	\begin{itemize}
	\item comparison is difficult, since it is also not perfect...
	\item 160210\_set1-2 are good examples, which show, that Leap is not perfect!
	\item 160210\_set2 however also shows, that it can work pretty good!
	\item values for ad-ab are not very good (e.g. 160210\_set3)
	\item 160210\_set4 also shows clearly (especially at the end), that Leap is not perfect!
	\item 160210\_set4 shows, that ad-ab can be detected quite good (70s-90s)
	\end{itemize}

\item \textbf{issues/observations with Leap}
	\begin{itemize}
	\item when finger are close together, tracking is harder
	\item DIP and PIP show the same constraint as I am using ($ PIP = 2/3 DIP $)
	\item angle in MCP introduces most times also angle in DIP/PIP (160210\_set2) \\
			$ \rightarrow $ my system is better for this!
	\end{itemize}

%
\item \textbf{influence of normed fitting gesture / use fist? and observed fitting values}
	\begin{itemize}
	\item scaling to flat scale fitting values leads to very bad results!
	\item fist is not good for fitting, since it is slightly different each time and everyone makes it different (pictures of videos and observed B-fields)
	\item (cardboard) has only slightly influence (dataset 160212\_set1)
	\item do it during the video (also with cardboard perhaps)
	\item because you change the height and orientation of your hand too much during calibration...
	\item so just perform the gesture during your recording
	\item fitting for ad-ab has almost no influence (dataset 160212\_set3), a clear, structured (performed alone) ad-ab can be detected (dataset 160212\_set4)
	\end{itemize}

%
\item \textbf{influence of exact hand dimensions/parameters}
	\begin{itemize}
	\item e.g. dataset: 160217\_set3 with handDim from 160210 and 160217 (but every other set should return the same...)
	\item for sPos there is a difference, but it is very small
	\item for bone lengths it's the same, only small differences (sometimes my own lengths better, sometimes wooden better)
	\item so in the end the parameters have to resemble the truth, but since the data is fitted to the calculated values, it's influence is not too big
	\end{itemize}

%
\item \textbf{influence of distance sensor to magnets}
	\begin{itemize}
	\item changes in magnetic field are too small...
	\item 160210\_set10 shows that the sensors are too far away! (the results within the methods vary a lot and they do not represent the truth!)
	\end{itemize}

%
\item \textbf{tries with MPU}
	\begin{itemize}
	\item as expected not good...
	\item start position can be \grqq detected\grqq after rotation again, but the results with rotation are bad...
	\item sets: 160217\_set5-6
	\end{itemize}

\item \textbf{results with ring}
	\begin{itemize}
	\item 160217\_set7 no difference to glued magnet
	\item 160217\_set8 very bad results... but I don't think that the ring is the reason
	\item \todo{do some more?}
	\end{itemize}

\item \textbf{for identifying/measuring the difference}
	\begin{itemize}
	\item plot the two results for two/several methods in one graph (colors)
	\item and plot numerical difference below
	\end{itemize}

\item \textbf{concluding observations}
	\begin{itemize}
	\item very fragile/sensitive system
	\item sensitive to hand/body movements, calibration gesture
	\item bad/difficult reproducibility of results/measurements
	\item quality of results is not totally comparable/identifiable by Leap (since this system is also faulty)
	\end{itemize}

\item 160210\_set4 nice for showing that Leap cannot detect DIP/PIP movement alone...

\item setting datarate to 25Hz has no effect... (160217\_set2)

\end{itemize}

Presentation and compare between EKF and minimizing approach

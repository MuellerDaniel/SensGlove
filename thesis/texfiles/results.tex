\lhead[\chaptername~\thechapter]{\rightmark}

\rhead[\leftmark]{}

\lfoot[\thepage]{}

\cfoot{}

\rfoot[]{\thepage}

\chapter{Results} \label{cha:results}

\section{Sensor Behaviour} \label{sec:dataRes}

The utilized LSM303D sensors show some general measurement characteristics. If they are exposed to a magnetic field, higher than the configured measurement range, a clipping of the returned value can be observed. In \ref{fig:clipping} this effect can be seen on the observed values along the magnetic $ x $-axis. The magnet is moved along this axis towards the sensor. As expected, the measured field increases/decreases, by shrinking the distance $ \Delta d $ between sensor and magnet. Looking at \ref{fig:negClip}, approximately at a distance of \SI{4.5}{\cm} between sensor and magnet, the measured field reaches the current lower range. The sensor first stays a while on this value, before it jumps to positive. By turning the magnet around \SI{180}{\degree} and therefore measuring an increasing magnetic field by the sensor, a similar behaviour can be recognized. This time the clipping obviously occurs from positive to negative and happens at a gap of \SI{4}{\cm} (see \ref{fig:posClip}). Further on, this time the returned value clips directly and does not stay on a maximum value. To overcome this effect, one has to set the magnetic full scale range to an appropriate value. However by setting for every user the maximum range of $ \pm \SI{1.2}{\milli \tesla} $, the precision of the measurements decreases. The approximately reachable maximum full scale range for one user can easily be determined. By simulating the range of possible flexion-extension, which would be performing a fist, and looking at the predicted outcome of the  model, one gets an image for the result of the measurable magnetic field. From that the measurement range can be determined. However due to the influence of surrounding magnetic fields, this value is only a guideline for the de facto measured field. Based on this context, the magnetic full scale range of the sensors for the ongoing measurements and experiments is set to $ \pm \SI{0.4}{\milli \tesla} $.

\begin{figure}[!htb]
\subfloat[Moving the negative pole towards the sensor]
{\includegraphics{pictures/plots/negClipping.png} \label{fig:negClip}}
\hfill
\subfloat[Moving the positive pole towards the sensor]
{\includegraphics{pictures/plots/posClipping.png} \label{fig:posClip}}
\caption[Clipping behaviour of sensor]
{The magnet is moved towards the measurement unit, to record the clipping behaviour of the sensors. The distance from sensor to magnet is represented by $ \Delta d $ in \si{\cm}. For the left figure, the oversteering of the sensor begins at around \SI{4.5}{\cm}, for \label{fig:posClip}, this effect starts at \SI{4}{\cm}. The sensor range is adjusted to be $ \pm \SI{0.4}{\milli \tesla} $, which can be observed by the dataplots.}
\label{fig:clipping}
%python script: 160214_clippingDetection.py
\end{figure}
For evaluating the timing behaviour of the system, the code on the RFduino is debugged. As described in \ref{cha:sensors}, the sensor data rate could be set to a maximum value of \SI{100}{\Hz}, such that one could retrieve new magnetometer values each \SI{10}{\milli \second}. The switching and forwarding of the clock signal via the utilized multiplexer takes only \SI{21}{\nano\second} into account. This value composes a ``break-before-make'' pause of \SI{6}{\nano \second}, to prevent crosstalk between the channels and a propagation delay time of \SI{15}{\nano \second}. In order to verify those values and to identify the overall time for acquiring, scaling to distortion factors, sending and receiving the measurement data by the host system, the respective code sections were timed. It is observed that the overall sampling frequency of the sensors can only be set to \SI{50}{\Hz}. The read out of the registers and the scaling for the hard- and soft-iron distortion values shows an insignificant influence on the timing. Further on it makes no difference whether only one sensor unit is read out, or all four, since they all show the same data rate and have measurements available after \SI{20}{\milli \second}.\\
The sending via BLE is implemented by the RFduino environment. The maximum transferable packet size is 20 bytes \cite{rfduino2015data}. The three float values of one sensor, plus an additional float for indicating the device number have a size of 16 bytes. The RFduino has implemented a queue of size 20 bytes, where the data is stored till it is sent. The sending frequency depends on the distance between the host PC (which represents the client) and the RFduino module (which is the server). It is specified to range from \SI[per-mode=symbol]{32}{\kilo \bit \per \second} to \SI[per-mode=symbol]{24}{\kilo \bit \per \second}. In order to ensure that no data packet is overwritten, before it is sent, one has to check the size of the queue each time before writing to it. The client registers via the \ac{GATT} protocol for listening to the notifications of the microcontroller. For the ongoing interpretation of the sensor values, only measurements from all four units are interesting. Therefore, for identifying the overall data rate, the time for receiving four individual data packets from the server is measured. It is observed that the receiving rate is not constant. For a sensor rate of \SI{50}{\Hz} approximately every \SI{50}{\ms} four new packets are received. This leads to a frequency of \SI{20}{\Hz} for the whole system. This high deviation from the actual possible sensor data rate is caused by the low sending frequency of the RFduino. Since the sensors are triggered with a frequency more than twice as high as the values can be received, their quality decreases. Therefore the data rate of the sensors is reduced to \SI{25}{\Hz}, in order to try to acquire more representative measurements. By doing this, the system frequency decreases to \SI{12.5}{\Hz}. However, this leads only to slightly more representative measurements, since the frequency for receiving the obtained data packets is still twice as high as the sensor data rate. Those results were observed, by measuring the acquisition time of 200 packets (each representing the measurements of four sensor units). The stated system frequencies represent the mean over the 200 observed timestamps. \ref{fig:sensTime} represents the distribution of the measured duration for both sensor frequencies.
%\todo{Source of error for timing: The sending via BLE is time critical!... The RFduino safes the data into a queue. When you just put more and more data into this queue, while it is not sent (or without taking care whether the queue is full), you will overwrite entries! With checking the \grqq fullness \grqq of the queue with a while-loop (look at code!) you wait, till you are allowed to push the data to the queue. This ensures, that you don't overwrite entries. However your system is stuck during that time and since the sensor datarate with 50Hz is faster than the sending, data packets are lost! So you should try to adapt the sensor data rate to the sending rate. Taking one measurement out of 5(case with $ f_{sens} \gg f_{ble} $) is less good, than taking one out of 2(case for $ f_{sens} \geq f_{ble} $ ). This could lead to an overall slower acquisition of packets (since you have to wait longer for you sensors), but the values from the sensor are more representative!}
\begin{figure}[!htb]
\centering
\includegraphics{pictures/plots/timingRFd_v2.png}
\caption[Timing behaviour of sensor system]
{200 acquisitions of complete measurement packets, comprising the values of four sensor units, where timed. One time the data rate of the sensor is set to \SI{50}{\Hz}, the other to \SI{25}{\Hz}. The mean value for $ f_{Sensor}=25\si{Hz} $ is higher than, for $ f_{Sensor}=50\si{Hz} $. However the variance of the acquisition time is smaller for this and therefore the outcome is more consistent. Further on the sensor values from a data rate of \SI{25}{\Hz} represent the actual obtained measurements slightly better, than the higher data rate. In the ongoing evaluation, the results for both data rates are taken into account and compared against each other.}
\label{fig:sensTime}
% python script: 160212_timing.py
\end{figure}

\FloatBarrier
\section{Quality of Calibration Procedures} \label{sec:cali}

\subsection{Calibration for Hard and Soft-Iron Effects}\label{subsec:resHardSoft}

Two methods are compared and classified for determining the hard- and soft-iron factors. The naive approach, declaring the distortion values by using the maximum and minimum of the obtained measurements for an axis. This one is chosen, since it is an often cited and easy method for compensating the distortion factors. On the other hand, the version from Freescale \cite{ozyagcilar2012calibrating} which takes a whole series of measurements into account and only compensates for the hard-iron effects. 1000 measurements were collected, by rotating the sensor slowly around all possible axes. The environment is a normal lab, without any protections against additional, artificial magnetic fields. 

In \ref{fig:hs2d} the measurements of each axis combination are plotted in a two dimensional representation, for a clearer identification. The raw values are represented by the red dots, the calibrated by the green and cyan ones. The outline of the corresponding ideal sphere is visualized by the blue circle. It can easily be observed, that the hard iron effects dominate the soft iron factors. The scaling factors for the soft iron values, obtained by the naive approach also reflect this. They lie in the range of $ 1 \pm 0.03 $. Another fact is, that both calibration methods lead almost the same results. In order to compare the quality of the two calibration methods, the distance of each calibrated measurement value to the perfect sphere with radius $ \mathrm{B}_{earth} $ is calculated. The deviation is plotted as a histogram in \ref{fig:devi}. The obtained mean for the Freescale approach is calculated to be $ \mu_{Freescale} = -0.02\si{\milli \tesla} $, for the calibrated values, using the naive method to be $ \mu_{naive} = -0.8\si{\milli \tesla} $. So in the end the values readjusted by the Freescale approach represent slightly more the shape of a perfect centered sphere. One reason for this is, that for the utilized sensors the hard iron distortion effects dominate over the soft iron ones. Further on, since the whole measurement series is taken into account, the behaviour of the sensor is represented much better. The naive approach is very sensitive for noisy signals, since only the peak values characterize the calibration factors. One has to note, that those observations hold only for the used sensor units. For another \ac{PCB} environment or device, the obtained values could be different and the soft iron factors could show a higher influence. For this, the naive approach would probably lead to better results than the Freescale. So in the end the calibration has to be verified and adjusted for the specific sensor and application.\\
The presented procedure was evaluated for several times and sensors, each time showing similar and constant results. As already mentioned in \ref{subsec:hardSoft} this calibration procedure has to be performed for each sensor and the observed values have to be additionally scaled to a common value for $ \mathrm{B}_{earth} $.
\begin{figure}[!htb]
\centering
	\subfloat[Obtained measurements along the $ x $ vs. the $ y $ axis]
	{\includegraphics{pictures/plots/cali_xy.png}\label{fig:xy}}
%	\hfill
	\subfloat[Obtained measurements along the $ y $ vs. the $ z $ axis]
	{\includegraphics{pictures/plots/cali_yz.png}\label{fig:yz}}
	\hfill
	\subfloat[Obtained measurements along the $ x $ vs. the $ z $ axis]
	{\includegraphics{pictures/plots/cali_xz.png}\label{fig:xz}}
\caption[2D representation of the raw calibration measurements]{
The measurements where recorded by rotating the sensor around each axis in an environment without artificial magnetic sources. 1000 measurements were collected. The obtained raw values are represented by the red dots. The results of the calibration procedures are plotted by the respectively color. The perfect centered sphere is represented by a circle with $ r=\mathrm{B}_{earth} $. Already the unscaled values show only a very low influence of soft iron distortion. It is also observed, that the calibrated results do not differ much.}
\label{fig:hs2d}
% python script: checkCalibration.py
\end{figure}
\begin{figure}[!htb]
\centering
\includegraphics{pictures/plots/cali_devi.png}
\caption[Deviation of the calibrated measurements to the perfect centered sphere]
{The procentual deviation of the two calibration methods to the corresponding perfect centered sphere is visualized. It can be obtained, that the Freescale approach leads to slightly more accurate results, since the mean and variance of the scaled measurements are smaller than for the naive approach.}
\label{fig:devi}
% python script: checkCalibration.py
\end{figure}
%\todo{make statement about the range!}

%In \todo{3d} the calibrated and the uncalibrated data points are plotted, with a sphere around them. One thing that is clearly visible here is the shift, caused by the hard iron effects. In \todo{2d} this offset is also observable. Recognizable by those two plots is also the fact, that the influence of the soft-iron distortion is very small. The data points lie already on an almost perfect sphere. This is also represented in those scale values, calculated by the Winer approach. They lie in the range of $ 1 \pm 0.03 $. That both calibration methods serve almost the same results can already be seen by the fact that the blue and green datapoints are overlapping each other. In order to compare the two calibration methods, the distance of each calibrated value to the perfect sphere with radius $ \mathrm{B}_earth $ is calculated. The outcome is shown in \todo{deviation}. In the end, the mean value of the deviation for the Freescale approach is smaller than those, calibrated by the Winer method (for this particular example: $ \mu_{Freescale} = -0.0842\si{\micro \tesla}, \mu_{Winer} = 6.5525\si{\micro \tesla} $). Because the hard-iron offsets dominate the soft-iron effects, the Freescale method is more accurate than the Winer approach, since it takes all the observed measurements into account and not just the minimum and maximum values. Therefore for the utilized sensor units, which show small soft-iron deviation, the Freescale approach is preferred. The provided procedure was evaluated several times, each time showing similar and constant results. As already mentioned in \ref{subsec:hardSoft} this calibration procedure has to be performed for each sensor and the observed values all have to be scaled to a common value for $ \mathrm{B}_{earth} $.


%\subsection{Determining the Fitting Parameters for the Model} \label{subsec:resModelFit}
%
%\todo{Rewrite!}
%For determining the scaling factors to the model, the sensor rack is placed onto a cardboard box with a height of \SI{2}{\cm}. The magnet is statically aligned in the same direction as the sensor $ x $-axis and is moved along the y-axis on a flat surface. The distance in $ x $-direction is held static. The origin of the coordinate frame is determined to be the position of the upper most sensor unit. The magnet is moved at a distance of $ x=5\si{\cm} $ and $ x=7\si{\cm} $ from $ y=-6\si{\cm} $ (which is the $ y $-height of the under most sensor) to $ y=0\si{cm} $ (which is the $ y $-heigth of the upper most sensor). For defining the scaling factors, the B-field, which should be observed by the sensor is calculated. The range of the actually measured values is then fitted to the simulated ones for each axis. As a initial step, the surrounding magnetic field has to be determined and cancelled statically. The scaling factors for the two different $ x $-positions should lie around 1 and be the same. However this is not the case. The obtained factors are listed in \ref{tab:scaleValues}. As expected, the values are near to one. The mean over all factors is $ 1 \pm 0.15 $. The influence of the two scaling factors on the de facto obtained measurements and the calculated values is visualized in \ref{fig:flatFit} for $ x=7\si{cm} $.
%\begin{table}[h]
%\centering
%\begin{tabular}{c c c c}
%\toprule
%distance in &  &  &  \\ 
%$ x $-direction [cm] & $ x $ & $ y $ &  $ z $ \\ \midrule
%5 & 1.21 & 1.26 & 1.13 \\
%7 & 1.09 & 1.18 & 1.04 \\ \bottomrule
%\end{tabular}
%\caption{The calculated scaling factors, for fitting the range of the sensor values to the one, predicted by the model. One can see, that the scaling factors are very small. This is reasonable, since the sensors are calibrated. However due to human interaction, not perfectly calibrated sensor values and the offset of the earth magnetic field, the scaling factors are not equal to 1. The mean over all 6 factors is 1.15, the standard deviation is 0. Further on it is observed that the factors for two different distances in $ x $-direction also lead to distinctive scaling values. The overall difference between the norm of those scaling vectors is 0.17.}
%\label{tab:scaleValues}
%\end{table}
%\begin{figure}[!htb]
%\includegraphics{pictures/plots/flatFit.png}
%\caption{Comparison between the scaling factors, observed from $ x=5\si{\cm} $ and $ x=7\si{\cm} $. The simulated data is for a distance of $ x=7\si{\cm} $. The $ y $-position to the corresponding time step is plotted in cyan. Please note, that the measured values cannot be compared directly to the results of the simulation. The simulation assumes a constant movement of the magnet. This however cannot be performed perfectly, since it is done by hand. Due to this fact, also the distance in $ x $ direction can slightly change. The plot reflects the differences between the two different scaling factors and the raw values for each sensor axis. In this way it can be observed, that the results show the most variation along the magnetic $ x $-axis. This is only reasonable, since on this axis the magnetic flux density changes the most.}
%\label{fig:flatFit}
%% python script: 160201_flatSensFit.py
%\end{figure}
%Especially the plot shows, that this deviation between the results is not negligible. The highest difference between model and measured values is observable along the magnetic $ x $-axis. Since the biggest change happens for the utilized motion on this axis, also the biggest impact of the scaling factors can be obtained here. An exact difference value between the perfect B-field values from the model and the actually observed and scaled ones is not directly possible. The values of the model assume a movement with a constant velocity. Since the magnet motion is performed by hand, the measured field is not changing constantly. However for this reason, the magnet is held still at some dedicated $ y $-positions, to get an impression on the difference of the scaled measurements to the calculated field. This is represented by the flat sections in the graph. The deviation from the perfect values for those positions is calculated and normed, to get a representation of the difference to the overall magnetic field strength. \ref{tab:diffScaled} presents the results.
%\begin{table}[h!]
%\centering
%\begin{tabular}{c c c c}
%\toprule
%\multirow{2}{*}{y-Position [\si{cm}]} & scaled to  & scaled to  & \multirow{2}{*}{raw [\si{\micro \tesla}]} \\ 
% & $ x=5\si{cm} $ [\si{\micro \tesla}] & $ x=7\si{cm} $ [\si{\micro \tesla}]  & \\ \midrule
%-6    & 0.0  &  0.0  & 0.0 	\\ 
%-4    &	0.94 &  2.5  & 3.7 	\\ 
%-2    &	2.5  &  2.6  & 5.7 	\\ 
%0.0	  &	6.0  &  0.0  & 6.3 	\\ \bottomrule
%\end{tabular}
%\caption{The deviation of the scaled magnetic field to the simulated at the stopped positions. The observed values show, that an exact fitting, by scaling only for the maximum and minimum values for a certain movement does not lead identical and exact results.}
%\label{tab:diffScaled}
%% python script: 160201_flatSensFit.py
%\end{table}
%As a source for the observed disagreement, one could head that the whole procedure is performed by a human being. The position values are only determinable up to a certain amount of accuracy, as well as the start and end points of the performed movement. Further on the already stated errors, induced by the hard and soft iron calibration routine have also an impact, which is not determinable. Also the surrounding magnetic field, though it is measured and subtracted before the measurements started, cannot be eliminated totally and has an impact. Since this calibration procedure does not result in clear scaling values for the sensors, it is not further used in this work. 
%
%So in the end, for scaling the observed sensor values to the model, a predefined fitting gesture is used, as described in \ref{subsec:modelFit}. As stated, this approach does not only compensate the slight errors, induced by the calibration procedure, it mainly adjusts the observed magnetic field to the provided hand and sensor positions. As described Since those values are only definable by a certain grade of accuracy by hand and  

\FloatBarrier

\subsection{Elimination of Earth Magnetic Field} \label{subsec:resEarthEli}

Since a constant elimination of the earth magnetic field would be very important for a portable system, two methods of the approach, presented in \ref{subsec:earthEli} are tested. The difference between those two lies in the determination of the sensor orientation. The one estimates it by using an implementation of a Madgwick Filter, provided by \cite{mikeshub2012}. This algorithm can directly be executed on one sensor device, since the accelerometer and the gyroscope are already on the breakout board. So for this method, no additional sensors have to be mounted onto the sensor bracket. The other approach uses an additional \ac{IMU}, which can output the orientation directly as quaternion. The MPU9250 from Invensense \cite{MPU2014} is used for this. This second method could lead to more exact orientation measurements, since the quaternion is calculated internally by the measurement unit itself. The orientation of the magnetometers relative to each other does not change, since they are placed inside the self designed bracket. Therefore it is sufficient, to determine the orientation of the sensor rack. For the implementation, follow the steps presented in \ref{subsec:earthEli}. As an intermediate step, the calculated relative orientation $ R_{d} $ of both methods was inspected and was proven to represent the truth.

As an early observation, the approach using the Madgwick filter is considered bad. Since the readings of the magnetometer are used, for guaranteeing a stable and non-drifting estimation of the orientation, the artificial magnets interfere this algorithm. This was observed by a constant drift of the values over time, when introducing the artificial magnets. So the further verification was only performed with the MPU9250 sensor, with whom this drift behaviour was not observed. Nevertheless it is mentionable that the upcoming results for cancelling solely the earth magnetic field (in absence of artificial magnets) were similar for both methods (beside the mentioned sensor drift). A proper working system should constantly return a magnetic field of almost \SI{0}{\tesla}, when it is rotated in an environment without artificial magnets. In order to verify this, the sensors are slowly moved around each axis. By comparing the results with and without the subtraction of the initially observed magnetic field, one should get an impression on the quality of the algorithm. In \ref{fig:earthCancelRes} the observed data of each axis is displayed.\\
\begin{figure}[!htb]
\centering
\includegraphics{pictures/plots/earthCanc.png} 
\caption[Quality of earth cancellation]
{The result for the cancellation of the earth magnetic field, relative to the sensor rotation is displayed. One change in orientation is performed, to observe the capability of constantly subtracting the initially observed surrounding magnetic field. The plot is divided into three orientation areas. The initial orientation at the beginning and end and a rotated orientation in between. That the surrounding magnetic field can be cancelled is shown if the sensor is orientated as introductory. However the large deviations from \SI{0}{\tesla} during the movement in between the two orientation areas show, that the implemented approach does somehow not work for every rotation. As comparison, the raw values without subtracting the rotated surrounding magnetic field are also plotted.}
\label{fig:earthCancelRes}
% python script: 160217_earthCancel.py
\end{figure}
The plot shows the measured magnetic field along all three axes for one sensor with and without subtracting the rotated initially observed field $ \mathrm{B}_{earth} $. It is visible, that unfortunately the elimination method does not work properly. At the beginning, the B-field with the cancellation is 0. However by rotating the device, the observed field changes a lot. For the results along the x-axis, the offset can be compensated relatively good. But for the values observed along the y- and z-axis, this does not hold. Moving the sensor in its starting position again, it can be seen that the surrounding field is eliminated pretty well again. So the dynamic behaviour of the utilized approach is very bad. This short example visualizes only a simple movement around one axis. Even for this, the surrounding magnetic field can not be eliminated. Small changes could be claimed upon calibration errors or small static magnetic sources in the environment, such as cell phones or metallic objects. But the observed deviation from 0 is much higher than this. So in the end, the surrounding magnetic field can not be cancelled with the presented method. Further investigation has to be done for this. However, since this work focuses on the evaluation for pose estimation with magnets, the cancellation of the earth magnetic field is left by that. For the ongoing evaluation, it has to be noted, that the sensors and therefore the hand is always held static, to determine the influence of surrounding magnetic fields at the beginning and subtract it statically.

\FloatBarrier
\section{Evaluation of the Magnetic Field Models} \label{sec:modelDif}

In order to verify the two introduced models for describing the magnetic field of a cylindrical bar magnet with real measurements, a simple movement is inspected. The sensor is placed at the origin and the magnet is moved along its $ x $-axis. The motion is performed from a distance of $ x=6\si{\cm} $ to $ x=13\si{\cm} $. The cylindrical model represents the ground truth for the utilized case of a bar magnet. Remember that we want to measure in this simple case the influence along the axis of magnetization. Therefore the complex cylindrical formula is reduced to the common known \ref{eq:b_z}. The results are plotted in \ref{fig:modCompFlat}.
\begin{figure}[!htb]
\centering
\includegraphics{pictures/plots/compX.png}
\caption[Comparing the models and sensor measurements for flat movement]
{The magnetic field, for increasing the distance in $ x $-direction from sensor to magnet. The values are calculated once by the both magnetic model equations. The plot in the middle shows the difference of the measurements and the dipole model to the exact values of the cylindrical model. One can observe, that the error of the dipole model decreases over the distance. The error of the measurements should not be overestimated, since the movement is performed by hand and a constant change in the distance $ x $, as assumed by the models, is not perfectly performable. This can be seen by the lower plot, where the position to the corresponding observed magnetic field values is plotted. Further on the accuracy of the performed movement is also not perfect. The small deviations from to the predicted values show however, that both models represent the truth.}
\label{fig:modCompFlat}
% python script: 160221_modelComp.py
\end{figure}
Note, that the offset due to the surrounding magnetic field of the measurement data is removed beforehand. The plot shows pretty well that the two magnetic field models have the same behaviour and can represent the measurements quite good. The dipole model serves as a reasonable approximation. As the influence of the magnet goes further away, also the deviation from the cylindrical model decreases. The maximum error is \SI{0.008}{\milli \tesla}. For the measurements, the highest deviation is observed to be \SI{-0.045}{\milli \tesla}. Note that the sensor values suffer from a not perfectly consistent moving speed and accuracy restrictions. This can be seen, by regarding the positions to the corresponding observed magnetic field. The by hand moved magnet shows a bit of a \grqq delay\grqq, compared to the cylindrical model. Therefore the observed differences from the simulated values should not be overweighted.


\FloatBarrier
\section{Evaluation of the Human Hand Model} \label{sec:evalHand}

A similar verification is done for classifying the measurements directly on the hand. Exemplary the bending of the \ac{MCP} joint about \SI{90}{\degree} of the index finger is evaluated. For calculating the expectable magnetic field at a single sensor unit for this movement, excited by wearing a single magnet on the fingertip, the hand parameters and positions for the corresponding bone lengths, sensor and knuckles of the hand have to be determined. As already mentioned beforehand, those values are determined by hand with a calliper, what in turn introduces deviations from the actual real anatomic dimensions. So the predicted magnetic field calculated by the model equations is expected to show slightly different values as the actual sensor measurements. For the simplest case of foreseeing the values, obtainable by a single sensor unit for one finger, the values to determine are the following: 3 bone lengths, one 3D position for the joint and one for the sensor, which makes in total 9 anatomic parameters. How good they can be measured and which influence they have on the predicted magnetic field is visualized in \ref{fig:measHand}a. Please mind again, that the actual behaviour over time should not be overestimated, since the model assumes a constant motion velocity, which in turn cannot be achieved by a human. In the end, the difference between measured and predicted values should show a similar behaviour as for the easy case, presented in \ref{sec:modelDif}. Beneath the predicted and de facto measured magnetic flux densities, the normed difference over the three dimensions is displayed. This should give an overall measure for the deviation of the measurements and the model results.\\
\begin{figure}[!htb]
\centering
\includegraphics{pictures/plots/lenFing.png}
\caption[Influence of erroneous hand dimensions on model predictions]
{The difference between the model prediction of the magnetic flux densities and the actual measured sensor data can be seen in figure a. The values are calculated by using the hand measured anatomic parameters for the bone lengths, the joint and sensor positions. The high deviation from the measurements is not negligible. Passing the sensor values as is into the estimator for the finger state vectors would not lead to the representative angles. In figure b, the sensor values are naively scaled to the model prediction. With those measurements, adapted to the introduced fitting gesture, reasonable results for the state estimation are expected.}
\label{fig:measHand}
\end{figure}
One can see, that the predicted and the measured values show a similar behaviour. But in the end, there is a high difference between them, especially for the values along the $ x $ axis. The highest observable divergence for the presented measurements and the applied anatomic dimensions lies at \SI{0.019}{\milli \tesla}. This shows that the hand determined parameters do not resemble the proband's hand anatomy good enough. Passing those sensor values to the minimization algorithm for the hand pose estimation, no good results would be observed. The obtained magnetic field is not representable with the model equations, using the underlying hand dimensions. However by remeasuring and adjusting the 9 dimensional parameters by trial and error, a set of satisfying values can be found. For the introduced easy case, a more reasonable magnetic field gets predicted by shrinking only the length of the proximal bone by \SI{3}{\cm}. A measurement error of such a size is in turn not possible. Therefore to achieve a reasonable error compensation, all 9 dimensional parameters should be changed only up to a size of around \SI{1}{\cm}, which seems a more reasonable measurement error. However performing this by trial and error did not lead to reasonable results and in the end would not be an option. Especially regarding the overall goal to estimate multiple sensors with several magnets and introducing therefore even more anatomic parameters. Also attempts to estimate those parameters did not lead to reasonable results. For the simple presented case for one magnet and sensor, one would have to estimate 9 anatomic parameters, with only a set of 3 measured magnetic flux densities. Additionally, one does not know the exact finger angles to the actual measurement and therefore one would also have to estimate the finger state vector. This brings in too much \ac{DOF} and in the end the anatomic parameters can only be determined up to the non-satisfying accuracy.\\ 
A naive possibility to achieve at least a satisfying relationship between the measurements and the model predictions would be to scale the obtained measurements along each axis to the respective predicted values. Of course, by this method the nonlinear influence of each of the dimensional parameter is dropped. However, \ref{fig:measHand}b shows that the difference can be reduced very efficiently for the introduced gesture of bending the \ac{MCP} joint about \SI{90}{\degree}. The maximum error between the measured and the predicted values is now \SI{0.005}{\milli \tesla}. One could state, that this error is induced by the not constantly performed motion. Of course, the obtained scaling factors for each axis are only adapted to the performed gesture. However, it comprises the most expected movements, being the bending of flexion-extension. Another pose, to fit the measurements to, would be the bending of the fingers to a fist. Here, also the errors, introduced by the bone lengths could be compensated. Passing those scaled values to the estimator for the finger state vectors, would lead in the end to reasonable results. The suitability of the introduction of a fitting gesture, to naively scale the measured values to the error-prone anatomic parameters has to be evaluated with a real measurement set.


\FloatBarrier
\section{Expectable Magnetic Flux Densities}

% Overall feeling about expectable magnetic fields...
By looking at the overall observed field for the individual measurement axes, one could get an impression on the forthcoming observable magnetic flux densities (for a recall on the underlying cartesian coordinate frame, please look at \ref{sec:handModel}). For a more natural picture of the observed magnetic field values along each axis, one could think of the three dimensional values as a vector in space, pointing from the magnetic south to the north pole. The sensor is almost directly beneath the \ac{MCP} joint, and since no adduction or abduction is performed, the change on the sensor $ y $-axis is small (for this example \SI{0.01}{\milli \tesla}). For a sensor showing a bigger offset in $ y $ direction from the joint, the influence on this magnetic axis would also be higher. A more observable difference can be seen on the $ x $- and $ z $-axis. Since the gesture starts with a stretched finger, the biggest value for the magnetic flux density is measured on the $ x $-axis. By bending the \ac{MCP} joint around \SI{90}{\degree}, the influence on the $ x $-axis decreases and on the $ z $-axis increases. Remember, that the position of the magnet is moved towards the negative $ z $ direction, therefore also the observed values for this axis are negative. Another observation, that can be made by this simple example is the range of the expected measurements. For these specific finger lengths and sensor positions, the observable values lie in a range of $ \pm 0.3\si{\milli \tesla} $. This is a small range, especially compared to the range of the surrounding magnetic field, observed in \ref{subsec:resHardSoft}. Therefore a constant elimination of this disturbing field would be very important. As already stated, the introduced example shows the movement of a single \ac{MCP} joint. By introducing multiple magnets on other fingers and performing a more involved gesture, like a fist, the expected values will vary more. Especially during the movement to a fist, the magnet on the fingertip gets nearer to the sensor again and therefore the influence increases. In order to get an impression for the anticipated magnetic values for different movements and combinations of magnets, some sequences are simulated and plotted in \ref{fig:compMovement}.\\
\begin{figure}[!htb]
\centering
\subfloat[Bending only the \ac{MCP} of the index finger. The values for the sensors beneath the index and the pinky finger are displayed. Only one magnet on the index fingertip is utilized.]
{\includegraphics{pictures/plots/bInd.png} \label{fig:indComp}} 
%\hfil
\subfloat[Simulating the maximum movement of adduction-abduction of the index finger. The measurements, observed by the sensor beneath the index finger are plotted. Only one magnet on the index fingertip is utilized.]
{\includegraphics{pictures/plots/bInd_A.png} \label{fig:ad-abComp}}\\

\subfloat[][Bending the \ac{MCP}, \ac{PIP} and \ac{DIP} of the four fingers simultaneously about \SI{90}{\degree}. The measurements,\\observed by the sensor beneath the middle and the pinky finger are plotted. All four fingers are\\equipped with magnets.]
{\includegraphics{pictures/plots/bFist.png} \label{fig:fistComp}}	
%\includegraphics{pictures/plots/fingSim.png}
\caption[Simulating the magnetic field for various finger movements]{The anticipated values for various sensor positions, predicted by the cylindrical model for wearing one magnet on the index finger (a, b) and four magnets (c). It is observable, that the predicted fields are distinguishable among the different sensors. Even the slight movement of adduction-abduction causes a remarkable influence.}
\label{fig:compMovement}
%script: 160221_anglePlot.py
\end{figure}
With \ref{fig:indComp} one can compare the influence of the sensor position on the observable magnetic field. The values, expected for the sensor beneath the index finger show higher differences, than for the one under the pinky finger. This is induced by the offset between the $ y $ positions of the two sensors and the movement direction of the fingertip. Comparing this to the value, which can be observed by the sensor, placed under the \ac{MCP} of the pinky finger, a greater change can be measured. This is only logical, since the offset in $ y $-direction is also bigger for this sensor. Further on the values, observed by a sensor with a greater distance to the studied magnet also show less influence. By \ref{fig:ad-abComp} the influence of the maximum achievable adduction-abduction movement of the stretched index finger is visualized. Here, the main change of the magnetic flux density can be observed along the $ y $-axis of $ s_{Index} $. Since the movement happens only in the $ x-y $ plane of the sensor, this is just reasonable. As a last example, the movement of the fist by all four fingers, each being equipped with magnets on its fingertip is plotted in ref{fig:fistComp}. Note, that the values for the \ac{MCP}, \ac{PIP} and \ac{DIP} are increasing simultaneously for each finger at the same time. The values, observed by $ s_{Middle} $ and $ s_{Pinky} $ are plotted exemplary. As mentioned beforehand, the values, especially observed along the $ x $- and $ z $-axis first go into the negative direction and then increase to the positive. This is because the magnet first is moved \grqq away \grqq by the motion and then gets nearer to the sensor units again. Along the $ y $-axis, once more the influence of the sensor position is detectable. The unit beneath the middle finger is influenced by magnets to the left (positive $ y $-direction) and to the right (negative $ y $-direction). The one beneath the pinky finger has only magnetic influences to the left of it. This is why the curve for the observable magnetic field along the $ y $-axis is in the end slightly increasing for $ s_{Middle} $ and decreasing for $ s_{Pinky} $. Another influence on this behaviour are the lengths of the bones, and therefore the overall distance, determined by the fingers itself. So for another constellation of hand parameters, the observed values could decrease or increase. Further on, one could note, that the overall observed magnetic flux density by four deployed magnets is around 10 times higher, than measuring only a single magnet.\\
By the introduced sensor rack, a constant localization of the measurement units relative to each other and to the \ac{MCP} joints is given. Therefore the observable behaviour and influence of the individual magnets on each unit is constant and serves as a characteristic. So by comparing the differences between each observed sensor measurement, one could make a first statement about the finger positions. This point is important for finding a suitable solution to the optimization problem and therefore for the pose estimation. By introducing more sensor units at various static positions, one would expect to get better results for the estimated pose, since each pose causes an individual magnetic field at each sensor. Last but not least, this is why the group of Ma et al. is using 6 sensors in total, to estimate the position of a single magnet. Since the goal of the underlying thesis is to utilize a flexible and wearable system, only four sensor units are used. The presented claim is proven throughout simulation and real measurements. The results are given in the following chapters of this work.

\FloatBarrier
\section{Pose estimation} \label{sec:estimationRes}

\subsection{Identification of the minimization process} \label{subsec:resSim}

\subsubsection{Utilized minimization methods} \label{subsubsec:miniMethod}

The size and complexity of the minimization problem, described in \ref{sec:estimation} is dependent on the number of exploited sensors and magnets. The beforehand introduced minimization problem \ref{eq:minimization} is stated here once again for clarity:
\begin{equation*} \label{eq:minimization}
\begin{aligned}
\underset{\mathrm{X}_K}{\text{minimize}} & & f(\mathrm{X}_K) = || \tilde{\mathrm{M}} - \mathrm{M}(\mathrm{X}_K) ||\\
\text{subject to} & & 0 & \leq {x}_1(\theta_{MCP}) \leq & 1/2 \cdot \pi, \\
				  & & 0 & \leq {x}_1(\theta_{PIP})  \leq & 110/180 \cdot \pi, \\
				  & & -30/180 \cdot \pi & \leq {x}_1(\phi_{MCP}) \leq & 30/180 \cdot \pi, \\
				  & & 0 & \leq {x}_2(\theta_{MCP})  \leq & 1/2 \cdot \pi, \\
				  & & \vdots \\
				  & & -30/180 \cdot \pi & \leq {x}_K(\phi_{MCP}) \leq & 30/180 \cdot \pi
\end{aligned}
\end{equation*}
Remind, that the overall size of the observable measurements $ \tilde{\mathrm{M}} $ is $ (3 \cdot N) \times 1 $ (with $ N $ being the number of sensors, taken into account) and the size of the system state X is $ (3 \cdot K) \times 1 $ (with $ K $ being the number of fingers/magnets to describe). In order to gather a fully determined system, the number of sensors taken into account has to be at least as high as the number of magnets. This means, trying to estimate the state of four fingers with only one sensor could lead to ambiguous and bad results. Further on, the problem is nonlinear, which restricts the selection of the minimization method. The problem can be solved by applying the anatomic constraints as bounds or not. The SciPy package comes with the \emph{minimize} function, which is especially for solving scalar minimization problems. It can be invoked with different algorithms and their corresponding additional options. Since the cylindrical model is a numerical approximation, the derivative can not be evaluated. Therefore the desired minimization algorithms need to approximate it by there own.

The following explanations should give a short overview on the principle of the utilized methods and why they were chosen. For further reading on numerical optimization methods, please have a look at \cite{nocedal2006numerical} (on which the following paragraphs are based).\\
For solving the problem without taking the anatomic bounds into account, the \ac{BFGS} algorithm is used. It is an approximation of Newton's method, for finding a solution. Newton's method describes derivative based approaches, to find local minima around a certain initial guess $ \mathrm{X}_{0} $. To find values for $ x $, which minimize the outcome of the function $ f $, different search methods exist. The line search approach tries to find the local minimum along a line, which is determined by the Jacobian $ \nabla f $ and Hessian $ \nabla^{2} f $. Since the \ac{BFGS} approximates the derivative  $ \nabla f $, it is called quasi-Newton. $ \nabla f $ is updated at every iteration. An iteration step consists of finding a value $ x_{k+1} $, which minimizes $ f $. This is done till the gradient norm $ || \nabla f|| < \epsilon ||$, with $ \epsilon $ representing the convergence tolerance. In other words, a solution is found, if the change in the value of $ ||\nabla f|| $ is smaller than $ \epsilon $. As a characteristic of the \ac{BFGS} method, only the first derivative needs to be approximated. The rate of convergence for the method is stated to be linear. Further on it is assumed to be robust. The \ac{BFGS} implementation of SciPy shows very good results for the default values for $ \epsilon = 1.5e-08 $. The overall termination tolerance, defining the magnitude of $ f(\mathrm{X}_K) $ is denoted to be $ 1.0e-07 $. Shrinking this value, would lead more exact results, but would also induce more iteration steps and therefore a higher computation time. The results for the presented value show a good trade off between time and accuracy.\\
For solving the problem by taking the anatomic conditions into account, SciPy provides a method called \ac{SLSQP}. The constraints can be passed in as a pair of $ (min,max) $ for each variable, and reflect hard bounds. The underlying principle is based on least-squares methods. Therefore the system has to be overdetermined or at least fully determined. It tries to fit the observed data (i.e. the measurements) to a given model, by adjusting the model parameters. This is actually often used for data-fitting. While a system state is desired and the model comes with no additional parameters, the method is used in a slightly different way. In contrast to the classic approach, the model is fitted to the measurement data. The parameters in this case are the values of the system state $ \mathrm{X}_K $. In the least squares sense, the sum of the errors between the model at state $ \mathrm{X}_{0} $ and the measurements is squared and minimized. Exactly this is expressed by the objective function $ f(\mathrm{X}_K) $. Again, a starting point $ \mathrm{X}_{0} $ has to be provided. For identifying the direction of $ x $ in each iteration step, Powells method \cite{powell1964efficient} is used. This derivative free approach identifies independent convergence vectors for each variable. It can be interpreted as the approximation of $ \nabla f $. At each iteration step, those search directions are defined and therefore the new system state can be expressed by a combination of them in turn. In order to bring in the constraints, $ f $ is modified to represent those restrictions as a non-negative least squares problem. As the name suggests, the restriction to the system state is the following $ \mathrm{X} \geq 0$. Those reformulations are done by the SciPy method, therefore no further adjustments to the model or the bounds have to be made by the user. In the end the recursion gets performed, till the termination tolerance for $ f(\mathrm{X}_K) $ is fulfilled.\\
It should be mentioned, that for the implemented estimation routine, the initial guess $ \mathrm{X}_{0} $ is chosen to be the state, estimated by the minimizer one step ahead. Since the estimation assumes to start with stretched fingers, the overall $ \mathrm{X}_{0} $ is a vector of 0.  


\subsubsection{Classifying the methods, based on simulated data}

In order to get an impression on the expectable results of the minimization method, it is tested with a simulated dataset. A self chosen predefined set of states is determined, which should represent the motion of the fingers. This sequence of joint angles is simulated using the cylindrical model, to obtain the value of the expectable magnetic flux density, measurable by a specific sensor for the corresponding system state. The cylindrical model is used, since it represents the behaviour of the bar magnet and is not just a simplification, like the dipole model. Those values for the expectable magnetic flux density are then passed to the minimization routine, to estimate the system states. The result of the minimization should of course reflect the predefined motion sequence. Therefore it can directly be compared to the known state values, to identify the quality of the solver and its behaviour.\\
There are several parameters for formalizing the estimation problem and to tune the solver:
\begin{itemize}
\item Expressing the minimization as an unconstrained (by using the \ac{BFGS} algorithm) or constrained (by using the \ac{SLSQP} algorithm) problem
\item Considering the influence of the movement of adduction-abduction or not.
\item Formalizing the objective function using the cylindrical or the dipole model.
\item The behaviour regarding different determinedness of the system, which means estimating the states of one or multiple fingers by taking one or multiple sensors into account.
\end{itemize}
The results will be compared by calculating the mean and standard deviation of the error-norm to the perfect system state for each finger. Further on the calculation times of the different methods can be compared.

%The with and without constraints. Further on, in this way the results of using one or multiple sensors for the estimation of the system states of one finger can be compared. In other words, the determinedness of the system can be varied and evaluated.
%Also the number of fingers to track can be changed and evaluated, to specify the capability of an increasing system complexity.
%The states are estimated once using the model with adduction-abduction and once without this state variable. This done, to evaluate how good this movement can be tracked and whether it shows an effect on the other system states. Further on, it can be evaluated how the system without regarding adduction-abduction behaves if such movements are actually performed and whether it would improve or worsen the overall results for flexion-extension.
%As a last characteristic, the states estimated by the objective function, using the dipole and the cylindrical model can be compared to each other. At the end of this section, one should have an impression on the capabilities and characteristics of the different models and how one would get the best results.

As a first step, the different optimization parameters are evaluated for the movement of a single finger. Therefore the size of the system state is $ \mathrm{X}_{1} = 3 $ for taking adduction-abduction into account and $ \mathrm{X}^\prime_{1} = 2 $ for neglecting this additional state. The size of the simulated measurements is dependent on the number of sensors, taken into account. The index finger is chosen for evaluating the different parameters, but the results are expected not to change, by choosing a different one. The utilized gesture sequence is displayed for the three states of the index finger in \ref{fig:indexStates}. The angular change, and therefore the stepwidth between two states is determined by combining the observations for the angular velocity from Ingram et al. \cite{ingram2008statistics} and the data rate of the sensor system. An acquisition frequency of \SI{20}{\Hz} in combination with a mean angular velocity of \SI[per-mode=symbol]{10}{\degree \per \second} leads to an observable maximum change of \SI[per-mode=symbol]{0.5}{\degree \per \second} (= \SI[per-mode=symbol]{0.0085}{\radian \per \second}). Therefore, the whole set is divided into 1419 datapoints, which results in a total duration of \SI{70.95}{\second} for performing the motion. The state values are plotted against time. The motion is constructed, to represent simple and complex movements of the finger, including flexion-extension, as well as adduction-abduction. The motion sequence includes joint movements, which happen as unique motions at a time. For example between \SI{0}{\second} and \SI{20}{\second} only the \ac{MCP} joint moves. Some, which arise together, like between \SI{40}{\second} and \SI{60}{\second}, where all three joints are performing flexion-extension. Also only small movements are simulated. Between \SI{38}{\second} and \SI{42}{\second}, $ \theta_{PIP} \; \text{and} \; \theta_{DIP} $ change only about \SI{0.26}{\radian}. The movement of adduction-abduction is applied during a short sequence, since the range of movement is small and also occurs more rarely, compared to flexion-extension. 
\begin{figure}
\includegraphics{pictures/plots/indexStates.png}
\caption{\todo{caption}}
\label{fig:indexStates}
% python script: 160224_plotSequence.py
\end{figure}
The obtained error means and standard deviations for each parameter combination are presented in \ref{tab:oneFing} in radians. The numbers in the very first column indicate the combination of fingers and sensors. The first number represents the estimated fingers, which is for this comparison always one, since only the states of the magnet at the index are estimated. The second number represents the amount of simulated measurements. By using only one simulated sensor reading, the one right beneath the index finger is meant. By taking two into account, the index and middle sensor are pointed. And four means that all four simulated units are regarded. The abbreviations in the second column reflect whether the movement of $ \phi_{MCP} $ is regarded or not. \grqq no ad-ab \grqq stands for no adduction-abduction movement and \grqq ad-ab \grqq for the opposite. \\
One thing, that can be observed directly, is that for the case \grqq 11\grqq which still represents a full determined system, the results show a very high deviation from the perfect values, regardless how the model is adjusted. The mean over all errors is \SI{0.289}{\radian}. The best observable values can be obtained by the method using the constrained cylindrical model and neglecting adduction-abduction. By regarding, that the inserted magnetic values were predicted by this model and that the overall system state is simplified, this seems reasonable. Further on the constraints restrict the algorithm not to drift to far away. \todo{figure 11cylNa1} shows the results for this best guess and the deviation from the perfect values over time. 
By deploying only one set of forecasted sensor values more, the results get much better. The mean of all error means is \SI{0.054}{\radian}. also the standard deviation is almost constant. One could even state, that by using all four simulated sensor units, the error does not decrease very much (the mean over all errors is \SI{0.045}{\radian}). Therefore it can be stated as a first observation, that the system has to be overdetermined. 
By comparing the error from the objective function using the dipole model with the cylindrical one, a decrease can be observed. As already stated for the \grqq 11\grqq case, this just seems reasonable, since the magnetic flux densities were calculated by the same. However for real observed measurements, this has to be further evaluated.
Also, while considering that the field values for estimation still comprise the movement of adduction-abduction and since the ability to estimate the system state with a reasonable accuracy, the neglecting of those values just results in worse results. The biggest difference to the perfect values occur here at the time, the adduction-abduction is performed. The remaining parts, where $ \phi_{MCP} = 0 $ are also almost perfect. For a visualization, the estimated values and their difference over time are plotted in \todo{figure 12or14cylNa1}. By looking at the difference between the results of the constrained and unconstrained methods a slight decrease of the error by regarding the constraints can be observed. The same reason, as mentioned beforehand can be named, which is, that the algorithm shows better convergence by the deployed constraints. In the end almost fault free results can be observed by the cylindrical model, which takes the movement of adduction-abduction into account. Here it does not count too much, whether the minimizer is constrained or not. \todo{concluding statement? what are bad results? which dimension of errors is acceptable?}

\begin{table}[h]
\centering
\begin{tabular}{l l c c c c}
\toprule
 & &          				\multicolumn{2}{c}{Unconstrained}          &		\multicolumn{2}{c}{Constrained}\\ \cmidrule(lr){3-4}\cmidrule(lr){5-6}
 & & 								Dipole   			   & Cylindrical 	 			 & 		Dipole 			& 		Cylindrical \\ \midrule[2pt]
\multirow{2}{*}{11} & no ad-ab    & $ 0.194 \pm 0.002 $ & $ 0.074 \pm 0.001 $ & $ 0.367 \pm 0.015 $ & $ 0.035 \pm 0.000 $ \\ 
					& ad-ab		 & $ 0.252 \pm 0.003 $ & $ 0.257 \pm 0.013 $ & $ 0.570 \pm 0.020 $ & $ 0.570 \pm 0.020 $ \\ \midrule
\multirow{2}{*}{12} & no ad-ab    & $ 0.124 \pm 0.001 $ & $ 0.094 \pm 0.001 $ & $ 0.052 \pm 0.000 $ & $ 0.035 \pm 0.000 $ \\ 
					& ad-ab		 & $ 0.071 \pm 0.000 $ & $ 0.000 \pm 0.000 $ & $ 0.058 \pm 0.000 $ & $ 0.000 \pm 0.000 $\\ \midrule
\multirow{2}{*}{14} & no ad-ab    & $ 0.112 \pm 0.001 $ & $ 0.098 \pm 0.001 $ & $ 0.040 \pm 0.000 $ & $ 0.033 \pm 0.000 $ \\ 
					& ad-ab		 & $ 0.042 \pm 0.000 $ & $ 0.000 \pm 0.000 $ & $ 0.038 \pm 0.000 $ & $ 0.000 \pm 0.000 $\\										
\bottomrule
\end{tabular}
\caption{\todo{Means and standard deviations! Write, that it is in radians!}}
\label{tab:oneFing}
\end{table}

\begin{figure}
\centering
\includegraphics{pictures/plots/difOne.png}
\caption{\todo{caption!}}
\label{fig:11cylNa1}
\end{figure}

In \ref{tab:timeOneFing} the mean time, needed for one estimation cycle is listed in seconds. It can be observed, that the time increases with the determinedness of the system. This is just logic, since the algorithm has more equations to take into account and to evaluate. Also the constrained methods show a faster timing behaviour, than the unconstrained. As a reason the restricted search space of the solver could be mentioned. The reduced system state by neglecting the adduction-abduction movement is also faster than the model, comprising this state, what is only reasonable, since one state variable less has to be estimated. The objective function, formulated with the dipole model shows also a faster evaluation time, compared to the one using the cylindrical. Since the cylindrical model represents a numerical approximation, which has to be evaluated at each iteration, the time consumption for evaluating is higher. The dipole model however consists of a relatively simple nonlinear equation matrix. So in the end the observed computation times are all reasonable. 
By comparing the quality of the solver with its timing behaviour, it can be stated that an increase in precision comes with higher computation times. For this example, using the perfect simulated data for the magnetic field, the estimation is not always fast enough, to match the observed sensor system frequency of \SI{20}{\Hz}. However it is evaluated, that the estimation results won't degrade drastically, if one or two measurements would be skipped, due to the computation time. The actually estimated system state is only used as initial starting guess for the next estimation. It is observed that the solver is capable, to intercept changes of a minimum of $ \pm 0.2 \si{\radian} $, between two measurements. So the initial starting point plays a not so important role for the solvability. For the assumed maximum angular velocity of \SI[per-mode=symbol]{0.175}{\radian \per \second} this change would reflect to a missing of one data set. What is more critical is the capability of estimating the state almost at real time. For the utilized simulated magnetic field values the best configuration for the minimizer to estimate the system state with an adequate frequency would be given by using the cylindrical magnetic model with adduction-abduction and taking the anatomic constraints into account. This would result in an estimation frequency of around \SI{7}{\Hz}, since the time needed to solve the problem is about \SI{0.148}{\second}. Compared to other hand tracking systems, this value is very high. However for getting a rough feedback on the actual finger state, this value should be sufficient. \\


\begin{table}[h]
\centering
\begin{tabular}{l l c c c c}
\toprule
 & &         			\multicolumn{2}{c}{Unconstrained}		 & 	\multicolumn{2}{c}{Constrained}\\ \cmidrule(lr){3-4} \cmidrule(lr){5-6}
 & & 								Dipole & Cylindrical & Dipole & Cylindrical \\ \midrule[2pt]
\multirow{2}{*}{11} & no ad-ab    & 0.037 & 0.077 & 0.008 & 0.017 \\ 
					& ad-ab		 & 0.089  & 0.119 & 0.029 & 0.037 \\ \midrule
\multirow{2}{*}{12} & no ad-ab    & 0.063 & 0.139 & 0.014 & 0.031 \\ 
					& ad-ab		 & 0.114 & 0.214 & 0.031 & 0.074  \\ \midrule
\multirow{2}{*}{14} & no ad-ab    & 0.110 &  0.251 & 0.025 & 0.059 \\ 
					& ad-ab		 & 0.216 & 0.409 & 0.056 & 0.148 \\										
\bottomrule
\end{tabular}
\caption{\todo{Estimation time! caption! Write, that it is in seconds!}}
\label{tab:timeOneFing}
\end{table}


For estimating the movement of multiple fingers, an adequate motion pattern is deployed. The simulated sequence consists only of 100 datapoints, reflecting a measurement time of only \SI{10}{\second}. This short time period is chosen, since first tests showed a very slow behaviour of the estimation stage. The utilized motion is visualized for each finger and each state in \ref{fig:multiFing}. As one can see, the finger are moving individually, to test whether the estimation is capable of that.
\begin{figure}
\centering
\includegraphics{pictures/plots/multiStates.png}
\caption{\todo{caption}}
\label{fig:multiFing}
% python script: 160224_plotSequenceMulti.py
\end{figure}
For getting an insight, how good the states for multiple magnets can be estimated, several sensor-magnet configurations are simulated. The evaluation is done for two fingers (the index and middle) and all four. As learned from the previous results, four sensors are used for the estimation of two fingers, to ensure overdetermindeness. However for estimating all four finger state vectors the introduced system can only satisfy determinedness. For reasons of completeness, four additional sensors were introduced to the simulation, placed behind the four existing ones. As for the estimation of one finger, the values are simulated using the cylindrical model. The results are listed in \ref{tab:multFing}. The corresponding parameters are typed in the same manner as beforehand. 

\begin{table}[h]
\centering
\begin{tabular}{l l c c c c}
\toprule
 & &          				\multicolumn{2}{c}{Unconstrained}          &		\multicolumn{2}{c}{Constrained}\\ \cmidrule(lr){3-4}\cmidrule(lr){5-6}
 & & 								Dipole   			   & Cylindrical 	 			 & 		Dipole 			& 		Cylindrical \\ \midrule[2pt]
\multirow{2}{*}{24} & no ad-ab    & $ 0.119 \pm 0.000 $ & $ 0.081 \pm 0.000 $ & $ 0.051 \pm 0.000 $ & $ 0.039 \pm 0.000 $ \\ 
					& ad-ab		 & $ 0.114 \pm 0.000 $ & $ 0.000 \pm 0.000 $ & $ 0.085 \pm 0.000 $ & $ 0.005 \pm 0.000 $ \\ \midrule
\multirow{2}{*}{44} & no ad-ab    & $ 0.941 \pm 0.006 $ & $ 0.484 \pm 0.001 $ & $ 0.314 \pm 0.000 $ & $ 0.216 \pm 0.000 $ \\
					& ad-ab		 & $ 1.361 \pm 0.022 $ & $ 0.024 \pm 0.000 $ & $ 0.223 \pm 0.000 $ & $ 0.140 \pm 0.000 $ \\ \midrule
\multirow{2}{*}{48} & no ad-ab    & $ 0.543 \pm 0.001 $ & $ 0.509 \pm 0.001 $ & $ 0.236 \pm 0.000 $ & $ 0.183 \pm 0.000 $ \\ 
					& ad-ab		 & $ 0.494 \pm 0.000 $ & $ 0.005 \pm 0.000 $ & $ 0.385 \pm 0.000 $ & $ 0.098 \pm 0.000 $\\										
\bottomrule
\end{tabular}
\caption{\todo{Means and standard deviations multiple! Write, that it is in radians!}}
\label{tab:multFing}
\end{table}

By looking at the results for the estimation of the state vectors for multiple fingers, a similar behaviour as mentioned for the case $ N = 1 $ can be obtained. \todo{Hier muss ich nicht nochmal das gleich schreiben oder?} However one interesting change can be observed. The unconstrained minimization method, described by the cylindrical model and taking $ \phi_{MCP} $ into account show here a better behaviour, than the constrained one. This is observable for each configuration of $ N $ and $ K $. One reason could be, that the \ac{BFGS} algorithm gives for those cases a better approximation for the search direction, than the constrained \ac{SLSQP} method. With the increasing number of system states, also the complexity increases. Therefore the constrained solver reaches its bounds, by using not good enough search directions. The unconstrained method however has more freedom, to look in each direction. For the estimation of two finger states, the unconstrained method using the cylindrical model with adduction-abduction serves the best result. For the estimation of all four fingers however, the minimization is not capable to reflect the perfect system states anymore. The overall smallest error for the estimation of four fingers with four sensors is \SI{0.024}{\radian}. So for the actually built system, comprising four sensors, an estimation of all four fingers will not lead to reasonable results. For getting an impression on the estimated states, compared to the perfect ones, those obtained values are plotted \todo{44CylAUc} However by introducing four additional sensors (case \grqq 48 \grqq), the results will get better. A mean error of \SI{0.005}{\radian} is observed by the unconstrained method, using the cylindrical model with adduction-abduction. However, as stated beforehand, the introduction of such a high number of magnets would break the goal of constructing a mobile and unobtrusive system. \\
By looking at the required estimation time of the several methods, a tremendous rise can be observed. To still observe reasonable results of the estimated states, more than \SI{1}{\second} is needed. This can be observed by almost all minimization configurations. This means a proper real time evaluation of the finger pose estimation is not possible anymore. By increasing the number of sensors $ N $ to 8, about \SI{17}{\second} would be needed to achieve reasonable results, which is obviously far away from real time behaviour or acceptance for post processing. To show the quality of the still best acceptable computation time, the estimated states for the fastest method for two finger state vectors is plotted in \todo{24dipNaC}. 

\begin{table}[h]
\centering
\begin{tabular}{l l c c c c}
\toprule
 & &         			\multicolumn{2}{c}{Unconstrained}		 & 	\multicolumn{2}{c}{Constrained}\\ \cmidrule(lr){3-4} \cmidrule(lr){5-6}
 & & 								Dipole & Cylindrical & Dipole & Cylindrical \\ \midrule[2pt]
\multirow{2}{*}{24} & no ad-ab    & 0.920 & 1.632 & 0.291 & 0.382 \\ 
					& ad-ab		 & 2.129  & 3.346 & 0.602 & 1.275 \\ \midrule
\multirow{2}{*}{44} & no ad-ab   & 3.365 & 5.012 & 0.629 & 0.947 \\ 
					& ad-ab		  & 8.322  & 8.419 & 1.696 & 2.684 \\ \midrule
\multirow{2}{*}{48} & no ad-ab    & 7.130 &  9.670 & 1.137 & 1.988 \\ 
					& ad-ab		 & 14.346 & 17.558 & 3.945 & 4.677 \\										
\bottomrule
\end{tabular}
\caption{\todo{Estimation time multi! caption! Write, that it is in seconds!}}
\label{tab:timeMultFing}
\end{table}

The presented results visualize the behaviour and influence of different system configurations $ N $ and $ M $ for different ways of describing minimization problem, based on perfect, simulated magnetic field values. The following concluding statements can be derived:
\begin{itemize}
\item The system has to be overdetermined, i.e. $ N > K $.
\item The estimation time increases significantly with the size $ K $ of the system state.
\item The estimation time increases in an acceptable size with the number of deployed sensors $ N $.
\item Therefore an estimation of four fingers with the designed system, consisting of four sensors is not possible in an adequate quality or real time behaviour.
\item The state $ \phi_{MCP} $ for adduction-abduction introduces higher estimation times, but can be estimated and should be used, to better reflect the human hand motion.
\end{itemize}
Since the results are based on perfect simulated values from the cylindrical model, the estimation procedures comprising this model lead also the best results. It is evaluated, that the cylindrical method, including adduction-abduction and anatomic constraints leads to the overall best results for the estimation of one finger. However when porting the observations to real measurements on a human hand, one has to note that several additional distortion factors are added to the system. Therefore in the ongoing estimation of real datasets, the cylindrical and the dipole method (both including the state $ \phi_{MCP} $ and constraints), are both used for the state estimation.





\subsection{Results for recorded/real data} \label{subsec:resMeas}

Describe the recording procedure

\begin{itemize}

%
\item \textbf{estimation of four fingerstates}
	\begin{itemize}
	\item very slow! And very poor results!
	\item 160210\_set6 nice for showing things for all fingers
	\item no good results without ad-ab (for 90 fitted gesture)
	\item moving together can a bit be tracked (initialization gesture reconstructed, for dip better than cyl)
	\item individual movement can not be tracked! (values 257:471)
	\item $ \rightarrow $ so no further details about four - four estimation, it is not possible!
	\end{itemize}

\item textbf{estimation one - one}
	\begin{itemize}
	\item results are \grqq a bit\grqq worse than with four sensors (should be valid for all sets, tested for 160217\_set3 (no ad-ab))
	\item results are much worse than with four sensors (160210\_set2)
	\item ad-ab can also be recognized (160217\_set4)
	\item \todo{more comparison of datasets!}
	\item estimation is ca. \textbf{two times} faster than one - four!
	\item e.g. four-one better than one-one 160210\_set4 (beginning of fist)
	\end{itemize}

%
\item \textbf{recognizing movements}
	\begin{itemize}
	\item not constantly \grqq good or bad\grqq, it differs and is dependent on beforehand movements (seen by almost all datasets) and movement of hand
	\item 160217\_set3 nice detection between 50s and 70s by cyl\_A (but directly afterwards, stretched position is not recognized...)
	\end{itemize}


\item \textbf{difference between model with and without ad-ab}:
	\begin{itemize}
	\item with ad-ab results are better (even for doing no movement in ad-ab)
	\item $ \rightarrow $ no ad-ab results over all are not so good...
	\item 160217\_set7: for simple movements almost no difference to with ad-ab
	\item 160210\_set1: big difference (no ad-ab worse...)
	\end{itemize}


\item \textbf{comparing results/difference for using cyl and dip}
	\begin{itemize}
	\item dip is much faster than cyl (around 7 times faster!)
	\item regarding accuracy, no clear statement possible...
	\item sometimes cyl is better, sometimes dip

	\end{itemize}

\item \textbf{detectability of ad-ab / comparing cyl\_A with dip\_A}
	\begin{itemize}
	\item 160210\_set4 shows that dip and cyl without ad-ab show only small differences!
	\item in the end: cylindrical estimation with ad-ab is more accurate than dipole with ad-ab! (160217\_set3)
	\item well it really depends... sometimes dip is better, sometimes cyl...
	\item but ad-ab can be estimated much better by cyl model, but only if finger is stretched! (160217\_set2)
	\item and only slow and clear ad-ab movements can be detected (160210\_set3 nothing is detected...)
	\item 160217\_set4 cyl\_A detects negative AND positive ad-ab!
	\item $ \rightarrow $ ad-ab is hard to detect. Even Leap is not capable
	\end{itemize}

\item \textbf{comparison to Leap}
	\begin{itemize}
	\item comparison is difficult, since it is also not perfect...
	\item 160210\_set1-2 are good examples, which show, that Leap is not perfect!
	\item 160210\_set2 however also shows, that it can work pretty good!
	\item values for ad-ab are not very good (e.g. 160210\_set3)
	\item 160210\_set4 also shows clearly (especially at the end), that Leap is not perfect!
	\item 160210\_set4 shows, that ad-ab can be detected quite good (70s-90s)
	\end{itemize}

\item \textbf{issues/observations with Leap}
	\begin{itemize}
	\item when finger are close together, tracking is harder
	\item DIP and PIP show the same constraint as I am using ($ PIP = 2/3 DIP $)
	\item angle in MCP introduces most times also angle in DIP/PIP (160210\_set2) \\
			$ \rightarrow $ my system is better for this!
	\end{itemize}

%
\item \textbf{influence of normed fitting gesture / use fist? and observed fitting values}
	\begin{itemize}
	\item scaling to flat scale fitting values leads to very bad results!
	\item fist is not good for fitting, since it is slightly different each time and everyone makes it different (pictures of videos and observed B-fields)
	\item (cardboard) has only slightly influence (dataset 160212\_set1)
	\item do it during the video (also with cardboard perhaps)
	\item because you change the height and orientation of your hand too much during calibration...
	\item so just perform the gesture during your recording
	\item fitting for ad-ab has almost no influence (dataset 160212\_set3), a clear, structured (performed alone) ad-ab can be detected (dataset 160212\_set4)
	\end{itemize}

%
\item \textbf{influence of exact hand dimensions/parameters}
	\begin{itemize}
	\item e.g. dataset: 160217\_set3 with handDim from 160210 and 160217 (but every other set should return the same...)
	\item for sPos there is a difference, but it is very small
	\item for bone lengths it's the same, only small differences (sometimes my own lengths better, sometimes wooden better)
	\item so in the end the parameters have to resemble the truth, but since the data is fitted to the calculated values, it's influence is not too big
	\end{itemize}

%
\item \textbf{influence of distance sensor to magnets}
	\begin{itemize}
	\item changes in magnetic field are too small...
	\item 160210\_set10 shows that the sensors are too far away! (the results within the methods vary a lot and they do not represent the truth!)
	\end{itemize}

%
\item \textbf{tries with MPU}
	\begin{itemize}
	\item as expected not good...
	\item start position can be \grqq detected\grqq after rotation again, but the results with rotation are bad...
	\item sets: 160217\_set5-6
	\end{itemize}

\item \textbf{results with ring}
	\begin{itemize}
	\item 160217\_set7 no difference to glued magnet
	\item 160217\_set8 very bad results... but I don't think that the ring is the reason
	\item \todo{do some more?}
	\end{itemize}

\item \textbf{for identifying/measuring the difference}
	\begin{itemize}
	\item plot the two results for two/several methods in one graph (colors)
	\item and plot numerical difference below
	\end{itemize}

\item \textbf{concluding observations}
	\begin{itemize}
	\item very fragile/sensitive system
	\item sensitive to hand/body movements, calibration gesture
	\item bad/difficult reproducibility of results/measurements
	\item quality of results is not totally comparable/identifiable by Leap (since this system is also faulty)
	\end{itemize}

\item 160210\_set4 nice for showing that Leap cannot detect DIP/PIP movement alone...

\item setting datarate to 25Hz has no effect... (160217\_set2)

\end{itemize}

Presentation and compare between EKF and minimizing approach



